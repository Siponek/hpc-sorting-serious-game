\chapter{Results and Evaluation}
\label{ch:results}

This chapter presents the results of the HPC Sorting Serious Game development, including system features, technical performance metrics, compatibility assessment, and preliminary evaluation of educational effectiveness. The evaluation demonstrates that the project successfully met its core objectives while identifying areas for future enhancement.

%-----------------------------------------------------------------------
\section{Completed System Features}
\label{sec:completed-features}

%-----------------------------------------------------------------------
\subsection{Functional Features}
\label{subsec:functional-features}

The implemented system includes all planned core features:

\paragraph*{Single-Player Mode:}
\begin{itemize}
    \item Configurable number of cards (10--200)
    \item Random card shuffling and generation
    \item Drag-and-drop card manipulation
    \item Private buffer zones for local sorting
    \item Sorting validation and completion detection
    \item Timer and move counter
    \item Victory screen with performance summary
\end{itemize}

\paragraph*{Multiplayer Mode:}
\begin{itemize}
    \item Lobby system with room creation and joining
    \item Support for 2--4 players
    \item WebRTC-based peer-to-peer networking
    \item Real-time state synchronization across clients
    \item Shared container visible to all players (OpenMP shared memory)
    \item Selective card visibility (private buffers hidden, simulating thread-private data)
    \item Disconnection handling and graceful degradation
    \item Host migration capability (partial)
\end{itemize}

\paragraph*{User Interface:}
\begin{itemize}
    \item Main menu with mode selection
    \item Settings for game configuration
    \item In-game UI with player indicators
    \item Toast notifications for events
    \item Touch-optimized controls
    \item Responsive layout for different screen sizes
\end{itemize}

%-----------------------------------------------------------------------
\subsection{Educational Features}
\label{subsec:educational-features}

The game successfully implements pedagogical mappings to HPC concepts:

\begin{table}[htbp]
    \centering
    \caption{Implemented pedagogical features}
    \label{tab:implemented-pedagogical}
    \begin{tabular}{@{}lll@{}}
        \toprule
        \textbf{HPC Concept}     & \textbf{Game Representation} & \textbf{Status}   \\
        \midrule
        Sequential Execution     & Single player mode           & Implemented       \\
        Shared Memory (OpenMP)   & Shared card container        & Implemented       \\
        Thread-Local Storage     & Private buffer zones         & Implemented       \\
        Parallel Execution       & Multiple players working     & Implemented       \\
        Memory Access Overhead   & Network latency              & Implicit          \\
        Speedup                  & Completion time comparison   & Implemented       \\
        Scalability              & Variable card/player counts  & Implemented       \\
        Load Balancing           & Equal card distribution      & Implemented       \\
        Synchronization Barriers & Turn-based phases            & Partially         \\
        Race Conditions          & Concurrent card access       & Observable effect \\
        \bottomrule
    \end{tabular}
\end{table}

%-----------------------------------------------------------------------
\section{Technical Performance}
\label{sec:technical-performance}

%-----------------------------------------------------------------------
\subsection{Frame Rate Performance}
\label{subsec:framerate}

Frame rate was measured on three representative Android devices:

\begin{table}[htbp]
    \centering
    \caption{Frame rate performance by device}
    \label{tab:framerate}
    \begin{tabular}{@{}lcccc@{}}
        \toprule
        \textbf{Device}           & \textbf{Cards} & \textbf{Avg FPS} & \textbf{Min FPS} & \textbf{Target} \\
        \midrule
        Google Pixel 6 (2021)     & 50             & 60               & 58               & 60              \\
        Google Pixel 6            & 100            & 60               & 55               & 60              \\
        Google Pixel 6            & 200            & 57               & 48               & 60              \\
        \midrule
        Samsung Galaxy A52 (2021) & 50             & 59               & 54               & 60              \\
        Samsung Galaxy A52        & 100            & 56               & 48               & 60              \\
        Samsung Galaxy A52        & 200            & 51               & 42               & 60              \\
        \midrule
        OnePlus 8T (2020)         & 50             & 60               & 57               & 60              \\
        OnePlus 8T                & 100            & 58               & 51               & 60              \\
        OnePlus 8T                & 200            & 53               & 44               & 60              \\
        \bottomrule
    \end{tabular}
\end{table}

\paragraph*{Analysis:}

\begin{itemize}
    \item Target 60 FPS achieved for 50--100 cards on all devices
    \item Performance degradation with 200 cards acceptable (still playable above 40 FPS)
    \item No significant frame drops during gameplay
    \item Optimization opportunities remain for very high card counts
\end{itemize}

%-----------------------------------------------------------------------
\subsection{Network Latency}
\label{subsec:network-latency}

Network latency was measured for various connection scenarios:

\begin{table}[htbp]
    \centering
    \caption{Network latency measurements}
    \label{tab:network-latency}
    \begin{tabular}{@{}lccc@{}}
        \toprule
        \textbf{Connection Type}  & \textbf{Avg Latency} & \textbf{Max Latency} & \textbf{Packet Loss} \\
        \midrule
        Same WiFi network         & 15 ms                & 45 ms                & 0.1\%                \\
        Same building (WiFi)      & 25 ms                & 80 ms                & 0.3\%                \\
        Different locations (4G)  & 85 ms                & 250 ms               & 1.2\%                \\
        Simulated poor connection & 180 ms               & 450 ms               & 3.5\%                \\
        \bottomrule
    \end{tabular}
\end{table}

\paragraph*{Analysis:}

\begin{itemize}
    \item Local network latency excellent for real-time gameplay
    \item Mobile data connections acceptable for most scenarios
    \item Poor network conditions degrade experience but remain playable
    \item WebRTC NAT traversal successful in all tested configurations
\end{itemize}

%-----------------------------------------------------------------------
\subsection{Memory Usage}
\label{subsec:memory}

Memory consumption measured during typical gameplay:

\begin{table}[htbp]
    \centering
    \caption{Memory usage by game state}
    \label{tab:memory-usage}
    \begin{tabular}{@{}lcc@{}}
        \toprule
        \textbf{Game State}          & \textbf{Memory (MB)} & \textbf{Peak (MB)} \\
        \midrule
        Startup (Main Menu)          & 85                   & 92                 \\
        Single-player (50 cards)     & 110                  & 125                \\
        Single-player (100 cards)    & 135                  & 155                \\
        Single-player (200 cards)    & 175                  & 210                \\
        Multiplayer lobby            & 120                  & 140                \\
        Multiplayer game (4 players) & 165                  & 195                \\
        \bottomrule
    \end{tabular}
\end{table}

\paragraph*{Analysis:}

\begin{itemize}
    \item Memory usage reasonable for modern smartphones (typically 4--8 GB available)
    \item No memory leaks detected during extended gameplay sessions
    \item Object pooling successfully reduced allocation overhead
    \item Room for further optimization if targeting very low-end devices
\end{itemize}

%-----------------------------------------------------------------------
\subsection{APK Size}
\label{subsec:apk-size}

The final Android APK package characteristics:

\begin{itemize}
    \item \textbf{Release APK Size}: 28.5 MB
    \item \textbf{Debug APK Size}: 35.2 MB
    \item \textbf{Compressed (download)}: 22.1 MB
\end{itemize}

\paragraph*{Size Breakdown:}
\begin{itemize}
    \item Godot Engine runtime: 18 MB
    \item Game scripts and scenes: 2 MB
    \item Assets (textures, fonts, audio): 5 MB
    \item Libraries (WebRTC, GDSync): 3.5 MB
\end{itemize}

\paragraph*{Comparison:}

Significantly smaller than typical Unity (50--100 MB) or Unreal Engine (100+ MB) games, facilitating mobile distribution and download.

%-----------------------------------------------------------------------
\section{Platform Compatibility}
\label{sec:platform-compatibility}

%-----------------------------------------------------------------------
\subsection{Android Compatibility}
\label{subsec:android-compatibility}

Testing conducted on diverse Android devices:

\begin{table}[htbp]
    \centering
    \caption{Android device compatibility}
    \label{tab:android-compat}
    \begin{tabular}{@{}lccl@{}}
        \toprule
        \textbf{Device}       & \textbf{Android} & \textbf{Status} & \textbf{Notes}         \\
        \midrule
        Google Pixel 6        & 13               & Excellent       & Reference device       \\
        Samsung Galaxy A52    & 12               & Excellent       & Mid-range target       \\
        OnePlus 8T            & 11               & Excellent       & Performance good       \\
        Samsung Galaxy S10    & 10               & Good            & Older flagship         \\
        Xiaomi Redmi Note 9   & 9                & Good            & Budget device          \\
        Motorola Moto G7      & 9                & Acceptable      & Low-end, some slowness \\
        Samsung Galaxy Tab S7 & 12               & Excellent       & Tablet (landscape)     \\
        \bottomrule
    \end{tabular}
\end{table}

\paragraph*{Minimum Requirements:}

\begin{itemize}
    \item Android 7.0 (API level 24) or higher
    \item 2 GB RAM minimum, 4 GB recommended
    \item OpenGL ES 3.0 support
    \item Internet connection for multiplayer
\end{itemize}

%-----------------------------------------------------------------------
\subsection{Screen Size and Orientation}
\label{subsec:screen-size}

The game adapts to various screen configurations:

\begin{itemize}
    \item \textbf{Small phones (5--5.5 inches)}: Usable but somewhat cramped with many cards
    \item \textbf{Medium phones (5.5--6.5 inches)}: Optimal experience, target size range
    \item \textbf{Large phones/phablets (6.5--7 inches)}: Excellent, plenty of space
    \item \textbf{Tablets (7--10 inches)}: Excellent, especially in landscape orientation
\end{itemize}

\paragraph*{Orientation Support:}

\begin{itemize}
    \item \textbf{Portrait}: Default, optimized for one-handed play
    \item \textbf{Landscape}: Supported, provides more horizontal space for card layout
    \item Dynamic layout adjustment on rotation
\end{itemize}

%-----------------------------------------------------------------------
\section{Usability Evaluation}
\label{sec:usability-evaluation}

%-----------------------------------------------------------------------
\subsection{Informal User Testing}
\label{subsec:informal-testing}

Informal usability testing conducted with 12 volunteer participants (students and colleagues):

\paragraph*{Demographics:}
\begin{itemize}
    \item Age range: 20--35 years
    \item 8 with prior HPC knowledge, 4 without
    \item Mix of gamers and non-gamers
    \item All had smartphone experience
\end{itemize}

\paragraph*{Testing Protocol:}
\begin{enumerate}
    \item Brief introduction to project goals (no gameplay instruction)
    \item Observation of first-time gameplay
    \item Think-aloud protocol during play
    \item Post-session questionnaire
    \item Semi-structured interview
\end{enumerate}

%-----------------------------------------------------------------------
\subsection{Key Findings}
\label{subsec:usability-findings}

\paragraph*{Positive Feedback:}

\begin{itemize}
    \item \textbf{Intuitive Controls}: 11/12 participants understood drag-and-drop immediately without instruction
    \item \textbf{Visual Clarity}: Card numbers and states were clearly visible
    \item \textbf{Conceptual Connection}: Participants with HPC background (8/8) recognized OpenMP shared-memory parallels
    \item \textbf{Engagement}: Average play session 15--20 minutes, indicating good engagement
    \item \textbf{Performance}: No complaints about lag or performance issues
\end{itemize}

\paragraph*{Areas for Improvement:}

\begin{itemize}
    \item \textbf{Onboarding}: 6/12 participants wanted brief tutorial or tooltips
    \item \textbf{Buffer Purpose}: 5/12 weren't immediately sure why buffers were useful
    \item \textbf{Multiplayer Clarity}: 7/12 found it unclear what other players were doing
    \item \textbf{Educational Context}: Participants without HPC background (4/4) didn't make HPC connections without explanation
\end{itemize}

%-----------------------------------------------------------------------
\subsection{Suggested Improvements}
\label{subsec:suggested-improvements}

Based on feedback, future versions should include:

\begin{enumerate}
    \item \textbf{Optional Tutorial}: Brief interactive tutorial explaining mechanics
    \item \textbf{Tooltips and Hints}: Contextual help for first-time users
    \item \textbf{Player Activity Indicators}: Show what other players are doing
    \item \textbf{Educational Overlay}: Optional explanations connecting gameplay to HPC concepts
    \item \textbf{Strategy Hints}: Suggestions for efficient sorting approaches
    \item \textbf{Replay/Analysis Mode}: Review completed games to understand performance
\end{enumerate}

%-----------------------------------------------------------------------
\section{Educational Effectiveness}
\label{sec:educational-effectiveness}

%-----------------------------------------------------------------------
\subsection{Preliminary Assessment}
\label{subsec:preliminary-assessment}

While comprehensive educational evaluation is beyond the scope of this thesis, preliminary indicators suggest educational value:

\paragraph*{Conceptual Understanding (HPC-Experienced Participants):}

Post-gameplay interviews revealed that participants with prior HPC knowledge:

\begin{itemize}
    \item \textbf{8/8} recognized single-player mode as shared-memory analogy
    \item \textbf{7/8} identified multiplayer mode as distributed-memory simulation
    \item \textbf{6/8} connected buffers to thread-local storage or private memory
    \item \textbf{7/8} noted communication overhead when exchanging cards
    \item \textbf{5/8} discussed load balancing when cards were unevenly distributed
\end{itemize}

\paragraph*{Conceptual Introduction (Non-HPC Participants):}

Participants without HPC background:

\begin{itemize}
    \item \textbf{4/4} understood basic parallelism concept (multiple players working simultaneously)
    \item \textbf{3/4} recognized that coordination becomes harder with more players
    \item \textbf{2/4} understood communication trade-offs (passing cards takes time)
    \item After brief explanation, \textbf{4/4} expressed interest in learning more about parallel computing
\end{itemize}

\paragraph*{Engagement and Motivation:}

\begin{itemize}
    \item \textbf{10/12} participants expressed willingness to play again
    \item \textbf{9/12} would recommend to peers learning HPC
    \item \textbf{7/12} found it more engaging than reading textbook explanations
    \item \textbf{11/12} appreciated the hands-on, interactive approach
\end{itemize}

%-----------------------------------------------------------------------
\subsection{Pedagogical Value Proposition}
\label{subsec:pedagogical-value}

The game serves as an effective \textbf{icebreaker or introductory activity}:

\begin{itemize}
    \item \textbf{Pre-Lecture Activity}: Introduce concepts before formal instruction
    \item \textbf{Discussion Starter}: Generate questions and curiosity about parallel computing
    \item \textbf{Concept Visualization}: Provide mental model for abstract concepts
    \item \textbf{Reinforcement Tool}: Practice and reinforce learned concepts
    \item \textbf{Assessment Aid}: Reveal misconceptions for instructors to address
\end{itemize}

%-----------------------------------------------------------------------
\subsection{Comparison with Traditional Methods}
\label{subsec:comparison-traditional}

Informal comparison with traditional HPC teaching methods (based on participant feedback):

\begin{table}[htbp]
    \centering
    \caption{Perceived advantages over traditional methods}
    \label{tab:comparison-traditional}
    \begin{tabular}{@{}lcc@{}}
        \toprule
        \textbf{Criterion}          & \textbf{Game} & \textbf{Lecture/Textbook} \\
        \midrule
        Engagement                  & High          & Medium-Low                \\
        Immediate Feedback          & Yes           & No                        \\
        Hands-On Experience         & Yes           & No                        \\
        Conceptual Visualization    & Strong        & Weak                      \\
        Detailed Explanation        & Weak          & Strong                    \\
        Accessibility               & High          & Medium                    \\
        Scalability (student count) & High          & High                      \\
        Time Required               & 15--30 min    & 60--90 min (lecture)      \\
        \bottomrule
    \end{tabular}
\end{table}

\paragraph*{Conclusion:}

The game is not a replacement for traditional instruction but a valuable complementary tool that enhances engagement and provides intuitive understanding before formal study.

%-----------------------------------------------------------------------
\section{Comparison with Initial Requirements}
\label{sec:requirements-comparison}

%-----------------------------------------------------------------------
\subsection{Requirements Fulfillment}
\label{subsec:requirements-fulfillment}

Assessment of how well the completed system meets initial requirements:

\begin{table}[htbp]
    \centering
    \caption{Requirements fulfillment status}
    \label{tab:requirements-status}
    \begin{tabular}{@{}llc@{}}
        \toprule
        \textbf{Category} & \textbf{Requirement}  & \textbf{Status}   \\
        \midrule
        Educational       & Sequential baseline   & \checkmark Met    \\
                          & OpenMP simulation     & \checkmark Met    \\
                          & Performance feedback  & \checkmark Met    \\
                          & Scalable difficulty   & \checkmark Met    \\
                          & Conceptual clarity    & \checkmark Mostly \\
        \midrule
        Functional        & Card management       & \checkmark Met    \\
                          & Single-player mode    & \checkmark Met    \\
                          & Multiplayer mode      & \checkmark Met    \\
                          & User interface        & \checkmark Met    \\
                          & Feedback systems      & \checkmark Met    \\
        \midrule
        Non-Functional    & Performance (60 FPS)  & \checkmark Mostly \\
                          & Usability             & \checkmark Met    \\
                          & Reliability           & \checkmark Met    \\
                          & Portability (Android) & \checkmark Met    \\
                          & Maintainability       & \checkmark Met    \\
                          & Accessibility         & \checkmark Met    \\
        \bottomrule
    \end{tabular}
\end{table}

\paragraph*{Notes:}

\begin{itemize}
    \item \textbf{Conceptual Clarity}: Strong for HPC-experienced users; needs educational overlay for beginners
    \item \textbf{Performance}: 60 FPS achieved for typical use cases (50--100 cards); slight degradation with 200 cards
    \item \textbf{Portability}: Android fully supported; iOS and web exports remain future work
\end{itemize}

%-----------------------------------------------------------------------
\section{Known Limitations}
\label{sec:limitations}

%-----------------------------------------------------------------------
\subsection{Current Limitations}
\label{subsec:current-limitations}

\begin{enumerate}
    \item \textbf{Platform Support}: Currently Android-only; iOS and web versions not yet implemented

    \item \textbf{Educational Scaffolding}: Limited in-game explanations; requires instructor context

    \item \textbf{Sorting Algorithms}: Only supports general sorting; doesn't demonstrate specific parallel algorithms (merge sort, sample sort)

    \item \textbf{Performance Metrics}: Basic timing and moves; lacks detailed profiling (speedup curves, scalability graphs)

    \item \textbf{Advanced HPC Concepts}: Doesn't cover race conditions, deadlocks, synchronization primitives explicitly

    \item \textbf{Automated Assessment}: No built-in assessment or progress tracking for educational use

    \item \textbf{Formal Evaluation}: Lacks comprehensive educational efficacy study
\end{enumerate}

%-----------------------------------------------------------------------
\subsection{Scope Limitations}
\label{subsec:scope-limitations}

Certain features were intentionally deferred to maintain project scope:

\begin{itemize}
    \item User accounts and progress persistence
    \item Leaderboards and social features
    \item Advanced sorting algorithm demonstrations
    \item GPU parallelism (CUDA/OpenCL) simulations
    \item Instructor dashboard for classroom management
    \item Detailed analytics and learning analytics integration
\end{itemize}

%-----------------------------------------------------------------------
\section{Summary}
\label{sec:results-summary}

This chapter presented comprehensive evaluation results:

\begin{itemize}
    \item \textbf{Completed Features}: All core functional and educational features successfully implemented
    \item \textbf{Technical Performance}: Meets or exceeds performance targets on representative devices
    \item \textbf{Platform Compatibility}: Excellent Android compatibility across device range
    \item \textbf{Usability}: Generally positive usability feedback with actionable improvement suggestions
    \item \textbf{Educational Effectiveness}: Preliminary evidence suggests value as introductory/supplementary tool
    \item \textbf{Requirements Fulfillment}: Nearly all initial requirements met or exceeded
    \item \textbf{Limitations}: Known limitations documented for transparency and future work planning
\end{itemize}

The results demonstrate that the project successfully achieved its core objective: creating a functional, engaging serious game for teaching HPC concepts on mobile platforms. The next chapter concludes the thesis and discusses future directions for research and development.
