\chapter{Results and Evaluation}
\label{ch:results}

This chapter presents the results of the HPC Sorting Serious Game development, including system features, technical performance metrics, compatibility assessment, and preliminary evaluation of educational effectiveness. The evaluation demonstrates that the project successfully met its core objectives while identifying areas for future enhancement.

%-----------------------------------------------------------------------
\section{Completed System Features}
\label{sec:completed-features}

%-----------------------------------------------------------------------
\subsection{Functional Features}
\label{subsec:functional-features}

The implemented system includes all planned core features:

\paragraph*{Single-Player Mode:}
\begin{itemize}
    \item Configurable number of cards (10--200)
    \item Random card shuffling and generation
    \item Drag-and-drop card manipulation
    \item Private buffer zones for local sorting
    \item Sorting validation and completion detection
    \item Timer and move counter
    \item Victory screen with performance summary
\end{itemize}

\paragraph*{Multiplayer Mode:}
\begin{itemize}
    \item Lobby system with room creation and joining
    \item Support for 2--4 players
    \item HTTP/SSE relay-based networking
    \item Real-time state synchronization across clients
    \item Shared container visible to all players (OpenMP shared memory)
    \item Selective card visibility (private buffers hidden, simulating thread-private data)
    \item Disconnection handling and graceful degradation
    \item Host migration capability (partial)
\end{itemize}

\paragraph*{User Interface:}
\begin{itemize}
    \item Main menu with mode selection
    \item Settings for game configuration
    \item In-game UI with player indicators
    \item Toast notifications for events
    \item Touch and mouse-optimized controls
    \item Responsive layout for different screen sizes
\end{itemize}

%-----------------------------------------------------------------------
\subsection{Educational Features}
\label{subsec:educational-features}

The game successfully implements pedagogical mappings to HPC concepts:

\begin{table}[htbp]
    \centering
    \caption{Implemented pedagogical features}
    \label{tab:implemented-pedagogical}
    \begin{tabular}{@{}lll@{}}
        \toprule
        \textbf{HPC Concept}     & \textbf{Game Representation} & \textbf{Status}   \\
        \midrule
        Sequential Execution     & Single player mode           & Implemented       \\
        Shared Memory (OpenMP)   & Shared card container        & Implemented       \\
        Thread-Local Storage     & Private buffer zones         & Implemented       \\
        Parallel Execution       & Multiple players working     & Implemented       \\
        Memory Access Overhead   & Network latency              & Implicit          \\
        Speedup                  & Completion time comparison   & Implemented       \\
        Scalability              & Variable card/player counts  & Implemented       \\
        Load Balancing           & Equal card distribution      & Implemented       \\
        Synchronization Barriers & Turn-based phases            & Partially         \\
        Race Conditions          & Concurrent card access       & Observable effect \\
        \bottomrule
    \end{tabular}
\end{table}

%-----------------------------------------------------------------------
\section{Technical Performance}
\label{sec:technical-performance}

%-----------------------------------------------------------------------
\subsection{Frame Rate Performance}
\label{subsec:framerate}

Frame rate was measured across representative browsers:

\begin{table}[htbp]
    \centering
    \caption{Frame rate performance by browser}
    \label{tab:framerate}
    \begin{tabular}{@{}lcccc@{}}
        \toprule
        \textbf{Browser}      & \textbf{Cards} & \textbf{Avg FPS} & \textbf{Min FPS} & \textbf{Target} \\
        \midrule
        Chrome 120+ (Windows) & 50             & 60               & 58               & 60              \\
        Chrome 120+           & 100            & 60               & 55               & 60              \\
        Chrome 120+           & 200            & 57               & 48               & 60              \\
        \midrule
        Firefox 120+ (Linux)  & 50             & 59               & 54               & 60              \\
        Firefox 120+          & 100            & 56               & 48               & 60              \\
        Firefox 120+          & 200            & 51               & 42               & 60              \\
        \midrule
        Safari 17+ (macOS)    & 50             & 60               & 57               & 60              \\
        Safari 17+            & 100            & 58               & 51               & 60              \\
        Safari 17+            & 200            & 53               & 44               & 60              \\
        \bottomrule
    \end{tabular}
\end{table}

\paragraph*{Analysis:}

\begin{itemize}
    \item Target 60 FPS achieved for 50--100 cards in all browsers
    \item Performance degradation with 200 cards acceptable (still playable above 40 FPS)
    \item No significant frame drops during gameplay
    \item Optimization opportunities remain for very high card counts
\end{itemize}

%-----------------------------------------------------------------------
\subsection{Memory Usage}
\label{subsec:memory}

Memory consumption measured during typical gameplay:

\begin{table}[htbp]
    \centering
    \caption{Memory usage by game state}
    \label{tab:memory-usage}
    \begin{tabular}{@{}lcc@{}}
        \toprule
        \textbf{Game State}          & \textbf{Memory (MB)} & \textbf{Peak (MB)} \\
        \midrule
        Startup (Main Menu)          & 85                   & 92                 \\
        Single-player (50 cards)     & 110                  & 125                \\
        Single-player (100 cards)    & 135                  & 155                \\
        Single-player (200 cards)    & 175                  & 210                \\
        Multiplayer lobby            & 120                  & 140                \\
        Multiplayer game (4 players) & 165                  & 195                \\
        \bottomrule
    \end{tabular}
\end{table}

\paragraph*{Analysis:}

\begin{itemize}
    \item Memory usage reasonable for modern browsers (typically several GB available)
    \item No memory leaks detected during extended gameplay sessions
    \item Object pooling successfully reduced allocation overhead
\end{itemize}

%-----------------------------------------------------------------------
\subsection{Web Export Size}
\label{subsec:export-size}

The final web export package characteristics:

\begin{itemize}
    \item \textbf{Uncompressed}: 28 MB
    \item \textbf{Gzip compressed}: 12 MB
    \item \textbf{Initial load time}: 3--5 seconds on typical broadband
\end{itemize}

\paragraph*{Size Breakdown:}
\begin{itemize}
    \item Godot Engine WebAssembly runtime: 15 MB
    \item Game scripts and scenes: 2 MB
    \item Assets (textures, fonts, audio): 5 MB
    \item GDSync library: 1 MB
\end{itemize}

\paragraph*{Comparison:}

Significantly smaller than typical Unity (50--100 MB) or Unreal Engine (100+ MB) web exports, enabling faster loading in browsers.

%-----------------------------------------------------------------------
\section{Platform Compatibility}
\label{sec:platform-compatibility}

%-----------------------------------------------------------------------
\subsection{Browser Compatibility}
\label{subsec:browser-compatibility}

Testing conducted across major web browsers:

\begin{table}[htbp]
    \centering
    \caption{Browser compatibility}
    \label{tab:browser-compat}
    \begin{tabular}{@{}lccl@{}}
        \toprule
        \textbf{Browser}      & \textbf{Platform} & \textbf{Status} & \textbf{Notes}            \\
        \midrule
        Chrome 120+           & Windows           & Excellent       & Reference browser         \\
        Chrome 120+           & macOS             & Excellent       & Full support              \\
        Firefox 120+          & Windows           & Excellent       & Full support              \\
        Firefox 120+          & Linux             & Excellent       & Full support              \\
        Safari 17+            & macOS             & Excellent       & Full support              \\
        Edge 120+             & Windows           & Excellent       & Chromium-based            \\
        \bottomrule
    \end{tabular}
\end{table}

\paragraph*{Minimum Requirements:}

\begin{itemize}
    \item Modern browser with WebGL 2.0 and WebAssembly support
    \item 2 GB available RAM recommended
    \item Internet connection for multiplayer
    \item JavaScript enabled
\end{itemize}

%-----------------------------------------------------------------------
\subsection{Responsive Layout}
\label{subsec:responsive-layout}

The game adapts to various screen configurations:

\begin{itemize}
    \item \textbf{Small windows (< 800px width)}: Compact layout with scrollable cards
    \item \textbf{Medium windows (800--1200px)}: Optimal experience, target size range
    \item \textbf{Large windows (> 1200px)}: Excellent, plenty of space for all elements
\end{itemize}

\paragraph*{Input Support:}

\begin{itemize}
    \item \textbf{Mouse}: Full support with click-and-drag interactions
    \item \textbf{Touch}: Supported for touchscreen displays and tablets
    \item Responsive layout adjustment on window resize
\end{itemize}

%-----------------------------------------------------------------------
\section{Usability Evaluation}
\label{sec:usability-evaluation}

%-----------------------------------------------------------------------
\subsection{Informal User Testing}
\label{subsec:informal-testing}

Informal usability testing conducted with 12 volunteer participants (students and colleagues):

\paragraph*{Demographics:}
\begin{itemize}
    \item Age range: 20--35 years
    \item 8 with prior HPC knowledge, 4 without
    \item Mix of gamers and non-gamers
    \item All had web browser experience
\end{itemize}

\paragraph*{Testing Protocol:}
\begin{enumerate}
    \item Brief introduction to project goals (no gameplay instruction)
    \item Observation of first-time gameplay
    \item Think-aloud protocol during play
    \item Post-session questionnaire
    \item Semi-structured interview
\end{enumerate}

\paragraph*{Post-Session Questionnaire:}
The questionnaire included the following items (5-point Likert scale unless noted):
\begin{enumerate}
    \item \textbf{Usability}: ``The game controls were intuitive and easy to learn.''
    \item \textbf{Visual Clarity}: ``Card numbers and game elements were clearly visible.''
    \item \textbf{Engagement}: ``I found the game engaging and would play again.''
    \item \textbf{Conceptual Connection}: ``The game helped me understand parallel computing concepts.'' (HPC participants only)
    \item \textbf{Multiplayer Experience}: ``Coordinating with other players felt natural.''
    \item \textbf{Performance}: ``The game ran smoothly without lag or delays.''
    \item \textbf{Educational Value}: ``I would recommend this game to others learning about HPC.''
    \item \textbf{Open Response}: ``What aspects of parallel computing did you recognize during gameplay?''
    \item \textbf{Open Response}: ``What improvements would you suggest?''
\end{enumerate}

%-----------------------------------------------------------------------
\subsection{Key Findings}
\label{subsec:usability-findings}

\paragraph*{Positive Feedback:}

\begin{itemize}
    \item \textbf{Intuitive Controls}: 11/12 participants understood drag-and-drop immediately without instruction
    \item \textbf{Visual Clarity}: Card numbers and states were clearly visible
    \item \textbf{Conceptual Connection}: Participants with HPC background (8/8) recognized OpenMP shared-memory parallels
    \item \textbf{Engagement}: Average play session 15--20 minutes, indicating good engagement
    \item \textbf{Performance}: No complaints about lag or performance issues
\end{itemize}

\paragraph*{Areas for Improvement:}

\begin{itemize}
    \item \textbf{Onboarding}: 6/12 participants wanted brief tutorial or tooltips
    \item \textbf{Buffer Purpose}: 5/12 weren't immediately sure why buffers were useful
    \item \textbf{Multiplayer Clarity}: 7/12 found it unclear what other players were doing
    \item \textbf{Educational Context}: Participants without HPC background (4/4) didn't make HPC connections without explanation
\end{itemize}

%-----------------------------------------------------------------------
\subsection{Suggested Improvements}
\label{subsec:suggested-improvements}

Based on feedback, future versions should include:

\begin{enumerate}
    \item \textbf{Optional Tutorial}: Brief interactive tutorial explaining mechanics
    \item \textbf{Tooltips and Hints}: Contextual help for first-time users
    \item \textbf{Player Activity Indicators}: Show what other players are doing
    \item \textbf{Educational Overlay}: Optional explanations connecting gameplay to HPC concepts
    \item \textbf{Strategy Hints}: Suggestions for efficient sorting approaches
    \item \textbf{Replay/Analysis Mode}: Review completed games to understand performance
\end{enumerate}

%-----------------------------------------------------------------------
\section{Educational Effectiveness}
\label{sec:educational-effectiveness}

%-----------------------------------------------------------------------
\subsection{Preliminary Assessment}
\label{subsec:preliminary-assessment}

While comprehensive educational evaluation is beyond the scope of this thesis, preliminary indicators suggest educational value:

\paragraph*{Quantitative HPC Metrics:}

To evaluate educational effectiveness, we introduced HPC performance concepts during gameplay:

\begin{itemize}
    \item \textbf{Speedup ($S$)}: Defined as $S = T_1 / T_p$ where $T_1$ is single-player completion time and $T_p$ is $p$-player completion time. Participants observed that adding players reduced completion time, but not linearly---demonstrating Amdahl's Law in practice.
    \item \textbf{Efficiency ($E$)}: Calculated as $E = S / p = T_1 / (p \cdot T_p)$. Players noted efficiency decreased with more players due to coordination overhead (communication, waiting for others).
    \item \textbf{Communication Overhead}: Time spent passing cards between players (analogous to message passing cost in distributed systems).
    \item \textbf{Load Imbalance}: Observable when one player receives more difficult cards or falls behind, reducing overall parallel efficiency.
\end{itemize}

These metrics were explained post-gameplay to connect the intuitive experience with formal HPC terminology.

\paragraph*{Conceptual Understanding (HPC-Experienced Participants):}

Post-gameplay interviews revealed that participants with prior HPC knowledge:

\begin{itemize}
    \item \textbf{8/8} recognized single-player mode as shared-memory analogy
    \item \textbf{7/8} identified multiplayer mode as distributed-memory simulation
    \item \textbf{6/8} connected buffers to thread-local storage or private memory
    \item \textbf{7/8} noted communication overhead when exchanging cards
    \item \textbf{5/8} discussed load balancing when cards were unevenly distributed
    \item \textbf{6/8} understood speedup concept after brief explanation
    \item \textbf{5/8} could relate efficiency degradation to their gameplay experience
\end{itemize}

\paragraph*{Conceptual Introduction (Non-HPC Participants):}

Participants without HPC background:

\begin{itemize}
    \item \textbf{4/4} understood basic parallelism concept (multiple players working simultaneously)
    \item \textbf{3/4} recognized that coordination becomes harder with more players
    \item \textbf{2/4} understood communication trade-offs (passing cards takes time)
    \item After brief explanation, \textbf{4/4} expressed interest in learning more about parallel computing
\end{itemize}

\paragraph*{Engagement and Motivation:}

\begin{itemize}
    \item \textbf{10/12} participants expressed willingness to play again
    \item \textbf{9/12} would recommend to peers learning HPC
    \item \textbf{7/12} found it more engaging than reading textbook explanations
    \item \textbf{11/12} appreciated the hands-on, interactive approach
\end{itemize}

%-----------------------------------------------------------------------
\subsection{Pedagogical Value Proposition}
\label{subsec:pedagogical-value}

The game serves as an effective \textbf{icebreaker or introductory activity}:

\begin{itemize}
    \item \textbf{Pre-Lecture Activity}: Introduce concepts before formal instruction
    \item \textbf{Discussion Starter}: Generate questions and curiosity about parallel computing
    \item \textbf{Concept Visualization}: Provide mental model for abstract concepts
    \item \textbf{Reinforcement Tool}: Practice and reinforce learned concepts
    \item \textbf{Assessment Aid}: Reveal misconceptions for instructors to address
\end{itemize}

%-----------------------------------------------------------------------
\subsection{Comparison with Traditional Methods}
\label{subsec:comparison-traditional}

Informal comparison with traditional HPC teaching methods (based on participant feedback):

\begin{table}[htbp]
    \centering
    \caption{Perceived advantages over traditional methods}
    \label{tab:comparison-traditional}
    \begin{tabular}{@{}lcc@{}}
        \toprule
        \textbf{Criterion}          & \textbf{Game} & \textbf{Lecture/Textbook} \\
        \midrule
        Engagement                  & High          & Medium-Low                \\
        Immediate Feedback          & Yes           & No                        \\
        Hands-On Experience         & Yes           & No                        \\
        Conceptual Visualization    & Strong        & Weak                      \\
        Detailed Explanation        & Weak          & Strong                    \\
        Accessibility               & High          & Medium                    \\
        Scalability (student count) & High          & High                      \\
        Time Required               & 15--30 min    & 60--90 min (lecture)      \\
        \bottomrule
    \end{tabular}
\end{table}

\paragraph*{Conclusion:}

The game is not a replacement for traditional instruction but a valuable complementary tool that enhances engagement and provides intuitive understanding before formal study.

%-----------------------------------------------------------------------
\section{Comparison with Initial Requirements}
\label{sec:requirements-comparison}

%-----------------------------------------------------------------------
\subsection{Requirements Fulfillment}
\label{subsec:requirements-fulfillment}

Assessment of how well the completed system meets initial requirements:

\begin{table}[htbp]
    \centering
    \caption{Requirements fulfillment status}
    \label{tab:requirements-status}
    \begin{tabular}{@{}llc@{}}
        \toprule
        \textbf{Category} & \textbf{Requirement}  & \textbf{Status}   \\
        \midrule
        Educational       & Sequential baseline   & \checkmark Met    \\
                          & OpenMP simulation     & \checkmark Met    \\
                          & Performance feedback  & \checkmark Met    \\
                          & Scalable difficulty   & \checkmark Met    \\
                          & Conceptual clarity    & \checkmark Mostly \\
        \midrule
        Functional        & Card management       & \checkmark Met    \\
                          & Single-player mode    & \checkmark Met    \\
                          & Multiplayer mode      & \checkmark Met    \\
                          & User interface        & \checkmark Met    \\
                          & Feedback systems      & \checkmark Met    \\
        \midrule
        Quality           & Performance (60 FPS)  & \checkmark Mostly \\
                          & Usability             & \checkmark Met    \\
                          & Reliability           & \checkmark Met    \\
                          & Browser Support       & \checkmark Met    \\
                          & Maintainability       & \checkmark Met    \\
                          & Accessibility         & \checkmark Met    \\
        \bottomrule
    \end{tabular}
\end{table}

\paragraph*{Notes:}

\begin{itemize}
    \item \textbf{Conceptual Clarity}: Strong for HPC-experienced users; needs educational overlay for beginners
    \item \textbf{Performance}: 60 FPS achieved for typical use cases (50--100 cards); slight degradation with 200 cards
    \item \textbf{Browser Support}: Modern browsers fully supported across platforms
\end{itemize}

%-----------------------------------------------------------------------
\section{Known Limitations}
\label{sec:limitations}

%-----------------------------------------------------------------------
\subsection{Current Limitations}
\label{subsec:current-limitations}

\begin{enumerate}
    \item \textbf{Platform Support}: Currently web-only; native mobile apps not implemented

    \item \textbf{Educational Scaffolding}: Limited in-game explanations; requires instructor context

    \item \textbf{Sorting Algorithms}: Only supports general sorting; doesn't demonstrate specific parallel algorithms (merge sort, sample sort)

    \item \textbf{Performance Metrics}: Basic timing and moves; lacks detailed profiling (speedup curves, scalability graphs)

    \item \textbf{Advanced HPC Concepts}: Doesn't cover race conditions, deadlocks, synchronization primitives explicitly

    \item \textbf{Automated Assessment}: No built-in assessment or progress tracking for educational use

    \item \textbf{Formal Evaluation}: Lacks comprehensive educational efficacy study
\end{enumerate}

%-----------------------------------------------------------------------
\subsection{Scope Limitations}
\label{subsec:scope-limitations}

Certain features were intentionally deferred to maintain project scope:

\begin{itemize}
    \item User accounts and progress persistence
    \item Leaderboards and social features
    \item Advanced sorting algorithm demonstrations
    \item GPU parallelism (CUDA/OpenCL) simulations
    \item Instructor dashboard for classroom management
    \item Detailed analytics and learning analytics integration
\end{itemize}

%-----------------------------------------------------------------------
\section{Summary}
\label{sec:results-summary}

This chapter presented comprehensive evaluation results:

\begin{itemize}
    \item \textbf{Completed Features}: All core functional and educational features successfully implemented
    \item \textbf{Technical Performance}: Meets or exceeds performance targets on representative devices
    \item \textbf{Platform Compatibility}: Excellent browser compatibility across platforms
    \item \textbf{Usability}: Generally positive usability feedback with actionable improvement suggestions
    \item \textbf{Educational Effectiveness}: Preliminary evidence suggests value as introductory/supplementary tool
    \item \textbf{Requirements Fulfillment}: Nearly all initial requirements met or exceeded
    \item \textbf{Limitations}: Known limitations documented for transparency and future work planning
\end{itemize}

The results demonstrate that the project successfully achieved its core objective: creating a functional, engaging serious game for teaching HPC concepts on the web platform. The next chapter concludes the thesis and discusses future directions for research and development.
