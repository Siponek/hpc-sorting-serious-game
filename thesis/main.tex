%%%%%%%%%%%%%%%%%%%%%%%%%%%%%%%%%%%%%%%%%
% Master Thesis - HPC Sorting Serious Game
% University of Genova
%
% Author: Szymon Zinkowicz
% Supervisor: Professor Daniele D'Agostino
% Reviewer: Professor Maura Cerioli
%
% License: MIT
%%%%%%%%%%%%%%%%%%%%%%%%%%%%%%%%%%%%%%%%%

\documentclass[twoside]{./class/masterthesis}

%-----------------------------------------------------------------------
%	PACKAGES
%-----------------------------------------------------------------------

\usepackage[utf8]{inputenc}
\usepackage[T1]{fontenc}
\usepackage[english]{babel}
\usepackage{csquotes}
\usepackage{graphicx}
\usepackage{amsmath,amssymb,amsthm}
\usepackage[table, dvipsnames]{xcolor}
\usepackage[backend=biber,style=ieee,sorting=none,natbib=true]{biblatex}
\usepackage{hyperref}
\usepackage{subcaption}
\usepackage{caption}
\usepackage{lipsum}
\usepackage{fancyhdr}
\usepackage{float}
\usepackage{siunitx}
\usepackage{booktabs}
\usepackage{array,multirow,makecell}
\usepackage{listings}
\usepackage{geometry}
\usepackage{titlesec}
\usepackage{tocloft}
\usepackage{mdframed}

%-----------------------------------------------------------------------
%	GEOMETRY
%-----------------------------------------------------------------------

\geometry{
	top=3cm,
	bottom=3cm,
	left=3.5cm,
	right=2.5cm,
	headheight=15pt
}

%-----------------------------------------------------------------------
%	HYPERREF SETUP
%-----------------------------------------------------------------------

\hypersetup{
	colorlinks=true,
	linktoc=all,
	linkcolor=blue,
	citecolor=blue,
	urlcolor=blue,
	pdfauthor={Szymon Zinkowicz},
	pdftitle={HPC Sorting Serious Game: A Web-First Educational Approach},
}

%-----------------------------------------------------------------------
%	CODE LISTING SETUP
%-----------------------------------------------------------------------

\definecolor{codegreen}{rgb}{0,0.6,0}
\definecolor{codegray}{rgb}{0.5,0.5,0.5}
\definecolor{codepurple}{rgb}{0.58,0,0.82}
\definecolor{backcolour}{rgb}{0.95,0.95,0.92}

\lstdefinestyle{gdscript}{
    backgroundcolor=\color{backcolour},
    commentstyle=\color{codegreen},
    keywordstyle=\color{magenta},
    numberstyle=\tiny\color{codegray},
    stringstyle=\color{codepurple},
    basicstyle=\ttfamily\footnotesize,
    breakatwhitespace=false,
    breaklines=true,
    captionpos=b,
    keepspaces=true,
    numbers=left,
    numbersep=5pt,
    showspaces=false,
    showstringspaces=false,
    showtabs=false,
    tabsize=4,
    language=Python, % GDScript is Python-like
    mathescape=false
}

\lstset{style=gdscript}

%-----------------------------------------------------------------------
%	HEADERS AND FOOTERS
%-----------------------------------------------------------------------

\pagestyle{fancy}
\fancyhf{}
\fancyhead[LE]{\leftmark}
\fancyhead[RO]{\rightmark}
\fancyfoot[C]{\thepage}
\renewcommand{\headrulewidth}{0.4pt}
\renewcommand{\footrulewidth}{0.4pt}

% Chapter and section formatting
\fancypagestyle{plain}{
	\fancyhf{}
	\fancyfoot[C]{\thepage}
	\renewcommand{\headrulewidth}{0pt}
}

%-----------------------------------------------------------------------
%	TITLE FORMATTING
%-----------------------------------------------------------------------

\titleformat{\chapter}[display]
{\normalfont\huge\bfseries}{\chaptertitlename\ \thechapter}{20pt}{\Huge}
\titlespacing*{\chapter}{0pt}{0pt}{40pt}

%-----------------------------------------------------------------------
%	BIBLIOGRAPHY
%-----------------------------------------------------------------------

\addbibresource{references.bib}

%-----------------------------------------------------------------------
%	CUSTOM MACROS
%-----------------------------------------------------------------------

% Convenient macros for frequently used terms
\newcommand*\openmp{OpenMP}
\newcommand*\mpi{MPI}
\newcommand*\gdscript{GDScript}
\newcommand*\hpc{HPC}

%-----------------------------------------------------------------------
%	BEGIN DOCUMENT
%-----------------------------------------------------------------------

\begin{document}

% Document information for masterthesis.cls
\title{A serious game for High Performance Computing}
\author{Szymon Zinkowicz}
\advisor{Daniele D'Agostino}
\examiner{Maura Cerioli}

% Generate title page using masterthesis.cls
\maketitle

%-----------------------------------------------------------------------
%	DEDICATION (Optional)
%-----------------------------------------------------------------------

% Uncomment and personalize the dedication below:
% TODO: Decide on dedication text
% \dedication{
% \textit{To [I need to decide to whom I want to dedicate this work],\\
% [Placeholder]}
% }

% Example dedications:
% \dedication{\textit{To my family,\\for their unwavering support}}
% \dedication{\textit{To my parents}}
% \dedication{\textit{To my friends,\\for their encouragement and understanding}}

%-----------------------------------------------------------------------
%	ABSTRACT
%-----------------------------------------------------------------------

\chapter*{Abstract}
\addcontentsline{toc}{chapter}{Abstract}

High-Performance Computing (HPC) and parallel programming pose significant pedagogical challenges, requiring students to internalize abstractions without tangible interactive experiences. Serious games facilitate intuitive understanding through engagement, surfacing misconceptions and enabling targeted instruction.

This thesis presents an educational serious game teaching fundamental parallel computing paradigms—specifically OpenMP (shared memory)—through interactive card-sorting mechanics on mobile devices. The game materializes abstract concepts through concrete manipulation: players sort cards collaboratively in multiplayer mode with private buffer zones, simulating OpenMP shared memory with thread-local storage. Students cooperate via a web interface to minimize sorting time under constraints, learning speedup, scalability, communication overhead, multithreading, and message passing.

Implemented using Godot Engine with web export capabilities, this work resolves pedagogical and technical constraints in web-based educational multiplayer games. Development encompassed platform identification supporting extensible networking and prototype iteration.

Key contributions include: (1) novel pedagogical mapping between card-sorting mechanics and HPC paradigms derived from physical classroom experiments, (2) robust multiplayer state synchronization for educational purposes using GDSync, (3) responsive UI/UX solutions for manipulating 50+ interactive elements across screen sizes in web browsers, (4) comprehensive documentation of technical challenges and solutions regarding GDSync and multiplayer synchronization, and (5) an open-source, extensible platform for HPC education research suitable for conference publication and scientific journals.

\textbf{Keywords:} Serious Games, High-Performance Computing, Parallel Computing Education, OpenMP,
%MPI, 
Mobile Game Development, Multiplayer Synchronization, Godot Engine, GDScript, Card Sorting, Educational Technology

%-----------------------------------------------------------------------
%	ACKNOWLEDGEMENTS
%-----------------------------------------------------------------------

\chapter*{Acknowledgements}
\addcontentsline{toc}{chapter}{Acknowledgements}

I would like to express my sincere gratitude to my supervisor, Professor Daniele D'Agostino, for his guidance, valuable insights, and continuous support throughout this thesis work. His much needed enthusiasm was what has put me on the rails for High-Performance Computing research. His expertise in High-Performance Computing and educational methodologies has been instrumental in shaping this research.

I am also grateful to Professor Maura Cerioli for serving as the reviewer of this thesis, steering me in right direction regarding organisational aspects and providing constructive feedback.

Special thanks to the Godot Engine community and the developers of the GDSync framework for their open-source contributions and support in resolving technical issues encountered during development.

Finally, I thank my family and friends for their encouragement and patience during the completion of this work.

%-----------------------------------------------------------------------
%	TABLE OF CONTENTS
%-----------------------------------------------------------------------

\tableofcontents
\listoffigures
\listoftables
\lstlistoflistings


%	MAIN CONTENT
%-----------------------------------------------------------------------

\chapter{Introduction}
\label{ch:introduction}

%----------------------------------------------------------------------------------------
\section{Context and Motivation}
\label{sec:context-motivation}

%----------------------------------------------------------------------------------------
\subsection{The Challenge of Teaching High-Performance Computing}
\label{subsec:hpc-challenge}

High-Performance Computing (HPC) has become an indispensable tool in modern computational science, powering everything from weather forecasting and molecular dynamics simulations to machine learning and big data analytics. From climate modeling and drug discovery to financial simulations and artificial intelligence, parallel computing techniques are fundamental to advancing scientific knowledge and technological innovation. As computational problems grow in scale and complexity, the ability to write efficient parallel programs has transitioned from a specialized skill to a fundamental competency for software engineers and computational scientists.

However, teaching parallel computing concepts presents unique pedagogical challenges. Traditional approaches, such as lectures and textbook exercises, often struggle to convey the dynamic, interactive nature of parallel processes. Students frequently find it difficult to visualize how multiple threads or processes interact, communicate, and coordinate to solve problems efficiently.

The principal challenges in HPC education include:

\begin{enumerate}
	\item \textbf{Abstraction Gap}: Parallel computing involves concepts like threads, processes, synchronization, and message passing that are inherently abstract. Students often struggle to visualize how multiple execution units interact, communicate, and coordinate.

	\item \textbf{Cognitive Load}: Understanding parallel algorithms requires simultaneous consideration of multiple execution flows, shared resources, race conditions, and synchronization primitives—a significant cognitive burden for learners.

	\item \textbf{Limited Immediate Feedback}: Traditional programming assignments in HPC courses often involve writing code, submitting to a cluster, waiting for results, and debugging—a slow feedback loop that impedes learning.

	\item \textbf{Lack of Visualization}: Parallel processes are inherently concurrent and difficult to observe in real-time. Traditional debugging tools show only snapshots of program state, making it hard to understand the flow of parallel execution.

	\item \textbf{High Entry Barrier}: Setting up HPC environments, compiling parallel programs, and debugging distributed systems requires significant technical expertise that can distract from learning core concepts.

	\item \textbf{Lack of Intuitive Mental Models}: Unlike sequential programming, where the execution model maps naturally to step-by-step thinking, parallel programming requires different mental models that are harder to internalize.
\end{enumerate}

The abstract nature of concepts like race conditions, synchronization barriers, message passing, ``shared memory'' vs. ``distributed memory,'' ``data parallelism'' vs. ``task parallelism,'' and ``synchronization overhead'' benefit from concrete, interactive demonstrations.

Moreover, parallel programming paradigms like OpenMP (Open Multi-Processing)
%and MPI (Message Passing Interface)
require students to think differently about algorithm design. In OpenMP, programmers must consider how to decompose problems into independent tasks that can execute simultaneously while sharing memory.
%In MPI, they must understand how to distribute data across processes that cannot directly access each other's memory and must communicate explicitly through message passing.

%----------------------------------------------------------------------------------------
\subsection{Traditional Teaching Methods vs. Serious Games}
\label{subsec:traditional-vs-games}

Traditional HPC education typically relies on several well-established approaches:

\begin{itemize}
	\item \textbf{Lecture-based instruction}: Professors explain concepts using slides, diagrams, and pseudocode
	\item \textbf{Textbook exercises}: Students work through theoretical problems
	\item \textbf{Programming assignments}: Implementation tasks on academic clusters or multicore machines
	\item \textbf{Performance profiling}: Analysis of speedup, efficiency, and scalability
\end{itemize}

While these methods provide theoretical foundations, they often fail to engage students emotionally or provide immediate, intuitive understanding of parallel execution dynamics. Traditional teaching methods (lectures, static code examples) may not sufficiently engage students, leading to superficial understanding without deep conceptual mastery.

\textbf{Serious games}—games designed with a primary purpose beyond entertainment—have emerged as a powerful educational tool across various domains. By leveraging game mechanics such as immediate feedback, progressive challenges, and interactive exploration, serious games can make complex concepts more accessible and engaging.

Educational games offer several pedagogical advantages:

\begin{itemize}
	\item \textbf{Provide Immediate Feedback}: Players see the consequences of their actions instantly, enabling rapid learning cycles
	\item \textbf{Create Engaging Experiences}: Game mechanics tap into intrinsic motivation, making learning enjoyable
	\item \textbf{Enable Active Learning}: Players learn by doing, not just reading or watching
	\item \textbf{Visualize Abstract Concepts}: Game representations make invisible processes visible
	\item \textbf{Support Experimentation}: Safe environments for trial and error without real-world consequences
	\item \textbf{Visualize Parallel Execution}: Show multiple processes or threads working simultaneously in real-time
	\item \textbf{Lower the Entry Barrier}: Abstract away complex setup and focus on core concepts
	\item \textbf{Increase Engagement}: Make learning fun and motivating through game mechanics
	\item \textbf{Enable Experimentation}: Let students try different approaches safely without expensive computational resources
\end{itemize}

In the context of HPC education, serious games can transform abstract parallelization strategies into tangible, manipulable activities. Instead of imagining threads operating on shared data, students can physically (or virtually) manipulate game elements that represent computation and data.

%----------------------------------------------------------------------------------------
\subsection{Physical Teaching Experiments as Inspiration}
\label{subsec:physical-experiments}

The pedagogical approach underlying this thesis stems from real-world classroom experiments conducted by Professor Daniele D'Agostino to teach parallel computing concepts in a tangible, memorable way.

\subsubsection{OpenMP Simulation (Shared Memory Parallelism)}
In one experiment, approximately 50 numbered cards were placed on a desk, and three students were asked to sort them collaboratively in an ``OpenMP fashion''—meaning no verbal communication, simulating independent threads operating on shared data. Each student could:

\begin{itemize}
	\item View all cards (representing shared memory accessible to all threads)
	\item Move cards in the main workspace (representing shared data manipulation)
	\item Use a private area for local reordering (representing thread-local storage)
	\item Merge their locally sorted subarrays back into the main array (representing synchronization and merging phases)
\end{itemize}

This exercise effectively demonstrated:
\begin{itemize}
	\item How threads can work independently without explicit communication
	\item The challenges of coordinating work without synchronization primitives
	\item The concept of private vs. shared data in parallel computing
	\item The merge phase characteristic of parallel sorting algorithms
\end{itemize}

\subsubsection{MPI Simulation (Distributed Memory Parallelism)}
In another experiment, students were positioned at different desks in the classroom (simulating distributed computational nodes), each receiving a subset of cards. They sorted their local subsets independently, then physically walked to a ``master'' student's desk to deliver their lowest card—simulating MPI message passing. The master student collected cards from all processes and merged them into a globally sorted order.

This exercise illustrated:
\begin{itemize}
	\item Distributed data ownership (each process has its own memory space)
	\item Explicit message passing for communication between processes
	\item The master-worker pattern common in parallel algorithms
	\item Network communication overhead (walking takes time, analogous to network latency)
\end{itemize}

These physical activities were highly effective at conveying parallel computing concepts in a tangible way. Students reported better intuitive understanding and found the exercises memorable. However, these exercises face limitations:

\begin{itemize}
	\item \textbf{Scalability}: Limited by classroom size and number of students
	\item \textbf{Repeatability}: Difficult to reproduce outside specific classroom settings
	\item \textbf{Flexibility}: Hard to vary parameters (number of cards, processes, etc.)
	\item \textbf{Accessibility}: Requires physical presence and synchronous participation
\end{itemize}

This thesis aims to capture that pedagogical effectiveness in a scalable, digital format accessible on mobile and desktop devices, enabling students worldwide to experience similar learning benefits without the constraints of physical classrooms.

%----------------------------------------------------------------------------------------
\subsection{Web-First Approach}
\label{subsec:web-first}

Modern students are increasingly comfortable with web-based applications, using browsers as primary computing platforms for learning and communication. A web-first educational game offers several strategic advantages:

\begin{itemize}
	\item \textbf{Accessibility}: Students can access the game from any device with a web browser—laptops, tablets, or smartphones—without installation
	\item \textbf{Cross-Platform}: Works on Windows, macOS, Linux, Android, and iOS without platform-specific builds
	\item \textbf{Ease of Deployment}: Instant updates and no app store approval processes
	\item \textbf{Collaboration}: Web technologies naturally support multiplayer experiences through WebRTC and WebSockets
	\item \textbf{Low Barrier to Entry}: No downloads, installations, or software licenses required
	\item \textbf{Portability}: Learning can occur outside traditional classroom settings on any internet-connected device
\end{itemize}

However, developing educational games for web platforms presents unique technical challenges, particularly when implementing multiplayer functionality that must maintain state consistency across multiple devices with varying network conditions and screen sizes. These challenges include:

\begin{itemize}
	\item Displaying numerous interactive elements (50+ cards) on small screens
	\item Implementing responsive touch-based drag-and-drop interactions
	\item Maintaining real-time synchronization with limited bandwidth
	\item Ensuring acceptable performance on resource-constrained devices
	\item Handling variable network latency and potential disconnections
\end{itemize}

%----------------------------------------------------------------------------------------
\subsection{The Gap Between Theory and Practice}
\label{subsec:theory-practice-gap}

Despite the availability of parallel programming frameworks like OpenMP and MPI, students often complete HPC courses without developing strong intuition for critical concepts:

\begin{itemize}
	\item When to use shared memory vs. distributed memory approaches
	\item How to decompose problems for parallel execution
	\item The performance implications of different synchronization strategies
	\item The trade-offs between communication overhead and parallel speedup
	\item How load balancing affects overall performance
	\item The impact of data locality and cache coherence
\end{itemize}

Educational games can bridge this theory-practice gap by providing:

\begin{itemize}
	\item \textbf{Low-stakes experimentation}: Students can try different strategies without breaking expensive computational clusters or wasting limited computing time allocations
	\item \textbf{Visual representations}: Seeing cards move between buffers and containers makes abstract data movement concrete and comprehensible
	\item \textbf{Performance feedback}: Timing measurements and move counting provide immediate, quantifiable metrics that encourage algorithmic thinking
	\item \textbf{Collaborative learning}: Multiplayer mode enables peer learning, discussion, and strategy sharing
	\item \textbf{Iterative refinement}: Students can quickly replay scenarios with different approaches, fostering experimentation
\end{itemize}

%----------------------------------------------------------------------------------------
\section{Problem Statement}
\label{sec:problem-statement}

This thesis addresses the following comprehensive research problem:

\begin{quote}
	\textit{How can we design and implement an effective serious game for mobile platforms that teaches fundamental High-Performance Computing concepts (specifically OpenMP and MPI paradigms) through interactive card-sorting gameplay while overcoming the technical challenges of multiplayer state synchronization and mobile UI/UX constraints?}
\end{quote}

This overarching problem can be decomposed into several interrelated sub-problems:

\subsection{Educational Problem}
\label{subsec:educational-problem}

\begin{itemize}
	\item How can abstract parallel computing concepts be mapped to concrete, understandable game mechanics?
	\item How can a game effectively differentiate between shared-memory (OpenMP) and distributed-memory (MPI) paradigms?
	\item What game mechanics best illustrate concepts like parallelism, data distribution, synchronization, speedup, scalability, and communication overhead?
	\item How can the game provide meaningful learning without oversimplifying or creating misleading mental models?
\end{itemize}

\subsection{Technical Problem}
\label{subsec:technical-problem}

\begin{itemize}
	\item How can we implement real-time multiplayer gameplay on mobile devices with acceptable latency?
	\item How can we maintain consistent game state across multiple clients with potentially unreliable network connections?
	\item How can we display and manipulate numerous cards (potentially 50--200) on small mobile screens effectively?
	\item What networking architecture and synchronization framework best supports educational multiplayer gaming?
	\item How can we handle client disconnections and reconnections gracefully?
\end{itemize}

\subsection{Design Problem}
\label{subsec:design-problem}

\begin{itemize}
	\item How can we balance educational effectiveness with engagement and playability?
	\item How can we design intuitive touch-based interactions for complex drag-and-drop operations?
	\item How can we provide appropriate real-time feedback to reinforce learning objectives?
	\item What visual design supports both usability and educational clarity?
\end{itemize}

%----------------------------------------------------------------------------------------
\section{Research Objectives}
\label{sec:objectives}

The primary objective of this thesis is to develop a functional serious game prototype that demonstrates the feasibility and effectiveness of teaching HPC concepts through mobile gaming. This work aims to create an open-source platform suitable for future educational research and publication in conferences and scientific journals.

%----------------------------------------------------------------------------------------
\subsection{Primary Objectives}
\label{subsec:primary-objectives}

\begin{enumerate}
	\item \textbf{Develop an Interactive Serious Game}: Create a game that uses card sorting as a metaphor for parallel sorting algorithms, accurately mapping game mechanics to OpenMP and MPI paradigms.

	\item \textbf{Implement Web-First Functionality}: Develop the game primarily as a web application that runs in browsers, with responsive design that works across different screen sizes, devices, and orientations, ensuring accessibility for students worldwide.

	\item \textbf{Create Multiplayer Capability}: Implement real-time, peer-to-peer multiplayer functionality that enables multiple players (2--4) to collaborate or compete, simulating distributed computing scenarios with explicit message passing.

	\item \textbf{Ensure Educational Effectiveness}: Design gameplay that clearly communicates HPC concepts—speedup, scalability, communication overhead, parallelism patterns—and helps players understand the fundamental differences between parallel programming models.
\end{enumerate}

%----------------------------------------------------------------------------------------
\subsection{Secondary Objectives}
\label{subsec:secondary-objectives}

\begin{enumerate}
	\setcounter{enumi}{4}

	\item \textbf{Evaluate Technical Feasibility}: Assess the challenges and solutions for multiplayer game development on mobile platforms, particularly regarding real-time state synchronization using the GDSync framework and WebRTC communication.

	\item \textbf{Document Development Process}: Provide comprehensive documentation of technology choices, implementation challenges (especially framework issues), and solutions to guide future developers of educational games.

	\item \textbf{Create Extensible Architecture}: Design the system to allow future additions of more sorting algorithms, HPC concepts (barriers, race conditions, load balancing), and platform support (iOS, web).

	\item \textbf{Assess User Experience}: Gather insights into usability, engagement, and educational effectiveness through testing and iteration, preparing materials for publication.
\end{enumerate}

%----------------------------------------------------------------------------------------
\subsection{Tertiary Objectives}
\label{subsec:tertiary-objectives}

\begin{enumerate}
	\setcounter{enumi}{8}

	\item Evaluate technical performance on real Android devices (frame rates, network latency, battery usage)
	\item Gather preliminary feedback on educational effectiveness through informal testing
	\item Document design patterns and lessons learned for the serious games community
	\item Prepare materials for publication at relevant conferences and in scientific journals
\end{enumerate}

%----------------------------------------------------------------------------------------
\section{Proposed Solution: HPC Sorting Serious Game}
\label{sec:proposed-solution}

This thesis presents the \textbf{HPC Sorting Serious Game}, a web-first educational game that teaches parallel computing through an interactive card-sorting experience. The game uses a simple yet effective metaphor: sorting numbered cards represents sorting data in parallel computing systems. By manipulating virtual cards, students gain hands-on experience with parallelization strategies without the complexity of actual parallel programming.

%----------------------------------------------------------------------------------------
\subsection{Core Concept}
\label{subsec:core-concept}

The game presents players with a deck of numbered cards that must be sorted in ascending order. The fundamental mechanics differ based on the educational mode, each designed to simulate a specific parallel computing paradigm:

\subsubsection{Single-Player Mode (OpenMP Simulation):}
Represents shared-memory parallelism where multiple threads cooperate on a single address space:

\begin{itemize}
	\item Players receive all cards in a shared visible area (representing shared memory accessible to all threads)
	\item Players can use private ``buffer zones'' to temporarily store and locally sort subsets of cards (representing thread-local storage and private variables)
	\item Multiple players (simulating threads) work simultaneously without explicit communication requirements (representing OpenMP's implicit synchronization model)
	\item Players can access any card at any time (representing shared memory access patterns)
	\item No forced turn-taking occurs (representing the independent, concurrent nature of threads)
\end{itemize}

This mode teaches:
\begin{itemize}
	\item Work decomposition strategies
	\item The concept of private vs. shared data
	\item Local sorting and global merging patterns
	\item Independence of parallel threads
	\item Merge-based parallel sorting algorithms
\end{itemize}

\subsubsection{Multiplayer Mode (MPI Simulation)}
Represents distributed-memory parallelism where processes have separate address spaces and communicate explicitly:

\begin{itemize}
	\item Each player receives a different subset of cards (representing data distribution across processes)
	\item Players cannot directly see cards held by other players (representing separate memory spaces and lack of shared memory)
	\item Players must explicitly exchange cards or information (representing MPI message-passing operations like \texttt{MPI\_Send} and \texttt{MPI\_Recv})
	\item A master player or coordinator may collect sorted results (representing \texttt{MPI\_Gather} and reduction operations)
	\item Communication overhead is visible through network latency (representing the cost of inter-process communication)
\end{itemize}

This mode teaches:
\begin{itemize}
	\item Data distribution and ownership concepts
	\item Explicit communication requirements
	\item The master-worker programming pattern
	\item Message-passing overhead and latency
	\item Collective communication operations
	\item Load balancing challenges
\end{itemize}

%----------------------------------------------------------------------------------------
\subsection{Key Features}
\label{subsec:key-features}

The game incorporates several features designed to enhance both educational effectiveness and user engagement:

\begin{enumerate}
	\item \textbf{Intuitive Touch Interface}: Drag-and-drop mechanics optimized for mobile touchscreens with visual feedback during interactions

	\item \textbf{Real-Time Multiplayer}: WebRTC-based peer-to-peer communication for low-latency gameplay without dedicated servers

	\item \textbf{Visual Feedback}: Color-coded cards, smooth animations, and toast notifications to guide players and provide immediate feedback

	\item \textbf{Performance Tracking}: Timer and move counter to encourage algorithmic efficiency and enable comparison of different strategies

	\item \textbf{Scalable Difficulty}: Support for variable numbers of cards (10--200) and players (2--4), allowing progressive complexity

	\item \textbf{No Authentication Required}: Direct gameplay without registration or login barriers, reducing friction for educational use

	\item \textbf{Open Source}: Full source code available on GitHub for educational and research purposes

	\item \textbf{Cross-Platform Potential}: Built with Godot Engine, enabling future deployment to iOS, web, and desktop platforms
\end{enumerate}

%----------------------------------------------------------------------------------------
\subsection{Technology Stack}
\label{subsec:technology-stack}

The game is built using carefully selected open-source technologies:

\begin{itemize}
	\item \textbf{Godot Engine 4.x}: Open-source, lightweight game engine with excellent 2D capabilities and mobile support. Chosen for its simplicity, active community, and lack of licensing costs.

	\item \textbf{GDScript}: Python-like scripting language for rapid development. Selected for its ease of learning (relevant for students who may examine the source code) and tight integration with Godot.

	\item \textbf{GDSync Framework}: Multiplayer state synchronization framework for Godot, providing high-level abstractions for remote function calls and node replication.

	\item \textbf{WebRTC/NodeWebSockets}: Real-time communication protocols enabling peer-to-peer networking without dedicated servers, reducing infrastructure requirements.

	\item \textbf{Supporting Plugins}:
	      \begin{itemize}
		      \item ToastParty: User notification system for feedback
		      \item Logger: Debugging and development tool
		      \item VarTree: Runtime variable inspection for development
		      \item Scene-Selector: Scene management utilities
	      \end{itemize}
\end{itemize}

The rationale for these technology choices is detailed in Chapter~\ref{ch:methodology}.

%----------------------------------------------------------------------------------------
\subsection{Pedagogical Mapping}
\label{subsec:pedagogical-mapping}

Table~\ref{tab:pedagogical-mapping} illustrates how game mechanics map to HPC concepts, ensuring educational fidelity:

\begin{table}[htbp]
	\centering
	\caption{Mapping of HPC concepts to game mechanics}
	\label{tab:pedagogical-mapping}
	\begin{tabular}{@{}ll@{}}
		\toprule
		\textbf{HPC Concept}          & \textbf{Game Mechanic}                    \\
		\midrule
		OpenMP Parallel Region        & Single-player game session                \\
		Shared Memory                 & Main card container visible to all        \\
		Thread-Local Storage          & Private buffer zones                      \\
		Thread Independence           & No forced turn-taking, free card movement \\
		Work Distribution             & Choosing which cards to sort              \\
		Parallel Sorting (Merge Sort) & Sorting in buffers, then merging          \\
		\midrule
		MPI Process                   & Individual player in multiplayer mode     \\
		Distributed Memory            & Each player's private buffers             \\
		Message Passing               & Moving cards between players              \\
		Master-Worker Pattern         & One player coordinates final merging      \\
		Communication Overhead        & Network latency for card exchanges        \\
		Synchronization               & Coordinating who works on which cards     \\
		\bottomrule
	\end{tabular}
\end{table}

%----------------------------------------------------------------------------------------
\section{Thesis Contributions}
\label{sec:contributions}

This thesis makes several contributions to the fields of HPC education, serious game development, and mobile multiplayer architecture:

%----------------------------------------------------------------------------------------
\subsection{Pedagogical Contributions}
\label{subsec:pedagogical-contributions}

\begin{enumerate}
	\item \textbf{Novel Educational Approach}: Introduces a game-based method for teaching parallel computing that emphasizes hands-on, interactive learning over passive instruction, directly inspired by successful physical classroom experiments.

	\item \textbf{Clear Paradigm Differentiation}: Provides distinct gameplay modes that clearly illustrate the fundamental differences between shared-memory (OpenMP) and distributed-memory (MPI) parallel programming paradigms.

	\item \textbf{Accessible Learning Tool}: Creates a freely available, open-source educational resource that requires no specialized hardware, expensive licenses, or complex setup procedures.

	\item \textbf{Validated Pedagogical Mapping}: Demonstrates a concrete mapping between game mechanics and HPC concepts, validated through the lens of established classroom teaching methods.
\end{enumerate}

%----------------------------------------------------------------------------------------
\subsection{Technical Contributions}
\label{subsec:technical-contributions}

\begin{enumerate}
	\setcounter{enumi}{4}

	\item \textbf{Mobile Multiplayer Architecture}: Demonstrates effective patterns for implementing real-time multiplayer educational games on mobile platforms using peer-to-peer networking.

	\item \textbf{State Synchronization Solutions}: Documents approaches and solutions for maintaining consistency across multiple clients in educational gaming contexts, particularly handling private state (player buffers) alongside shared state (main card container).

	\item \textbf{Design Patterns for Godot}: Provides reusable patterns for serious game development using Godot Engine, including component architecture, scene management, and signal-based communication.

	\item \textbf{GDSync Framework Integration}: Contributes documentation and workarounds for the GDSync framework, including identification and resolution of framework issues, with contributions to the open-source project.
\end{enumerate}

%----------------------------------------------------------------------------------------
\subsection{Practical Contributions}
\label{subsec:practical-contributions}

\begin{enumerate}
	\setcounter{enumi}{8}

	\item \textbf{Working Prototype}: Delivers a functional, deployable game that can be immediately used in educational settings for teaching HPC concepts.

	\item \textbf{Comprehensive Documentation}: Provides detailed documentation of challenges encountered and solutions implemented, serving as a practical guide for future developers of educational multiplayer games.

	\item \textbf{Extensible Framework}: Creates an architecture that can be extended with additional sorting algorithms, HPC concepts (synchronization barriers, race conditions, load balancing), and platform support (iOS, web, desktop).

	\item \textbf{Open-Source Contribution}: Makes the complete codebase publicly available on GitHub, enabling other educators and researchers to build upon this work.
\end{enumerate}

%----------------------------------------------------------------------------------------
\section{Thesis Organization}
\label{sec:thesis-organization}

The remainder of this thesis is organized as follows:

\textbf{Chapter~\ref{ch:background}: Background and Literature Review} provides comprehensive context on parallel computing fundamentals (OpenMP and MPI), serious games in computer science education, mobile game development challenges, and multiplayer architecture patterns. It reviews related work in educational games and HPC teaching tools, identifying gaps that this thesis addresses.

\textbf{Chapter~\ref{ch:methodology}: Methodology} describes the research approach (Design Science Research methodology), comprehensive requirements analysis (educational, functional, and non-functional), and detailed justification for technology selections (Godot Engine, GDScript, GDSync, WebRTC). It explains the game design methodology for both OpenMP and MPI simulations and outlines the iterative development process.

\textbf{Chapter~\ref{ch:architecture}: System Design and Architecture} presents the high-level architecture of the game system, including detailed scene structure (Main Menu, Lobby, Single-Player, Multiplayer), core component design (Card Manager, Card Component, Buffer System, Timer), multiplayer architecture with host-authoritative state management, GDSync integration patterns, mobile UI/UX design considerations, and comprehensive data flow diagrams.

\textbf{Chapter~\ref{ch:implementation}: Implementation} details the technical implementation, including development environment setup, single-player implementation (card generation, layout, drag-and-drop, buffer zones, sorting validation), multiplayer implementation (GDSync setup, lobby system, state synchronization logic, conflict resolution), code structure and design patterns, mobile-specific optimizations, and debugging tools integration.

\textbf{Chapter~\ref{ch:problems}: Problems and Challenges} provides an honest and comprehensive analysis of difficulties encountered during development, including technology selection trade-offs, GDSync framework issues (protected mode blocking, documentation gaps, GitHub issue resolution), multiplayer synchronization challenges (card order synchronization, timing issues, different client views), mobile UI/UX constraints (displaying many cards, touch interaction precision), state management complexity, performance challenges, testing difficulties, and lessons learned with recommendations for future developers.

\textbf{Chapter~\ref{ch:results}: Results and Evaluation} presents the completed system features with screenshots and demonstrations, technical performance metrics (frame rates, network latency, memory usage, scalability), platform compatibility assessment, comparison with initial requirements, discussion of known limitations, and preliminary evaluation of educational effectiveness (if user studies were conducted).

\textbf{Chapter~\ref{ch:conclusion}: Conclusion and Future Work} summarizes the contributions of this thesis, reflects on how research questions were answered, discusses implications for HPC education and serious game development, outlines comprehensive future work directions (short-term improvements, feature extensions, platform expansions, educational enhancements, research directions), describes publication plans for conferences and journals, and provides final remarks on the potential impact of this work.

\textbf{Appendices} provide supplementary materials including a detailed user manual (installation, gameplay instructions, controls reference), key code listings (algorithms, class definitions, configuration files), API documentation (GDSync usage, plugin APIs), and study materials (if applicable).

%----------------------------------------------------------------------------------------

\chapter{Background and Literature Review}
\label{ch:background}

This chapter provides the foundational knowledge and contextual background necessary to understand the HPC Sorting Serious Game. We review key concepts in parallel computing, serious games in education, web-based game development, and multiplayer networking architectures.

%-----------------------------------------------------------------------
\section{Parallel Computing Fundamentals}
\label{sec:parallel-computing}

Parallel computing is the simultaneous execution of multiple computational tasks to solve problems more efficiently than sequential processing. As Moore's Law approaches its physical limits, parallel computing has become essential for improving computational performance across scientific computing, data analytics, and real-time systems.

%-----------------------------------------------------------------------
\subsection{Parallel Programming Paradigms}
\label{subsec:parallel-paradigms}

Two primary paradigms dominate parallel computing education and practice: shared-memory parallelism and distributed-memory parallelism.

\subsubsection{Shared-Memory Parallelism}

In shared-memory systems, multiple processors or cores access a common address space. All threads can read and write to the same memory locations, enabling efficient data sharing but requiring careful synchronization to prevent race conditions.

\textbf{Characteristics:}
\begin{itemize}
    \item All processors access the same physical memory
    \item Communication occurs through shared variables
    \item Synchronization primitives (locks, barriers, atomic operations) prevent conflicts
    \item Lower communication overhead compared to distributed memory
    \item Limited scalability due to memory bandwidth constraints
\end{itemize}

\textbf{Applications:} Shared-memory parallelism is ideal for problems requiring frequent communication between processing units, such as iterative algorithms, graph processing, and data analytics on multicore processors.

\subsubsection{Distributed-Memory Parallelism}

In distributed-memory systems, each processor has its own local memory. Processors communicate by explicitly sending and receiving messages over a network, requiring programmers to manage data distribution and communication patterns.

\textbf{Characteristics:}
\begin{itemize}
    \item Each processor has private memory space
    \item Communication requires explicit message passing
    \item Scalable to thousands or millions of processors
    \item Higher communication overhead and latency
    \item No shared state eliminates race conditions but complicates programming
\end{itemize}

\textbf{Applications:} Distributed-memory parallelism is essential for large-scale scientific simulations, weather modeling, molecular dynamics, and big data processing across cluster and supercomputer architectures.

%-----------------------------------------------------------------------
\subsection{OpenMP: Shared-Memory Programming}
\label{subsec:openmp}

OpenMP (Open Multi-Processing) is a widely-used API for shared-memory parallel programming in C, C++, and Fortran. It uses compiler directives (pragmas) to specify parallel regions, enabling incremental parallelization of sequential code.

\textbf{Core Concepts:}
\begin{description}
    \item[Parallel Regions] Code sections executed by multiple threads simultaneously
    \item[Work-Sharing Constructs] Distribute loop iterations or tasks among threads
    \item[Data Scoping] Variables can be shared (visible to all threads) or private (thread-local copies)
    \item[Synchronization] Barriers, critical sections, and atomic operations coordinate thread execution
    \item[Scheduling] Strategies (static, dynamic, guided) control how iterations are assigned to threads
\end{description}

\paragraph*{Example Code:}

\begin{lstlisting}[language=C++, caption={Simple OpenMP parallel loop}]
#pragma omp parallel for schedule(dynamic)
for (int i = 0; i < N; i++) {
    array[i] = compute_value(i);
}
\end{lstlisting}

\textbf{Educational Value:} OpenMP's simplicity makes it excellent for teaching parallel programming concepts. The incremental approach—starting with sequential code and adding pragmas—allows students to observe immediate performance improvements while learning about thread management, data dependencies, and race conditions.

%-----------------------------------------------------------------------
\subsection{MPI: Distributed-Memory Programming}
\label{subsec:mpi}

\begin{mdframed}[backgroundcolor=yellow!10, linecolor=orange, linewidth=2pt]
    \textbf{Note:} This section provides background on MPI for completeness and context within the broader HPC landscape. \textbf{The game implemented in this thesis focuses exclusively on OpenMP shared-memory parallelism.} MPI distributed-memory concepts are discussed here for educational context but are \textbf{not implemented} in the current version. See Chapter~\ref{ch:conclusion} for discussion of MPI as potential future work.
\end{mdframed}

MPI (Message Passing Interface) is the de facto standard for distributed-memory parallel programming. It provides a rich set of communication primitives for sending and receiving messages between processes, enabling scalable parallel computing on clusters and supercomputers.

\textbf{Core Operations:}
\begin{description}
    \item[Point-to-Point Communication] \texttt{MPI\_Send} and \texttt{MPI\_Recv} exchange messages between specific processes
    \item[Collective Communication] Operations like \texttt{MPI\_Bcast}, \texttt{MPI\_Gather}, and \texttt{MPI\_Reduce} involve all processes in a communicator
    \item[Process Groups] Communicators organize processes and define communication contexts
    \item[Data Types] MPI supports both primitive and user-defined data types for message content
\end{description}

\paragraph*{Example Code:}

\begin{lstlisting}[language=C++, caption={Simple MPI message passing}]
MPI_Init(&argc, &argv);
int rank, size;
MPI_Comm_rank(MPI_COMM_WORLD, &rank);
MPI_Comm_size(MPI_COMM_WORLD, &size);

if (rank == 0) {
    // Master process
    MPI_Send(data, count, MPI_INT, 1, 0, MPI_COMM_WORLD);
} else if (rank == 1) {
    // Worker process
    MPI_Recv(data, count, MPI_INT, 0, 0, MPI_COMM_WORLD, &status);
}

MPI_Finalize();
\end{lstlisting}

\textbf{Educational Value:} MPI teaches students about distributed computing challenges including data distribution, explicit communication, latency management, and scalability. The message-passing model forces programmers to think carefully about data ownership and communication patterns, skills essential for modern distributed systems.

%-----------------------------------------------------------------------
\subsection{Parallel Sorting Algorithms}
\label{subsec:parallel-sorting}

Sorting is a fundamental algorithmic problem that benefits significantly from parallelization. Several parallel sorting strategies exist, each with different communication and synchronization requirements.

\paragraph*{Parallel Merge Sort:}
\begin{itemize}
    \item Recursively divide array into subarrays
    \item Sort subarrays in parallel
    \item Merge sorted subarrays (potentially in parallel)
    \item Well-suited for shared-memory systems
    \item Time complexity: $O(\frac{n \log n}{p})$ with $p$ processors
\end{itemize}

\paragraph*{Sample Sort (MPI Pattern):}
\begin{itemize}
    \item Distribute data across processes
    \item Each process sorts local data
    \item Select sample elements to determine global pivots
    \item Redistribute data based on pivots
    \item Each process sorts its final partition
    \item Commonly used in distributed-memory systems
    \item \textit{Note: Not implemented in this thesis; discussed for context}
\end{itemize}

\textbf{Relevance to Educational Game:} The card-sorting game mechanics map to parallel sorting algorithms. In the implemented game, players act as threads (OpenMP) coordinating through shared memory to sort cards efficiently, making abstract algorithmic concepts tangible through gameplay. Future extensions could explore MPI-style distributed sorting (see Chapter~\ref{ch:conclusion}).

%-----------------------------------------------------------------------
\section{Serious Games in Education}
\label{sec:serious-games}

Serious games—games designed primarily for purposes beyond entertainment—have emerged as powerful educational tools. This section reviews the theoretical foundations and empirical evidence supporting game-based learning.

%-----------------------------------------------------------------------
\subsection{Definition and Characteristics}
\label{subsec:serious-games-definition}

\citet{zyda2005visual} defines serious games as games that ``do not have entertainment, enjoyment, or fun as their primary purpose.'' These games leverage game design principles to achieve learning objectives, skill development, or behavioral change.

\textbf{Key Characteristics:}
\begin{itemize}
    \item \textbf{Clear Learning Objectives}: Explicit educational goals aligned with curricula
    \item \textbf{Engaging Mechanics}: Game elements (challenges, rewards, feedback) maintain motivation
    \item \textbf{Active Learning}: Players learn by doing, not passive observation
    \item \textbf{Immediate Feedback}: Instant consequences of actions reinforce learning
    \item \textbf{Safe Experimentation}: Risk-free environment for trial and error
    \item \textbf{Progressive Difficulty}: Challenges scale with player skill development
\end{itemize}

%-----------------------------------------------------------------------
\subsection{Theoretical Foundations}
\label{subsec:game-based-learning-theory}

Game-based learning draws on several established educational theories:

\subsubsection{Constructivism}

Constructivist theory posits that learners actively construct knowledge through experience rather than passively receiving information. Games naturally support constructivist learning by placing players in problem-solving scenarios where they discover principles through experimentation.

\subsubsection{Flow Theory}

\citet{csikszentmihalyi1990flow} describes ``flow'' as the optimal psychological state where challenge level matches skill level, producing deep engagement. Well-designed educational games maintain flow by dynamically adjusting difficulty, keeping learners in the zone of proximal development.

\subsubsection{Experiential Learning}

Kolb's experiential learning cycle—concrete experience, reflective observation, abstract conceptualization, active experimentation—aligns perfectly with game mechanics. Players experience situations, observe outcomes, form hypotheses, and test them iteratively.

%-----------------------------------------------------------------------
\subsection{Empirical Evidence}
\label{subsec:serious-games-evidence}

Research consistently demonstrates the effectiveness of serious games for learning:

\paragraph*{\citet{connolly2012systematic} Meta-Analysis:}

A systematic literature review of 129 papers found that serious games positively impact learning outcomes across multiple domains. Key findings include:

\begin{itemize}
    \item Improved knowledge acquisition and retention
    \item Enhanced perceptual and cognitive skills
    \item Increased motivation and engagement
    \item Better affective and motivational outcomes
    \item Most effective when integrated with traditional instruction
\end{itemize}

\paragraph*{\citet{papastergiou2009digital} Computer Science Education:}

A study of digital game-based learning in high school computer science found that students using educational games demonstrated:

\begin{itemize}
    \item Significantly higher knowledge acquisition
    \item Greater enjoyment and motivation
    \item Preference for game-based learning over traditional methods
    \item Improved attitudes toward computer science
\end{itemize}

%-----------------------------------------------------------------------
\subsection{Serious Games in Computer Science}
\label{subsec:games-cs-education}

Several successful serious games target computer science education:

\paragraph*{CodeCombat:}
\begin{itemize}
    \item Teaches programming through RPG-style gameplay
    \item Players write real code to control characters
    \item Supports Python, JavaScript, and other languages
    \item Used in thousands of classrooms worldwide
\end{itemize}

\paragraph*{Lightbot:}
\begin{itemize}
    \item Introduces programming concepts through puzzle-solving
    \item Teaches sequencing, loops, and procedures
    \item No syntax required—visual programming interface
    \item Suitable for young learners
\end{itemize}

\paragraph*{Parallel Computing Games:}

Few games specifically target parallel computing education. Notable exceptions include:

\begin{itemize}
    \item \textbf{PARMACS Simulator}: Interactive visualization of parallel algorithms
    \item \textbf{ParSim}: Educational tool for parallel algorithm design
    \item \textbf{CUDA Learning Games}: GPU programming challenges
\end{itemize}

However, these tools typically focus on visualization rather than hands-on, game-like interaction, and few provide multiplayer experiences.

%-----------------------------------------------------------------------
\subsection{Serious Games for HPC Education}
\label{subsec:serious-games-hpc}

High-Performance Computing education faces unique challenges that serious games can address:

\paragraph*{Why Games for HPC?}

\begin{enumerate}
    \item \textbf{Abstract Concepts}: HPC involves invisible processes (threads, messages, synchronization) that are difficult to visualize through static diagrams
    \item \textbf{Temporal Dynamics}: Parallelism involves timing, race conditions, and coordination that unfold over time---games naturally model dynamic systems
    \item \textbf{Scale Mismatch}: Real HPC systems involve millions of operations per second; games can slow down and make these concepts observable
    \item \textbf{Hands-On Experience}: Students learn parallelism best by experiencing coordination challenges firsthand
\end{enumerate}

\paragraph*{Mapping HPC Concepts to Game Mechanics:}
The following mappings enable intuitive understanding of HPC concepts through gameplay (see Table~\ref{tab:hpc-game-mapping}).

\begin{table}[htbp]
    \centering
    \caption{HPC concepts mapped to game mechanics}
    \label{tab:hpc-game-mapping}
    \begin{tabular}{@{}ll@{}}
        \toprule
        \textbf{HPC Concept}   & \textbf{Game Mechanic}                   \\
        \midrule
        Threads/Processes      & Players                                  \\
        Shared Memory          & Shared game board/container              \\
        Thread-Local Storage   & Private player inventory                 \\
        Synchronization        & Turn-taking or barriers                  \\
        Race Conditions        & Conflicting simultaneous actions         \\
        Speedup                & Time comparison (1 player vs. N players) \\
        Communication Overhead & Time to transfer items between players   \\
        Load Balancing         & Equal distribution of work               \\
        \bottomrule
    \end{tabular}
\end{table}

\paragraph*{Advantages of Game-Based HPC Learning:}

\begin{itemize}
    \item \textbf{No Prerequisites}: Students can experience parallel concepts before learning programming syntax
    \item \textbf{Misconception Surfacing}: Games reveal incorrect mental models (e.g., ``more threads always faster'')
    \item \textbf{Quantifiable Learning}: Performance metrics (time, moves) provide concrete feedback
    \item \textbf{Collaborative Learning}: Multiplayer games naturally create discussion and peer learning
    \item \textbf{Scalable Delivery}: Web-based games reach students regardless of institutional HPC infrastructure
\end{itemize}

%-----------------------------------------------------------------------
\section{Web-Based Game Development}
\label{sec:web-game-dev}

Web browsers provide a universal platform for educational game deployment, eliminating installation barriers and enabling access from any device with a modern browser.

%-----------------------------------------------------------------------
\subsection{Web Platform Advantages}
\label{subsec:web-advantages}

\textbf{Accessibility:} Web games require no installation—students access the game via URL, eliminating software distribution challenges and platform compatibility concerns.

\textbf{Cross-Platform:} A single web build runs on Windows, macOS, Linux, and even mobile browsers, maximizing reach with minimal development overhead.

\textbf{Instant Updates:} Deploying updates is immediate—no app store approval processes or user-initiated updates required.

\textbf{Low Barrier:} Modern browsers include powerful JavaScript engines, WebGL for graphics, and APIs for real-time communication.

%-----------------------------------------------------------------------
\subsection{Web Development Challenges}
\label{subsec:web-challenges}

\textbf{Browser Sandboxing:} Security restrictions prevent raw socket access, UDP communication, and local peer discovery—complicating multiplayer networking.

\textbf{Performance Overhead:} WebAssembly and JavaScript have overhead compared to native code, though performance is acceptable for 2D games.

\textbf{Networking Limitations:} Traditional peer-to-peer networking (like WebRTC data channels) can be complex to configure and may not work reliably across all network configurations.

%-----------------------------------------------------------------------
\subsection{Game Engines for Web}
\label{subsec:web-game-engines}

Godot Engine emerged as the optimal choice for this project due to:

\begin{itemize}
    \item Completely free and open-source (MIT license)
    \item Excellent 2D rendering performance
    \item Lightweight web exports (smaller than Unity or Unreal)
    \item GDScript language similar to Python (accessible for students)
    \item Active community and extensive documentation
    \item Built-in HTML5 export templates
    \item Scene-based architecture promotes modular design
\end{itemize}

%-----------------------------------------------------------------------
\section{Multiplayer Game Architecture}
\label{sec:bg-multiplayer-architecture}

Multiplayer functionality is essential for simulating OpenMP shared-memory parallelism with multiple players collaborating on shared data. This section reviews common multiplayer architectures and networking technologies.

%-----------------------------------------------------------------------
\subsection{Client-Server Architecture}
\label{subsec:client-server}

Traditional multiplayer games use a client-server model:

\textbf{Characteristics:}
\begin{itemize}
    \item Dedicated server hosts game state
    \item Clients send inputs to server
    \item Server validates actions, updates state, broadcasts to clients
    \item Authoritative server prevents cheating
    \item Requires server infrastructure and maintenance
\end{itemize}

\textbf{Advantages:}
\begin{itemize}
    \item Strong consistency and cheat prevention
    \item Clients need not communicate directly
    \item Server can enforce rules and validate actions
\end{itemize}

\textbf{Disadvantages:}
\begin{itemize}
    \item Requires dedicated servers (cost, maintenance)
    \item Single point of failure
    \item Latency for all communications
    \item Scalability challenges with many concurrent games
\end{itemize}

%-----------------------------------------------------------------------
\subsection{Peer-to-Peer Architecture}
\label{subsec:p2p}

Peer-to-peer (P2P) networks eliminate dedicated servers by allowing clients to communicate directly:

\textbf{Characteristics:}
\begin{itemize}
    \item Clients connect directly to each other
    \item No central server required (except for matchmaking)
    \item One peer often designated as ``host'' for authority
    \item Lower latency for direct connections
    \item More complex connection establishment (NAT traversal)
\end{itemize}

\textbf{Advantages:}
\begin{itemize}
    \item No server infrastructure costs
    \item Lower latency between peers
    \item Decentralized—no single point of failure
    \item Scalable for small groups (2--8 players)
\end{itemize}

\textbf{Disadvantages:}
\begin{itemize}
    \item NAT traversal complexity
    \item Host has performance burden
    \item Cheat prevention more difficult
    \item Connection quality varies with peer networks
\end{itemize}

%-----------------------------------------------------------------------
\subsection{Web Transport Technologies}
\label{subsec:web-transport}

Web browsers offer several options for real-time communication, each with trade-offs:

\paragraph*{HTTP Polling:}
\begin{itemize}
    \item Client repeatedly requests updates from server
    \item Simple to implement, universally supported
    \item Higher latency and server load due to frequent requests
    \item Suitable for low-frequency updates
\end{itemize}

\paragraph*{Server-Sent Events (SSE):}
\begin{itemize}
    \item Server pushes events to client over HTTP
    \item Simpler than WebSocket (HTTP-based, no special protocol)
    \item Unidirectional (server to client only)
    \item Automatic reconnection built into browser API
    \item Good browser support and proxy compatibility
\end{itemize}

\paragraph*{HTTP + SSE Relay Architecture:}

For web-based multiplayer games requiring broad browser compatibility, a relay server architecture using HTTP POST for client-to-server communication and SSE for server-to-client events provides:

\begin{itemize}
    \item Universal browser support without WebSocket complexity
    \item Works through firewalls and proxies that block WebSocket
    \item Simplified connection establishment (no STUN/TURN required)
    \item Trade-off: All traffic routes through server (no P2P)
    \item Acceptable latency for turn-based or slower-paced games
\end{itemize}

%-----------------------------------------------------------------------
\subsection{State Synchronization Patterns}
\label{subsec:state-sync}

Maintaining consistent game state across multiple clients is challenging. Common patterns include:

\paragraph*{Deterministic Lockstep:}
\begin{itemize}
    \item All clients simulate the same game logic
    \item Inputs synchronized and executed simultaneously
    \item Guarantees identical state on all clients
    \item Sensitive to latency and packet loss
    \item Used in RTS games (Age of Empires, StarCraft)
\end{itemize}

\paragraph*{Client-Side Prediction:}
\begin{itemize}
    \item Clients predict results of actions immediately
    \item Server validates and corrects if necessary
    \item Provides responsive feel despite latency
    \item Requires reconciliation when predictions wrong
    \item Used in FPS games (Counter-Strike, Overwatch)
\end{itemize}

\paragraph*{Authoritative Server/Host:}
\begin{itemize}
    \item Server/host maintains canonical state
    \item Clients send actions, receive state updates
    \item Simple to implement and reason about
    \item Higher latency but stronger consistency
    \item Suitable for turn-based or slower-paced games
\end{itemize}

\paragraph*{Selected Approach:}

For the HPC Sorting Game, a \textbf{host-authoritative model} was chosen because:

\begin{itemize}
    \item Educational focus prioritizes clarity over competitive performance
    \item Collaborative nature tolerates moderate latency
    \item Simpler implementation reduces development complexity
    \item Prevents inconsistencies and ensures fair learning environment
    \item Aligns with centralized coordination patterns in parallel systems
\end{itemize}

%-----------------------------------------------------------------------
\section{Related Work}
\label{sec:related-work}

This section reviews existing educational tools and games related to parallel computing and serious games.

%-----------------------------------------------------------------------
\subsection{Parallel Computing Educational Tools}
\label{subsec:parallel-tools}

Several tools exist for teaching parallel computing, each with strengths and limitations:

\paragraph*{PARMACS:}
\begin{itemize}
    \item Parallel algorithm visualization tool
    \item Shows thread/process execution timelines
    \item Limited interactivity—primarily observational
    \item Desktop-only platform
\end{itemize}

\paragraph*{OpenMP Visualization Tools:}
\begin{itemize}
    \item Intel VTune, TAU (Tuning and Analysis Utilities)
    \item Focused on performance profiling
    \item Steep learning curve for students
    \item Not designed for introductory education
\end{itemize}

\paragraph*{MPI Tools:}
\begin{itemize}
    \item MPICH, Open MPI with debugging tools
    \item Require cluster access or virtual machines
    \item Focus on actual parallel programming, not conceptual learning
    \item \textit{Note: MPI concepts discussed for context; not covered in this thesis}
\end{itemize}

\paragraph*{Gap Identification:}

Existing tools generally fall into two categories:

\begin{enumerate}
    \item \textbf{Visualization tools}: Show parallel execution but lack hands-on interaction
    \item \textbf{Programming environments}: Require coding skills, deterring beginners
\end{enumerate}

Few tools bridge the gap between conceptual understanding and practical implementation. Fewer still are designed for web platforms or incorporate game mechanics to enhance engagement.

%-----------------------------------------------------------------------
\subsection{Serious Games for Computing Education}
\label{subsec:cs-games}

Several serious games successfully teach computing concepts:

\paragraph*{CodeCombat:}
\begin{itemize}
    \item Teaches programming through RPG gameplay
    \item Players write real code in Python/JavaScript
    \item Excellent for syntax and basic algorithms
    \item Does not address parallel computing
\end{itemize}

\paragraph*{Human Resource Machine:}
\begin{itemize}
    \item Puzzle game teaching assembly-like programming
    \item Visual programming—no syntax barriers
    \item Excellent for teaching sequential thinking
    \item Single-threaded—no parallelism concepts
\end{itemize}

\paragraph*{Shenzhen I/O:}
\begin{itemize}
    \item Hardware design and programming game
    \item Teaches low-level optimization
    \item Complex and targeted at experienced programmers
    \item No explicit parallel computing focus
\end{itemize}

\paragraph*{Gap in Parallel Computing Games:}

No widely-adopted serious game specifically teaches parallel computing concepts like OpenMP shared-memory parallelism through interactive, game-based mechanics on web platforms. This gap motivated the development of the HPC Sorting Serious Game, which focuses on OpenMP concepts with potential for future MPI extensions.

%-----------------------------------------------------------------------
\subsection{Physical Teaching Methods}
\label{subsec:physical-methods}

The pedagogical approach for this thesis draws inspiration from physical classroom activities:

\paragraph*{CS Unplugged:}
\begin{itemize}
    \item Collection of unplugged computer science activities
    \item Teaches algorithms, sorting, searching without computers
    \item Highly engaging but not scalable or repeatable
    \item Limited to in-person classroom settings
\end{itemize}

\paragraph*{Professor D'Agostino's Card Sorting Experiments:}

As described in Chapter~\ref{ch:introduction}, Professor Daniele D'Agostino conducted classroom experiments where students physically sorted numbered cards to simulate parallel computing paradigms. These experiments demonstrated:

\begin{itemize}
    \item High student engagement and retention
    \item Intuitive understanding of parallelism concepts
    \item Memorable hands-on experience
    \item Clear differentiation between sequential and parallel execution models
\end{itemize}

However, physical experiments have limitations:

\begin{itemize}
    \item Require physical classroom presence
    \item Not scalable to large classes
    \item Difficult to reproduce consistently
    \item Limited flexibility in parameters (card count, student count)
    \item No performance metrics or automated feedback
\end{itemize}

\paragraph*{Digital Transformation:}

This thesis aims to capture the pedagogical effectiveness of physical card-sorting experiments in a digital, web-first serious game that overcomes the limitations of physical activities while preserving their educational benefits.

%-----------------------------------------------------------------------
\section{Summary}
\label{sec:background-summary}

This chapter established the foundational knowledge necessary to understand the HPC Sorting Serious Game:

\begin{itemize}
    \item \textbf{Parallel Computing}: Reviewed OpenMP shared-memory parallelism (the focus of this thesis) and MPI distributed-memory paradigms (for context), highlighting their educational challenges and importance.

    \item \textbf{Serious Games}: Examined theoretical foundations, empirical evidence, and successful examples of game-based learning in computer science education.

    \item \textbf{Web Development}: Discussed advantages and challenges of web platforms for educational games, and the rationale for selecting Godot Engine with web export capabilities.

    \item \textbf{Multiplayer Architecture}: Compared client-server and peer-to-peer models, introduced web transport technologies (HTTP/SSE), and justified the host-authoritative approach for OpenMP simulation.

    \item \textbf{Related Work}: Identified gaps in existing parallel computing educational tools and serious games, motivating the development of this project.
\end{itemize}

The next chapter describes the research methodology, requirements analysis, and technology selection decisions that guided the development of the HPC Sorting Serious Game.

\chapter{Methodology}
\label{ch:methodology}

This chapter describes the research methodology, requirements analysis, design approach, and technology selection decisions that guided the development of the HPC Sorting Serious Game. The methodology follows a Design Science Research approach, emphasizing iterative development and evaluation.

%-----------------------------------------------------------------------
\section{Research Methodology}
\label{sec:research-methodology}

%-----------------------------------------------------------------------
\subsection{Design Science Research Framework}
\label{subsec:design-science}

This thesis adopts the Design Science Research (DSR) methodology \citep{hevner2004design, peffers2007design}, which is particularly well-suited for developing innovative artifacts that address practical problems while contributing to theoretical knowledge.

\paragraph*{DSR Process Model:}

The DSR process consists of six iterative activities:

\begin{enumerate}
    \item \textbf{Problem Identification and Motivation}: Identify the research problem (teaching HPC concepts) and justify the value of a solution (serious game on mobile platforms).

    \item \textbf{Define Objectives of a Solution}: Infer solution objectives from problem definition and existing knowledge (create engaging, accessible, educational multiplayer game).

    \item \textbf{Design and Development}: Create the artifact (HPC Sorting Serious Game) by selecting appropriate technologies and implementing game mechanics.

    \item \textbf{Demonstration}: Demonstrate the artifact's utility by applying it to the problem context (students learning parallel computing).

    \item \textbf{Evaluation}: Observe and measure how well the artifact supports a solution to the problem (usability testing, performance benchmarking).

    \item \textbf{Communication}: Communicate the problem, artifact, evaluation, and contributions to relevant audiences (thesis, potential publications).
\end{enumerate}

\paragraph*{Rationale for DSR:}

DSR is appropriate for this research because:

\begin{itemize}
    \item The goal is to create a functional artifact (serious game) rather than merely analyze existing phenomena
    \item The problem is practical and relevant to HPC education
    \item The solution requires integrating knowledge from multiple domains (pedagogy, game design, parallel computing, mobile development)
    \item Evaluation can be both utility-based (does it work?) and design-based (is it well-constructed?)
\end{itemize}

%-----------------------------------------------------------------------
\subsection{Development Approach}
\label{subsec:development-approach}

The game development followed an iterative, incremental approach:

\paragraph*{Phase 1: Concept and Prototyping (Weeks 1--3)}
\begin{itemize}
    \item Defined core mechanics based on physical card-sorting experiments
    \item Created paper prototypes and mockups
    \item Evaluated technology options (game engines, frameworks)
    \item Developed minimal prototype with basic card rendering
\end{itemize}

\paragraph*{Phase 2: Single-Player Implementation (Weeks 4--7)}
\begin{itemize}
    \item Implemented card generation, shuffling, and rendering
    \item Developed drag-and-drop mechanics for mobile touch input
    \item Created buffer zones for private card storage
    \item Implemented sorting validation and game completion logic
    \item Added timer and move counting
\end{itemize}

\paragraph*{Phase 3: Multiplayer Foundation (Weeks 8--12)}
\begin{itemize}
    \item Integrated WebRTC for peer-to-peer connections
    \item Implemented lobby system for matchmaking
    \item Developed basic state synchronization
    \item Integrated GDSync framework
    \item Tested connection establishment and stability
\end{itemize}

\paragraph*{Phase 4: Multiplayer Refinement (Weeks 13--16)}
\begin{itemize}
    \item Implemented card visibility management (private buffers)
    \item Developed host-authoritative state management
    \item Added disconnection handling and recovery
    \item Refined synchronization for card movements
    \item Addressed framework issues (GDSync protected mode)
\end{itemize}

\paragraph*{Phase 5: Polish and Testing (Weeks 17--20)}
\begin{itemize}
    \item UI/UX improvements for mobile usability
    \item Performance optimization (rendering, networking)
    \item Extensive testing on real Android devices
    \item Documentation and code cleanup
    \item Preparation for evaluation and thesis writing
\end{itemize}

%-----------------------------------------------------------------------
\section{Requirements Analysis}
\label{sec:requirements}

A comprehensive requirements analysis was conducted to ensure the game met educational, functional, and non-functional objectives.

%-----------------------------------------------------------------------
\subsection{Educational Requirements}
\label{subsec:educational-requirements}

The primary purpose of the game is education. Requirements were derived from HPC learning objectives and parallel computing curricula:

\paragraph*{ER1: OpenMP Simulation}
\begin{itemize}
    \item The game must simulate shared-memory parallelism
    \item All players must see all cards in the main container (shared memory)
    \item Players must have private buffers (thread-local storage)
    \item No explicit communication required between players (implicit synchronization)
\end{itemize}

\paragraph*{ER2: Sequential Execution Baseline}
\begin{itemize}
    \item The game must provide a single-player mode for sequential execution comparison
    \item A single player must sort all cards without parallelism
    \item This mode serves as a baseline for understanding speedup from parallelization
    \item Players should be able to compare sequential vs. parallel performance
\end{itemize}

\paragraph*{ER3: Performance Feedback}
\begin{itemize}
    \item The game must provide timing information (simulating speedup measurement)
    \item Move counting must encourage algorithmic efficiency
    \item Visual feedback must reinforce correct and incorrect actions
\end{itemize}

\paragraph*{ER4: Progressive Difficulty}
\begin{itemize}
    \item Support variable numbers of cards (10--200) for scalability lessons
    \item Support variable numbers of players (2--4) for parallelism degree
    \item Provide different sorting challenges (random, partially sorted, reverse sorted)
\end{itemize}

\paragraph*{ER5: Conceptual Clarity}
\begin{itemize}
    \item Game mechanics must clearly map to HPC concepts
    \item Visual design must distinguish shared vs. private data
    \item Terminology should align with parallel computing vocabulary when appropriate
\end{itemize}

%-----------------------------------------------------------------------
\subsection{Functional Requirements}
\label{subsec:functional-requirements}

Functional requirements specify what the system must do:

\paragraph*{FR1: Card Management}
\begin{itemize}
    \item Generate configurable numbers of cards with unique values
    \item Shuffle cards randomly at game start
    \item Render cards with clear visual representation (number, state)
    \item Support dragging and dropping cards between containers
\end{itemize}

\paragraph*{FR2: Single-Player Mode}
\begin{itemize}
    \item Provide a single-player game mode representing sequential execution
    \item Display all cards in a scrollable container
    \item Provide work buffer zones for temporary storage
    \item Validate sorting order upon completion
    \item Display completion time and move count
\end{itemize}

\paragraph*{FR3: Multiplayer Mode}
\begin{itemize}
    \item Enable 2--4 players to connect via P2P networking
    \item Provide lobby for matchmaking and game configuration
    \item Display shared container visible to all players (OpenMP shared memory)
    \item Synchronize card movements across all clients
    \item Provide private buffers for each player (thread-local storage)
    \item Hide cards in other players' private buffers (thread-private data)
    \item Handle player disconnections gracefully
\end{itemize}

\paragraph*{FR4: User Interface}
\begin{itemize}
    \item Main menu with navigation to single-player and multiplayer modes
    \item Lobby interface for creating/joining games
    \item In-game UI with timer, move counter, player indicators
    \item Victory screen with performance summary
    \item Settings for theme, controls, and game parameters
\end{itemize}

\paragraph*{FR5: Feedback Systems}
\begin{itemize}
    \item Toast notifications for events (player joined, card moved, errors)
    \item Visual animations for card pickup, drop, invalid moves
    \item Sound effects for actions (optional, can be disabled)
    \item Confetti or celebration effects upon game completion
\end{itemize}

%-----------------------------------------------------------------------
\subsection{Non-Functional Requirements}
\label{subsec:nonfunctional-requirements}

Non-functional requirements specify quality attributes and constraints:

\paragraph*{NFR1: Performance}
\begin{itemize}
    \item Target 60 FPS on mid-range Android devices (2020+)
    \item Game initialization and card generation under 2 seconds
    \item Network latency tolerance up to 200ms for playable experience
    \item Support up to 200 cards without significant performance degradation
\end{itemize}

\paragraph*{NFR2: Usability}
\begin{itemize}
    \item Intuitive touch interface requiring no tutorial for basic gameplay
    \item Responsive drag-and-drop with visual feedback
    \item Readable text and cards on 5-inch screens
    \item Support both portrait and landscape orientations
\end{itemize}

\paragraph*{NFR3: Reliability}
\begin{itemize}
    \item No crashes during normal gameplay
    \item Graceful handling of network disconnections
    \item State recovery after brief network interruptions
    \item Data validation to prevent inconsistent game states
\end{itemize}

\paragraph*{NFR4: Portability}
\begin{itemize}
    \item Primary target: Android 7.0+ (API level 24+)
    \item Desirable: iOS 12+ support
    \item Desirable: Web export (HTML5) for browser play
    \item Architecture should allow future platform additions
\end{itemize}

\paragraph*{NFR5: Maintainability}
\begin{itemize}
    \item Clean code structure with clear separation of concerns
    \item Comprehensive inline documentation and comments
    \item Modular design allowing feature additions
    \item Version-controlled codebase (Git)
    \item Open-source license (MIT) for educational reuse
\end{itemize}

\paragraph*{NFR6: Accessibility}
\begin{itemize}
    \item No account creation or authentication required
    \item Free to download and play
    \item Minimal permissions (network access only)
    \item Offline single-player mode
    \item Low data usage for multiplayer
\end{itemize}

%-----------------------------------------------------------------------
\section{Technology Selection}
\label{sec:technology-selection}

Technology choices significantly impact development efficiency, game performance, and future extensibility. This section justifies the selection of each major technology component.

%-----------------------------------------------------------------------
\subsection{Game Engine: Godot 4.x}
\label{subsec:godot-selection}

\paragraph*{Evaluation Criteria:}

Game engines were evaluated based on:
\begin{itemize}
    \item License and cost
    \item 2D rendering capabilities
    \item Mobile platform support
    \item Learning curve and documentation
    \item Community support and plugin ecosystem
    \item Multiplayer/networking support
    \item Performance on mobile devices
\end{itemize}

\paragraph*{Alternatives Considered:}

\begin{table}[htbp]
    \centering
    \caption{Game engine comparison for HPC Sorting Game}
    \label{tab:engine-comparison}
    \begin{tabular}{@{}lcccc@{}}
        \toprule
        \textbf{Criterion} & \textbf{Unity} & \textbf{Unreal} & \textbf{Godot} & \textbf{Cocos2d} \\
        \midrule
        Free/Open Source   & Partial        & Partial         & Yes            & Yes              \\
        2D Performance     & Good           & Fair            & Excellent      & Excellent        \\
        Mobile Support     & Excellent      & Good            & Excellent      & Good             \\
        Learning Curve     & Medium         & Steep           & Gentle         & Medium           \\
        Networking APIs    & Good           & Excellent       & Good           & Fair             \\
        Community Size     & Large          & Large           & Medium         & Small            \\
        File Size (APK)    & Large          & Very Large      & Small          & Medium           \\
        \bottomrule
    \end{tabular}
\end{table}

\paragraph*{Decision Rationale:}

Godot Engine was selected because:

\begin{enumerate}
    \item \textbf{Cost}: Completely free and open-source (MIT license), no royalties, subscriptions, or hidden fees. Critical for educational projects with no budget.

    \item \textbf{2D Optimization}: Godot is specifically optimized for 2D games, unlike Unity and Unreal which prioritize 3D. This results in better performance and smaller APK sizes.

    \item \textbf{Lightweight}: Export builds are significantly smaller (10--20 MB) compared to Unity (50--100 MB) or Unreal (100+ MB), important for mobile distribution.

    \item \textbf{GDScript}: Python-like scripting language is accessible to students who may examine the source code. Lower barrier to entry than C\# (Unity) or C++ (Unreal).

    \item \textbf{Scene Architecture}: Godot's scene-based architecture promotes modular design and code reuse, aligning well with software engineering best practices.

    \item \textbf{Built-in Multiplayer}: High-level networking APIs simplify P2P connection management, though third-party frameworks (GDSync) provide additional abstractions.

    \item \textbf{Active Community}: Growing community with extensive tutorials, documentation, and plugins. Version 4.x introduced significant improvements over 3.x.

    \item \textbf{Ethical Alignment}: Open-source philosophy aligns with educational values and ensures long-term availability without corporate dependencies.
\end{enumerate}

\paragraph*{Trade-offs:}

Godot's limitations include:
\begin{itemize}
    \item Smaller community and fewer third-party assets compared to Unity
    \item Fewer commercial games as reference implementations
    \item Some plugins less mature than Unity Asset Store equivalents
    \item 3D capabilities lag behind Unity/Unreal (not relevant for this project)
\end{itemize}

%-----------------------------------------------------------------------
\subsection{Programming Language: GDScript}
\label{subsec:gdscript-selection}

\paragraph*{Language Options in Godot:}

Godot supports three primary programming approaches:
\begin{enumerate}
    \item \textbf{GDScript}: Native scripting language, Python-like syntax
    \item \textbf{C\#}: .NET support via Mono runtime
    \item \textbf{C++}: Native modules via GDNative/GDExtension
\end{enumerate}

\paragraph*{Decision Rationale:}

GDScript was chosen because:

\begin{itemize}
    \item \textbf{Integration}: Tight integration with Godot Engine, optimized for game scripting
    \item \textbf{Simplicity}: Python-like syntax is readable and approachable for students
    \item \textbf{Rapid Development}: Dynamic typing and concise syntax accelerate prototyping
    \item \textbf{Documentation}: Most Godot tutorials and examples use GDScript
    \item \textbf{Performance}: Sufficient for 2D game logic; bottlenecks typically in rendering, not scripting
    \item \textbf{No Build Step}: Interpreted language—no compilation required for iteration
\end{itemize}

\paragraph*{Performance Considerations:}

GDScript is slower than C++ but acceptable for this use case because:
\begin{itemize}
    \item Card sorting logic is not computationally intensive
    \item Godot's rendering engine (written in C++) handles performance-critical operations
    \item Network latency dominates over local computation time
    \item Development speed more important than raw execution speed
\end{itemize}

%-----------------------------------------------------------------------
\subsection{Multiplayer Framework: GDSync}
\label{subsec:gdsync-selection}

\paragraph*{Multiplayer Implementation Options:}

Three approaches were considered for multiplayer networking:

\begin{enumerate}
    \item \textbf{Raw Godot Networking}: Use built-in \texttt{ENetMultiplayerPeer} and RPC system
    \item \textbf{GDSync Framework}: Third-party addon providing high-level synchronization abstractions
    \item \textbf{Custom Protocol}: Build custom WebRTC integration from scratch
\end{enumerate}

\paragraph*{Decision Rationale:}

GDSync was selected because:

\begin{itemize}
    \item \textbf{Abstraction Level}: Provides higher-level abstractions than raw Godot networking, reducing boilerplate code
    \item \textbf{State Synchronization}: Built-in support for synchronizing node properties automatically
    \item \textbf{RPC Simplification}: Cleaner RPC syntax and management compared to Godot's built-in system
    \item \textbf{Active Development}: Maintained project with community support
    \item \textbf{Documentation}: Adequate documentation and examples for common use cases
    \item \textbf{Compatibility}: Works well with Godot 4.x and WebRTC connections
\end{itemize}

\paragraph*{Challenges Encountered:}

GDSync introduced some difficulties:

\begin{itemize}
    \item \textbf{Protected Mode Issue}: Default "protected" mode blocked RPC calls between peers, requiring workarounds (see Chapter~\ref{ch:problems})
    \item \textbf{Documentation Gaps}: Some features underdocumented or unclear
    \item \textbf{Debugging}: Opaque error messages complicated troubleshooting
    \item \textbf{Framework Overhead}: Additional layer of abstraction sometimes obscured underlying networking behavior
\end{itemize}

Despite these challenges, GDSync provided net benefits by significantly reducing the amount of low-level networking code required.

%-----------------------------------------------------------------------
\subsection{Networking Technology: WebRTC}
\label{subsec:webrtc-selection}

\paragraph*{Alternatives Considered:}

\begin{itemize}
    \item \textbf{WebSocket}: Reliable, bi-directional communication; requires server infrastructure
    \item \textbf{TCP Sockets}: Low-level control; complex NAT traversal, no built-in encryption
    \item \textbf{WebRTC}: Peer-to-peer data channels with NAT traversal; modern standard
\end{itemize}

\paragraph*{Decision Rationale:}

WebRTC was chosen because:

\begin{enumerate}
    \item \textbf{Serverless Gameplay}: P2P connections eliminate server costs for actual gameplay
    \item \textbf{NAT Traversal}: Built-in STUN/TURN support handles firewall and NAT issues
    \item \textbf{Low Latency}: Direct peer connections minimize round-trip time
    \item \textbf{Security}: Encrypted by default (DTLS)
    \item \textbf{Cross-Platform}: Available on Android, iOS, web browsers, desktop
    \item \textbf{Godot Support}: Good integration via official WebRTC plugin and GDSync
\end{enumerate}

\paragraph*{Signaling Server:}

WebRTC requires a signaling server only for connection establishment:
\begin{itemize}
    \item Lightweight Node.js server for exchanging connection information
    \item Minimal infrastructure cost (can use free-tier hosting)
    \item Does not relay game data—only coordinates initial handshake
    \item Room-based matchmaking with unique codes
\end{itemize}

%-----------------------------------------------------------------------
\subsection{Supporting Plugins and Tools}
\label{subsec:supporting-tools}

Several Godot plugins were integrated to enhance development:

\paragraph*{ToastParty:}
\begin{itemize}
    \item Purpose: Display non-intrusive notification messages
    \item Use: Player joined/left, card moved, errors, game events
    \item Benefit: Provides feedback without blocking gameplay
\end{itemize}

\paragraph*{Logger:}
\begin{itemize}
    \item Purpose: Enhanced debugging output with categorization
    \item Use: Development troubleshooting and performance analysis
    \item Benefit: Structured logging with filtering capabilities
\end{itemize}

\paragraph*{VarTree:}
\begin{itemize}
    \item Purpose: Runtime variable inspection and modification
    \item Use: Debugging game state during development
    \item Benefit: Observe and manipulate variables without recompiling
\end{itemize}

\paragraph*{Scene-Selector:}
\begin{itemize}
    \item Purpose: Quick scene navigation during development
    \item Use: Switch between game scenes without main menu navigation
    \item Benefit: Faster iteration during development
\end{itemize}

%-----------------------------------------------------------------------
\section{Game Design Methodology}
\label{sec:game-design}

%-----------------------------------------------------------------------
\subsection{Mapping HPC Concepts to Game Mechanics}
\label{subsec:concept-mapping}

The core challenge was translating abstract parallel computing concepts into tangible game mechanics. Table~\ref{tab:concept-mapping-methodology} shows the systematic mapping process:

\begin{table}[htbp]
    \centering
    \caption{Systematic mapping of HPC concepts to game mechanics}
    \label{tab:concept-mapping-methodology}
    \begin{tabular}{@{}p{3.5cm}p{4cm}p{5cm}@{}}
        \toprule
        \textbf{HPC Concept}   & \textbf{Real-World Analogy}        & \textbf{Game Mechanic}             \\
        \midrule
        Sequential Execution   & Single student working alone       & Single player in solo mode         \\
        Thread (OpenMP)        & Student with access to shared desk & Player in multiplayer mode         \\
        Shared Memory          & Desk visible to all students       & Shared card container              \\
        Thread-Local Storage   & Student's private notebook         & Player's private buffer zones      \\
        Parallel Execution     & Students work simultaneously       & Multiple players, no turn-taking   \\
        Synchronization        & Students coordinate access         & Shared container access patterns   \\
        Memory Access Overhead & Time to reach shared desk          & Network latency to shared data     \\
        Speedup                & Multiple students finish faster    & Timer comparison: solo vs. multi   \\
        Load Balancing         & Even distribution of work          & Fair work distribution among players \\
        \bottomrule
    \end{tabular}
\end{table}

\paragraph*{Design Validation:}

The pedagogical mapping was validated through:
\begin{itemize}
    \item Comparison with Professor D'Agostino's physical experiments
    \item Review by HPC instructors for conceptual accuracy
    \item Pilot testing with students familiar with parallel computing
    \item Iterative refinement based on feedback
\end{itemize}

%-----------------------------------------------------------------------
\subsection{User Experience Design}
\label{subsec:ux-design}

\paragraph*{Design Principles:}

\begin{enumerate}
    \item \textbf{Minimize Cognitive Load}: Simple, clear visuals; avoid information overload
    \item \textbf{Immediate Feedback}: Every action produces visual/audio response
    \item \textbf{Progressive Disclosure}: Show information when needed, hide complexity initially
    \item \textbf{Error Prevention}: Design interface to prevent common mistakes
    \item \textbf{Aesthetic Simplicity}: Clean, uncluttered design focuses attention on cards
\end{enumerate}

\paragraph*{Interaction Design:}

Card interactions were designed around mobile touch paradigms:

\begin{itemize}
    \item \textbf{Tap to Select}: Quick tap selects a card (alternative to drag)
    \item \textbf{Drag to Move}: Long-press initiates drag, release drops card
    \item \textbf{Swipe to Scroll}: Vertical swipe scrolls card container
    \item \textbf{Visual Feedback}: Cards scale up when dragged, drop zones highlight
    \item \textbf{Haptic Feedback}: Vibration on card pickup (optional)
\end{itemize}

\paragraph*{Color Coding:}

Colors convey semantic meaning:

\begin{itemize}
    \item \textbf{Blue}: Normal cards, main container
    \item \textbf{Yellow}: Highlighted cards during drag
    \item \textbf{Gray}: Disabled/hidden cards (other players' buffers)
    \item \textbf{Purple}: Player-specific UI elements (buffers, indicators)
\end{itemize}

%-----------------------------------------------------------------------
% Note: Difficulty progression (Easy/Medium/Hard modes) was not implemented
% in the current version. This is discussed as future work in Chapter 8.

%-----------------------------------------------------------------------
\section{Evaluation Methodology}
\label{sec:evaluation-methodology}

\paragraph*{Evaluation Criteria:}

The game's success was evaluated across multiple dimensions:

\begin{enumerate}
    \item \textbf{Technical Performance}: Frame rate, network latency, memory usage, battery consumption
    \item \textbf{Functional Completeness}: Implementation of all required features
    \item \textbf{Usability}: Ease of use, learnability, satisfaction
    \item \textbf{Educational Effectiveness}: Conceptual understanding, engagement, learning outcomes
\end{enumerate}

\paragraph*{Evaluation Methods:}

\subparagraph{Technical Benchmarking:}
\begin{itemize}
    \item Performance profiling using Godot's built-in tools
    \item Frame rate monitoring on target devices
    \item Network latency measurement with varying connection qualities
    \item Memory usage tracking during extended gameplay
\end{itemize}

\subparagraph{Usability Testing:}
\begin{itemize}
    \item Think-aloud protocols with test users
    \item Observation of first-time players
    \item Task completion rates and times
    \item System Usability Scale (SUS) questionnaire (if time permits)
\end{itemize}

\subparagraph{Educational Assessment:}
\begin{itemize}
    \item Pre/post-test knowledge assessments (if formal study conducted)
    \item Concept mapping to evaluate mental model formation
    \item Comparison with traditional teaching methods (if feasible)
    \item Qualitative feedback on perceived learning value
\end{itemize}

%-----------------------------------------------------------------------
\section{Development Tools and Environment}
\label{sec:dev-environment}

\paragraph*{Development Platform:}
\begin{itemize}
    \item \textbf{OS}: Windows 11 / Linux (Ubuntu 22.04)
    \item \textbf{IDE}: Visual Studio Code with GDScript extension
    \item \textbf{Engine}: Godot Engine 4.2.x
    \item \textbf{Version Control}: Git with GitHub repository
\end{itemize}

\paragraph*{Testing Devices:}
\begin{itemize}
    \item Primary: Google Pixel 6 (Android 13)
    \item Secondary: Samsung Galaxy A52 (Android 12)
    \item Tertiary: OnePlus 8T (Android 11)
    \item Emulators: Android Studio AVD for additional testing
\end{itemize}

\paragraph*{Build and Deployment:}
\begin{itemize}
    \item Android APK built via Godot export templates
    \item Debug builds for development testing
    \item Release builds with optimizations for distribution
    \item APK signing for Google Play compatibility (future)
\end{itemize}

%-----------------------------------------------------------------------
\section{Summary}
\label{sec:methodology-summary}

This chapter described the comprehensive methodology employed to develop the HPC Sorting Serious Game:

\begin{itemize}
    \item \textbf{Research Approach}: Design Science Research framework with iterative development
    \item \textbf{Requirements}: Educational, functional, and non-functional requirements derived from learning objectives
    \item \textbf{Technology Selection}: Justified choices of Godot Engine, GDScript, GDSync, and WebRTC based on project needs
    \item \textbf{Game Design}: Systematic mapping of HPC concepts to game mechanics with UX design principles
    \item \textbf{Evaluation}: Multi-faceted evaluation approach combining technical, usability, and educational assessment
\end{itemize}

The next chapter presents the detailed system architecture resulting from these methodological decisions, including scene structure, component design, and multiplayer networking architecture.

\chapter{System Design and Architecture}
\label{ch:architecture}

This chapter describes the comprehensive design and architecture of the HPC Sorting Serious Game. We present the system from multiple perspectives: high-level scene structure, core component design, multiplayer networking architecture, responsive UI considerations, and detailed data flow patterns.

%-----------------------------------------------------------------------
\section{Architectural Overview}
\label{sec:architectural-overview}

The game follows a component-based architecture typical of modern game engines, with clear separation of concerns between presentation, game logic, and networking layers. The architecture is designed to support both single-player (sequential execution) and multiplayer (OpenMP shared-memory simulation) modes with significant code reuse.

%-----------------------------------------------------------------------
\subsection{Design Principles}
\label{subsec:design-principles}

The architecture adheres to several key design principles:

\begin{enumerate}
    \item \textbf{Component-Based Design}: Game entities (cards, containers, buffers) are modular components that can be composed and reused.

    \item \textbf{Separation of Concerns}: Clear boundaries exist between UI presentation, game logic, and networking code.

    \item \textbf{Signal-Driven Communication}: Components communicate through Godot's signal system rather than tight coupling, improving maintainability.

    \item \textbf{Host-Authoritative Multiplayer}: One client (the host) serves as the authoritative source of truth to prevent cheating and simplify conflict resolution.

    \item \textbf{Responsive UI}: All interfaces are designed with responsive layouts that adapt to different screen sizes and support both touch and mouse input.

    \item \textbf{State Synchronization}: Clear patterns for synchronizing game state across multiple clients using the GDSync framework.
\end{enumerate}

%-----------------------------------------------------------------------
\subsection{System Layers}
\label{subsec:system-layers}

The system is organized into distinct logical layers:

\begin{description}
    \item[Presentation Layer] Handles all visual rendering, animations, user input (touch/mouse events), and UI components (menus, dialogs, toasts).

    \item[Game Logic Layer] Implements core game mechanics including card sorting validation, buffer management, timer control, and game state transitions.

    \item[Networking Layer] Manages relay server connections via HTTP/SSE, state synchronization through GDSync, lobby system for player matchmaking, and host election.

    \item[Framework Layer] Provides reusable utilities including scene management, settings persistence, theme management, logging, and debug tools.
\end{description}

%-----------------------------------------------------------------------
\section{Scene Structure}
\label{sec:scene-structure}

The game is organized into several distinct scenes, each representing a major screen or functional area. Godot's scene-based architecture naturally supports this modular organization.

%-----------------------------------------------------------------------
\subsection{Main Menu Scene}
\label{subsec:main-menu-scene}

The entry point of the application, providing navigation to all major features.

\paragraph*{Responsibilities:}
\begin{itemize}
    \item Display game title and branding
    \item Provide navigation buttons to single-player and multiplayer modes
    \item Access to settings (theme, audio, controls)
    \item Exit game or close browser tab
\end{itemize}

\paragraph*{Key Components:}
\begin{lstlisting}[language=Python, caption={Main menu scene structure (simplified)}]
MainMenuScene (Control)
|-- BackgroundPanel (Panel)
|-- TitleLabel (Label) "HPC Sorting Game"
|-- ButtonContainer (VBoxContainer)
|   |-- SinglePlayerButton (Button)
|   |-- MultiplayerButton (Button)
|   |-- SettingsButton (Button)
|   +-- QuitButton (Button)
+-- VersionLabel (Label)
\end{lstlisting}

%-----------------------------------------------------------------------
\subsection{Single-Player Game Scene}
\label{subsec:singleplayer-scene}

Implements the OpenMP simulation gameplay where players sort cards using private buffers.

\paragraph*{Responsibilities:}
\begin{itemize}
    \item Display all cards in a shared scrollable container
    \item Provide private buffer zones for local sorting
    \item Track game timer and move count
    \item Validate sorting order and detect completion
    \item Show victory screen with performance metrics
\end{itemize}

\paragraph*{Architecture:}

\begin{lstlisting}[language=Python, caption={Single-player scene hierarchy}]
SinglePlayerScene (Control)
|-- CardManager (Node) [Script: card_manager.gd]
|   |-- Timer (Timer)
|   +-- GameState (Dictionary)
|-- UI (Control)
|   |-- TopBar (HBoxContainer)
|   |   |-- TimerLabel (Label)
|   |   |-- MoveCountLabel (Label)
|   |   +-- BackButton (Button)
|   |-- CardScrollContainer (ScrollContainer)
|   |   +-- CardFlowContainer (HFlowContainer)
|   |       +-- [Card instances added dynamically]
|   +-- BufferContainer (VBoxContainer)
|       |-- Buffer1 (Panel)
|       |-- Buffer2 (Panel)
|       +-- Buffer3 (Panel)
+-- FinishGameWindow (PopupPanel)
    |-- CongratsLabel (Label)
    |-- TimeLabel (Label)
    |-- MovesLabel (Label)
    |-- RestartButton (Button)
    +-- MenuButton (Button)
\end{lstlisting}

\paragraph*{Key Design Decisions:}
\begin{itemize}
    \item \textbf{ScrollContainer for Cards}: Enables viewing large numbers of cards (50--200) on screens of various sizes by providing vertical scrolling.

    \item \textbf{FlowContainer Layout}: Automatically wraps cards to multiple rows based on screen width, providing responsive layout.

    \item \textbf{Separate Buffer Zones}: Physically distinct areas on screen to clearly represent private storage separate from shared container.

    \item \textbf{Single CardManager Node}: Centralized logic for game state, validation, and event coordination.
\end{itemize}

%-----------------------------------------------------------------------
\subsection{Multiplayer Lobby Scene}
\label{subsec:lobby-scene}

Handles player matchmaking, room creation/joining, and connection management before gameplay.

\paragraph*{Responsibilities:}
\begin{itemize}
    \item Create or join game rooms via unique room codes
    \item Display list of connected players
    \item Handle host election and role assignment
    \item Configure game settings (number of cards, rules)
    \item Start game when all players are ready
    \item Manage connection status and timeouts
\end{itemize}

\paragraph*{Architecture:}

\begin{lstlisting}[language=Python, caption={Lobby scene structure}]
LobbyScene (Control)
|-- ConnectionManager (Node) [Script: connection_manager.gd]
|   |-- HTTPSSEConnection (Node)
|   |-- SignalingClient (WebSocketClient)
|   +-- PeerConnections (Dictionary)
|-- UI (Control)
|   |-- CreateRoomPanel (Panel)
|   |   |-- RoomCodeLabel (Label)
|   |   |-- CopyCodeButton (Button)
|   |   +-- StartGameButton (Button)
|   |-- JoinRoomPanel (Panel)
|   |   |-- RoomCodeInput (LineEdit)
|   |   +-- JoinButton (Button)
|   |-- PlayerListContainer (VBoxContainer)
|   |   +-- [PlayerCard instances]
|   +-- SettingsPanel (Panel)
|       |-- CardCountSlider (HSlider)
|       +-- ReadyCheckbox (CheckBox)
+-- ToastParty (Node) [Notifications]
\end{lstlisting}

\paragraph*{Connection Flow:}

\begin{enumerate}
    \item Host creates room → receives unique room code from signaling server
    \item Host shares room code with other players (via messaging, email, etc.)
    \item Other players enter room code and click Join
    \item Relay server facilitates peer discovery and connection via HTTP/SSE
    \item Direct peer-to-peer connections established (no server relay for game data)
    \item When all players ready, host initiates game scene transition
    \item GDSync framework ensures all clients load the multiplayer game scene synchronously
\end{enumerate}

%-----------------------------------------------------------------------
\subsection{Multiplayer Game Scene}
\label{subsec:multiplayer-scene}

Implements the OpenMP shared-memory simulation gameplay with collaborative sorting and private buffers.

\paragraph*{Responsibilities:}
\begin{itemize}
    \item Display shared container visible to all players (shared memory)
    \item Provide private buffer zones for each player (thread-local storage)
    \item Hide cards in other players' private buffers (thread-private data)
    \item Synchronize card movements across all clients
    \item Enable simultaneous access to shared container (parallel execution)
    \item Handle disconnections and reconnections gracefully
    \item Validate global sorting order
\end{itemize}

\paragraph*{Architecture:}

\begin{lstlisting}[language=Python, caption={Multiplayer scene hierarchy}]
MultiplayerScene (Control)
|-- MultiplayerCardManager (Node) [Script: multiplayer_card_manager.gd]
|   |-- Inherits: CardManager
|   |-- GDSyncNode (Node)
|   +-- NetworkState (Dictionary)
|       |-- cards_in_other_buffers: Dictionary
|       +-- pending_sync_operations: Array
|-- UI (Control)
|   |-- TopBar (HBoxContainer)
|   |   |-- PlayerIndicator (Label) "Player 2/4"
|   |   |-- TimerLabel (Label)
|   |   +-- ConnectionStatus (TextureRect)
|   |-- CardScrollContainer (ScrollContainer)
|   |   +-- [Cards visible to this player]
|   +-- BufferContainer (VBoxContainer)
|       +-- [Private buffers for this player]
+-- ConnectionManager (Node)
    +-- [Shared with lobby, manages peer connections]
\end{lstlisting}

\paragraph*{State Management:}

The multiplayer scene maintains several types of state:

\begin{description}
    \item[Local State] Information known only to this client (e.g., cards currently being dragged, UI state).

    \item[Private State] Information owned by this player but not visible to others (e.g., cards in this player's buffers).

    \item[Shared State] Information visible to all players (e.g., cards in the main container, timer, game completion status).

    \item[Authoritative State] The host's version of shared state, which is the source of truth for conflict resolution.
\end{description}

%-----------------------------------------------------------------------
\section{Core Components}
\label{sec:core-components}

%-----------------------------------------------------------------------
\subsection{Card Component}
\label{subsec:card-component}

The Card is the fundamental interactive element of the game, representing a single sortable item.

\paragraph*{Class Structure:}

\begin{lstlisting}[language=Python, caption={Card component class definition}]
class_name Card
extends Control

# Card properties
var card_value: int
var card_id: String  # Unique identifier (UUID)
var current_location: String  # "container", "buffer1", "buffer2", etc.
var owner_player_id: String  # For multiplayer

# Visual components
@onready var panel: Panel = $Panel
@onready var label: Label = $Panel/Label
@onready var animation_player: AnimationPlayer = $AnimationPlayer

# State
var is_being_dragged: bool = false
var original_position: Vector2
var original_slot_index: int

# Signals
signal card_picked_up(card: Card)
signal card_dropped(card: Card, target_container: Control)
signal card_clicked(card: Card)
\end{lstlisting}

\paragraph*{Key Responsibilities:}
\begin{itemize}
    \item Visual representation (panel background, number label, highlighting)
    \item Input handling (touch/mouse drag-and-drop)
    \item State tracking (location, ownership, drag state)
    \item Animations (pickup, drop, highlighting, shake for invalid moves)
    \item Signal emission for game logic integration
\end{itemize}

\paragraph*{Drag-and-Drop Implementation:}

\begin{lstlisting}[language=Python, caption={Card drag-and-drop logic}]
func _gui_input(event: InputEvent) -> void:
    if event is InputEventScreenTouch or event is InputEventMouseButton:
        if event.pressed:
            _on_pickup(event.position)
        else:
            _on_drop(event.position)
    elif event is InputEventScreenDrag or event is InputEventMouseMotion:
        if is_being_dragged:
            _on_drag(event.position)

func _on_pickup(position: Vector2) -> void:
    is_being_dragged = true
    original_position = global_position
    z_index = 100  # Bring to front
    scale = Vector2(1.1, 1.1)  # Visual feedback
    card_picked_up.emit(self)

func _on_drop(position: Vector2) -> void:
    is_being_dragged = false
    z_index = 0
    scale = Vector2(1.0, 1.0)
    
    var drop_target = _get_drop_target(position)
    if drop_target:
        card_dropped.emit(self, drop_target)
    else:
        # Return to original position
        _animate_return_to_original()
\end{lstlisting}

\paragraph*{Visual States:}
\begin{itemize}
    \item \textbf{Normal}: Default appearance
    \item \textbf{Highlighted}: When being dragged (scale 1.1x, slight shadow)
    \item \textbf{Disabled}: When in another player's buffer (semi-transparent, grayed out)
    \item \textbf{Invalid}: Shake animation when dropped in invalid location
    \item \textbf{Correct}: Green highlight when in correct sorted position
\end{itemize}

%-----------------------------------------------------------------------
\subsection{Card Manager}
\label{subsec:card-manager}

The CardManager is the central controller for game logic in single-player mode, responsible for card lifecycle, validation, and game state transitions.

\paragraph*{Class Structure:}

\begin{lstlisting}[language=Python, caption={CardManager class structure}]
class_name CardManager
extends Node

# Configuration
var number_of_cards: int = 50
var card_scene: PackedScene = preload("res://scenes/CardScene/Card.tscn")

# Game state
var cards: Array[Card] = []
var game_started: bool = false
var start_time: float = 0.0
var move_count: int = 0

# References to UI containers
@onready var card_container: HFlowContainer
@onready var buffer_zones: Array[Panel] = []

# Signals
signal game_started()
signal game_finished(time: float, moves: int)
signal card_placed_in_container(card: Card, was_in_buffer: bool, slot: int)
signal card_entered_buffer(card: Card, buffer_index: int)
signal card_left_buffer(card: Card, buffer_index: int)
\end{lstlisting}

\paragraph*{Key Methods:}

\begin{lstlisting}[language=Python, caption={CardManager core methods}]
func initialize_game() -> void:
    """Generate and shuffle cards, start timer."""
    cards.clear()
    var values = range(1, number_of_cards + 1)
    values.shuffle()
    
    for value in values:
        var card = card_scene.instantiate()
        card.card_value = value
        card.card_id = UUID.generate()
        card.card_dropped.connect(_on_card_dropped)
        cards.append(card)
        card_container.add_child(card)
    
    start_time = Time.get_ticks_msec()
    game_started = true
    game_started.emit()

func check_sorting_order() -> bool:
    """Validate if cards in container are sorted."""
    var container_cards = _get_cards_in_container()
    if container_cards.size() != cards.size():
        return false  # Some cards still in buffers
    
    for i in range(container_cards.size() - 1):
        if container_cards[i].card_value > container_cards[i+1].card_value:
            return false  # Not sorted
    
    return true

func _on_card_dropped(card: Card, target_container: Control) -> void:
    """Handle card placement in container or buffer."""
    move_count += 1
    
    var was_in_buffer = card.current_location.begins_with("buffer")
    
    if target_container == card_container:
        var slot = _calculate_drop_slot(card.global_position)
        _move_card_to_slot(card, slot)
        card.current_location = "container"
        
        if was_in_buffer:
            card_left_buffer.emit(card, _get_buffer_index(card))
        
        card_placed_in_container.emit(card, was_in_buffer, slot)
        
        if check_sorting_order():
            _finish_game()
    
    elif target_container in buffer_zones:
        var buffer_index = buffer_zones.find(target_container)
        _move_card_to_buffer(card, buffer_index)
        card.current_location = "buffer%d" % buffer_index
        card_entered_buffer.emit(card, buffer_index)

func _finish_game() -> void:
    """Handle game completion."""
    var elapsed_time = (Time.get_ticks_msec() - start_time) / 1000.0
    game_finished.emit(elapsed_time, move_count)
    _show_finish_window(elapsed_time, move_count)
\end{lstlisting}

\paragraph*{Responsibilities Summary:}
\begin{itemize}
    \item Card generation and shuffling
    \item Timer management
    \item Move counting
    \item Drag-and-drop event coordination
    \item Buffer zone management
    \item Sorting validation
    \item Game completion detection
    \item Signal emission for UI updates
\end{itemize}

%-----------------------------------------------------------------------
\subsection{Multiplayer Card Manager}
\label{subsec:multiplayer-card-manager}

The MultiplayerCardManager extends CardManager with synchronization logic for multiplayer gameplay.

\paragraph*{Architecture:}

\begin{lstlisting}[language=Python, caption={MultiplayerCardManager inheritance}]
class_name MultiplayerCardManager
extends CardManager

# Multiplayer-specific state
var is_host: bool = false
var my_player_id: String
var connected_players: Dictionary = {}  # player_id -> player_info
var cards_in_other_buffers: Dictionary = {}  # card_id -> player_id

# GDSync node reference
@onready var gdsync: Node

# Additional signals
signal sync_required(operation: String, data: Dictionary)
signal remote_card_moved(card_id: String, new_location: String)
\end{lstlisting}

\paragraph*{Host-Authoritative Model:}

The multiplayer architecture follows a host-authoritative pattern where the host maintains the canonical game state and clients request state changes:

\begin{enumerate}
    \item \textbf{Client Action}: Player drags a card and drops it
    \item \textbf{Local Update}: Client immediately updates local visuals for responsiveness
    \item \textbf{Broadcast to Host}: Client sends action to host via GDSync RPC
    \item \textbf{Host Validation}: Host validates the move (e.g., checks if card is available)
    \item \textbf{Host Broadcast}: If valid, host broadcasts confirmed state to all clients
    \item \textbf{Client Reconciliation}: Clients apply authoritative update, correcting any discrepancies
\end{enumerate}

\paragraph*{Synchronization Methods:}

\begin{lstlisting}[language=Python, caption={Key synchronization methods}]
@rpc("any_peer", "call_remote", "reliable")
func sync_card_moved(card_id: String, new_slot: int, player_id: String) -> void:
    """Synchronize card reordering in main container."""
    if not is_host:
        # Only host processes sync requests
        return
    
    var card = _get_card_by_id(card_id)
    if card:
        _move_card_to_slot(card, new_slot)
        # Broadcast to all clients
        rpc("apply_card_moved", card_id, new_slot)

@rpc("authority", "call_remote", "reliable")
func apply_card_moved(card_id: String, new_slot: int) -> void:
    """Apply card movement from authoritative host."""
    var card = _get_card_by_id(card_id)
    if card:
        _move_card_to_slot(card, new_slot)

@rpc("any_peer", "call_remote", "reliable")
func sync_card_entered_buffer(card_id: String, player_id: String, buffer_index: int) -> void:
    """Synchronize card moving to a player's private buffer."""
    if not is_host:
        return
    
    # Record which player has this card
    cards_in_other_buffers[card_id] = player_id
    
    # Broadcast to all clients to hide the card
    rpc("apply_card_entered_buffer", card_id, player_id, buffer_index)

@rpc("authority", "call_remote", "reliable")
func apply_card_entered_buffer(card_id: String, player_id: String, buffer_index: int) -> void:
    """Hide card that entered another player's buffer."""
    if player_id == my_player_id:
        return  # Owner still sees their own cards
    
    var card = _get_card_by_id(card_id)
    if card:
        card.visible = false
        card.set_process_input(false)  # Disable interaction

@rpc("any_peer", "call_remote", "reliable")
func sync_card_left_buffer(card_id: String, player_id: String) -> void:
    """Synchronize card leaving a player's buffer back to container."""
    if not is_host:
        return
    
    if card_id in cards_in_other_buffers:
        cards_in_other_buffers.erase(card_id)
    
    # Broadcast to make card visible again
    rpc("apply_card_left_buffer", card_id)

@rpc("authority", "call_remote", "reliable")
func apply_card_left_buffer(card_id: String) -> void:
    """Show card that returned from a buffer."""
    var card = _get_card_by_id(card_id)
    if card:
        card.visible = true
        card.set_process_input(true)

func sync_complete_game_state(target_peer_id: int) -> void:
    """Send full game state to newly connected player."""
    if not is_host:
        return
    
    var state = {
        "cards": _serialize_card_state(),
        "timer": get_elapsed_time(),
        "moves": move_count,
        "buffers": cards_in_other_buffers.duplicate()
    }
    
    rpc_id(target_peer_id, "apply_complete_game_state", state)
\end{lstlisting}

\paragraph*{Visibility Management:}

A critical aspect of the OpenMP simulation is maintaining correct visibility of cards to represent thread-private data:

\begin{itemize}
    \item \textbf{Cards in shared container}: Visible to all players (shared memory)
    \item \textbf{Cards in my private buffers}: Visible only to me (my thread-local storage)
    \item \textbf{Cards in other players' private buffers}: Hidden from view (other threads' private data)
    \item \textbf{Cards being moved}: Brief transition animation showing memory access
\end{itemize}

The \texttt{cards\_in\_other\_buffers} dictionary tracks which cards are currently private:

\begin{lstlisting}[language=Python, caption={Visibility tracking}]
var cards_in_other_buffers: Dictionary = {}
# Example state:
# {
#   "card_uuid_123": "player_2",
#   "card_uuid_456": "player_2",
#   "card_uuid_789": "player_3"
# }
\end{lstlisting}

%-----------------------------------------------------------------------
\subsection{Scroll Container and Layout}
\label{subsec:scroll-container}

Managing 50--200 cards on screen requires efficient layout and scrolling.

\paragraph*{Architecture:}

\begin{lstlisting}[language=Python, caption={Scrollable card container structure}]
CardScrollContainer (ScrollContainer)
|-- scroll_vertical_enabled = true
|-- scroll_horizontal_enabled = false
+-- CardFlowContainer (HFlowContainer)
    |-- alignment = ALIGNMENT_BEGIN
    |-- separation = 10
    +-- [Card instances]
\end{lstlisting}

\paragraph*{Layout Strategy:}

\begin{itemize}
    \item \textbf{HFlowContainer}: Automatically wraps cards to multiple rows based on available width
    \item \textbf{Responsive Card Size}: Cards scale based on screen width to fit approximately 4--6 per row
    \item \textbf{Vertical Scrolling Only}: Horizontal scrolling disabled for simpler interaction
    \item \textbf{Input-Optimized}: Large click/touch targets, smooth scroll physics
\end{itemize}

\paragraph*{Drop Target Detection:}

\begin{lstlisting}[language=Python, caption={Drop slot calculation}]
func _calculate_drop_slot(drop_position: Vector2) -> int:
    """Determine which slot a card should be dropped into."""
    var container_cards = card_container.get_children()
    
    # Convert global position to local container position
    var local_pos = card_container.to_local(drop_position)
    
    # Iterate through existing cards to find insertion point
    for i in range(container_cards.size()):
        var card = container_cards[i]
        var card_center = card.position + card.size / 2
        
        if local_pos.x < card_center.x:
            return i
    
    # Drop after all cards
    return container_cards.size()
\end{lstlisting}

%-----------------------------------------------------------------------
\section{Multiplayer Architecture}
\label{sec:multiplayer-architecture}

%-----------------------------------------------------------------------
\subsection{Network Topology}
\label{subsec:network-topology}

The game uses a \textbf{peer-to-peer (P2P) mesh network} topology with \textbf{host-authoritative state management}.

\paragraph*{Topology Diagram:}

\begin{verbatim}
    Player 1 (Host)
        |  \  \
        |   \  \___
        |    \      \
    Player 2  Player 3  Player 4
        |   ___/     /
        |  /    ____/
        | /    /
       \|/   |/
    Direct P2P Connections
\end{verbatim}

\paragraph*{Connection Characteristics:}
\begin{itemize}
    \item Each player connects through the relay server to communicate with other players
    \item No dedicated server required for gameplay (signaling server only used for initial connection)
    \item Low latency for small groups (2--4 players)
    \item Host election occurs if original host disconnects
\end{itemize}

%-----------------------------------------------------------------------
\subsection{State Synchronization Patterns}
\label{subsec:state-sync-patterns}

\paragraph*{Pattern 1: Optimistic Update with Confirmation}

Used for non-critical actions where responsiveness is important:

\begin{enumerate}
    \item Client performs action locally (e.g., moving card within own buffer)
    \item Client updates local visuals immediately
    \item Client sends action to host
    \item Host validates and broadcasts
    \item If host rejects, client receives correction and reverts
\end{enumerate}

\paragraph*{Pattern 2: Host Broadcast}

Used for host-initiated events:

\begin{enumerate}
    \item Host detects event (e.g., game timer expires, sorting complete)
    \item Host broadcasts event to all clients
    \item Clients apply event simultaneously
\end{enumerate}

%-----------------------------------------------------------------------
\subsection{GDSync Integration}
\label{subsec:gdsync-integration}

GDSync is a third-party multiplayer framework for Godot that provides high-level abstractions for state synchronization. For web browser compatibility, \textbf{a custom HTTP/SSE transport layer was implemented} to replace GDSync's native networking. The game primarily interacts with GDSync's RPC API, which then uses the custom transport layer for browser-based communication.

\paragraph*{Key Features Used:}
\begin{itemize}
    \item \textbf{RPC (Remote Procedure Calls)}: Call functions on remote peers
    \item \textbf{Node Replication}: Automatically sync node properties
    \item \textbf{Ownership System}: Track which peer owns which nodes
    \item \textbf{Authority Modes}: Specify who can call functions ("any\_peer", "authority")
    \item \textbf{Transport Abstraction}: Custom HTTP/SSE transport layer provides browser-compatible networking
\end{itemize}

\begin{figure}[htbp]
    \centering
    \includegraphics[width=0.85\textwidth]{figures/diagrams/http-sse-architecture.png}
    \caption{HTTP/SSE relay architecture: star topology with centralized relay server}
    \label{fig:http-sse-architecture}
\end{figure}

\paragraph*{Configuration:}

\begin{lstlisting}[language=Python, caption={GDSync setup in MultiplayerCardManager}]
func _ready():
    super._ready()  # Call CardManager._ready()
    
    # Initialize GDSync
    gdsync = $GDSyncNode
    gdsync.set_multiplayer_authority(1)  # Host = peer ID 1
    
    # Register synchronized functions
    gdsync.register_rpc("sync_card_moved", 
                        Callable(self, "sync_card_moved"))
    gdsync.register_rpc("sync_card_entered_buffer", 
                        Callable(self, "sync_card_entered_buffer"))
    gdsync.register_rpc("sync_card_left_buffer", 
                        Callable(self, "sync_card_left_buffer"))
    
    # Connect to networking events
    multiplayer.peer_connected.connect(_on_peer_connected)
    multiplayer.peer_disconnected.connect(_on_peer_disconnected)
\end{lstlisting}

\paragraph*{RPC Call Patterns:}

\begin{lstlisting}[language=Python, caption={Using GDSync RPCs}]
# Pattern 1: Broadcast to all peers
func broadcast_card_moved(card_id: String, slot: int):
    rpc("sync_card_moved", card_id, slot, my_player_id)

# Pattern 2: Call specific peer
func send_to_host(data: Dictionary):
    rpc_id(1, "process_client_action", data)

# Pattern 3: Host broadcasts authoritative update
func host_broadcast_state():
    if is_host:
        rpc("apply_card_moved", card_id, new_slot)
\end{lstlisting}

%-----------------------------------------------------------------------
\subsection{Connection Management}
\label{subsec:connection-management}

\paragraph*{Lobby Phase:}
\begin{enumerate}
    \item Host creates room → signaling server assigns unique code
    \item Host shares code with players
    \item Players connect via relay server
    \item Each player establishes direct P2P connection with all others
    \item Host waits for all players ready
    \item Host triggers synchronized scene transition
\end{enumerate}

\paragraph*{Gameplay Phase:}
\begin{enumerate}
    \item Host initializes game state (shuffle cards, start timer)
    \item Host broadcasts initial state to all clients
    \item Clients apply initial state and begin gameplay
    \item Players send actions via RPC
    \item Host validates and broadcasts confirmations
    \item Game progresses until completion or disconnection
\end{enumerate}

\paragraph*{Disconnection Handling:}

\begin{lstlisting}[language=Python, caption={Handling player disconnections}]
func _on_peer_disconnected(peer_id: int):
    var player_name = connected_players[peer_id].name
    ToastParty.show_toast("%s disconnected" % player_name)
    
    connected_players.erase(peer_id)
    
    # If host disconnected, elect new host
    if peer_id == 1 and not is_host:
        _elect_new_host()
    
    # Release cards owned by disconnected player
    for card_id in cards_in_other_buffers.keys():
        if cards_in_other_buffers[card_id] == peer_id:
            cards_in_other_buffers.erase(card_id)
            _return_card_to_container(card_id)

func _elect_new_host():
    """Simple host election: lowest remaining peer ID becomes host."""
    var peer_ids = connected_players.keys()
    peer_ids.sort()
    var new_host_id = peer_ids[0]
    
    if multiplayer.get_unique_id() == new_host_id:
        is_host = true
        ToastParty.show_toast("You are now the host")
        gdsync.set_multiplayer_authority(new_host_id)
\end{lstlisting}

%-----------------------------------------------------------------------
\section{Data Flow Diagrams}
\label{sec:data-flow}

%-----------------------------------------------------------------------
\subsection{Single-Player Game Flow}
\label{subsec:singleplayer-flow}

\begin{verbatim}
User Input (Touch)
    |
    v
Card._gui_input()
    |
    v
Card.card_dropped.emit(card, target_container)
    |
    v
ScrollContainer._on_card_dropped() [signal forwarding]
    |
    v
CardManager._on_card_placed_in_container(card, was_in_buffer, slot)
    |
    +---> Update card position in container
    +---> Increment move counter
    +---> If was_in_buffer: emit card_left_buffer signal
    +---> emit card_placed_in_container signal
    |
    v
CardManager.check_sorting_order()
    |
    +---> If not sorted: continue gameplay
    |
    +---> If sorted:
          |
          v
          CardManager._finish_game()
              |
              +---> Stop timer
              +---> Calculate elapsed time
              +---> emit game_finished signal
              +---> Show finish window with confetti
\end{verbatim}

%-----------------------------------------------------------------------
\subsection{Multiplayer Card Movement Flow}
\label{subsec:multiplayer-flow}

\paragraph*{Scenario: Player 2 moves a card in the shared container}

\begin{verbatim}
Player 2 Client:
    User drags card to new position
        |
        v
    Card.card_dropped.emit()
        |
        v
    MultiplayerCardManager._on_card_placed_in_container()
        |
        +---> Update local visuals (immediate feedback)
        +---> Calculate new slot index
        |
        v
    RPC: sync_card_moved(card_id, new_slot, "player_2")
        |
        | [Network transmission]
        v

Host (Player 1):
    Receives sync_card_moved RPC
        |
        v
    Validate move (check card availability)
        |
        +---> If invalid: send rejection (not implemented yet)
        |
        +---> If valid:
              |
              v
              Apply move to host's game state
              |
              v
              RPC broadcast: apply_card_moved(card_id, new_slot)
                  |
                  | [Broadcast to all clients]
                  v

All Clients (including Player 2):
    Receive apply_card_moved RPC
        |
        v
    Apply authoritative card position
        |
        v
    Update UI: _move_card_to_slot(card, new_slot)
        |
        v
    Animation: smooth transition to new position
\end{verbatim}

%-----------------------------------------------------------------------
\subsection{Buffer Visibility Synchronization}
\label{subsec:buffer-visibility-flow}

\paragraph*{Scenario: Player 3 takes a card into their private buffer}

\begin{verbatim}
Player 3 Client:
    User drags card from container to buffer zone
        |
        v
    Card.card_dropped.emit(card, buffer_panel)
        |
        v
    MultiplayerCardManager._on_card_dropped()
        |
        +---> Determine target is buffer
        +---> Move card to buffer visually
        +---> Update card.current_location = "buffer1"
        |
        v
    RPC: sync_card_entered_buffer(card_id, "player_3", buffer_index)
        |
        | [Network transmission]
        v

Host (Player 1):
    Receives sync_card_entered_buffer RPC
        |
        v
    Record: cards_in_other_buffers[card_id] = "player_3"
        |
        v
    RPC broadcast: apply_card_entered_buffer(card_id, "player_3", buffer_index)
        |
        | [Broadcast to all clients]
        v

Player 1 & Player 2 Clients (others):
    Receive apply_card_entered_buffer RPC
        |
        v
    Check: Is this my card? (player_id == my_player_id)
        |
        +---> Yes (should not happen): do nothing
        |
        +---> No (card belongs to Player 3):
              |
              v
              Hide card: card.visible = false
              Disable interaction: card.set_process_input(false)

Player 3 Client (owner):
    Receive apply_card_entered_buffer RPC
        |
        v
    Check: Is this my card? YES
        |
        v
    Do nothing (card remains visible and interactive)
\end{verbatim}

\paragraph*{Result:}
\begin{itemize}
    \item Player 3 sees the card in their buffer and can interact with it
    \item Players 1, 2, and 4 do not see the card (simulating distributed memory)
    \item Card remains in game state but is hidden from other players' views
    \item When Player 3 returns the card to the container, \texttt{sync\_card\_left\_buffer} makes it visible again to everyone
\end{itemize}

%-----------------------------------------------------------------------
\section{Responsive UI/UX Design}
\label{sec:responsive-ui-ux}

%-----------------------------------------------------------------------
\subsection{Design Constraints}
\label{subsec:design-constraints}

Web platforms and varying screen sizes impose several constraints that significantly influenced design decisions:

\begin{itemize}
    \item \textbf{Screen Size}: 5--7 inch displays, approximately 1080x2400 pixels
    \item \textbf{Touch Input}: Fingers occlude content, require larger targets (minimum 44x44 dp)
    \item \textbf{Performance}: Variable GPU/CPU capabilities, battery constraints
    \item \textbf{Network}: Variable bandwidth, potential disconnections
    \item \textbf{Orientation}: Support both portrait and landscape
\end{itemize}

%-----------------------------------------------------------------------
\subsection{Layout Strategies}
\label{subsec:layout-strategies}

\paragraph*{Card Display:}
\begin{itemize}
    \item Cards sized dynamically: \texttt{min(screen\_width / 6, 120px)}
    \item Approximately 4--6 cards per row on typical phones
    \item 10--20 rows visible depending on card count and screen size
    \item Vertical scrolling for navigation
\end{itemize}

\paragraph*{Buffer Zones:}
\begin{itemize}
    \item Fixed height panels at bottom of screen
    \item 2--3 buffers per player (configurable)
    \item Each buffer displays up to 10--15 cards in a horizontal flow
    \item Collapse/expand toggles to save screen space
\end{itemize}

\paragraph*{Top Bar:}
\begin{itemize}
    \item Fixed position header with timer, move count, player indicator
    \item Always visible during gameplay
    \item Minimal height to maximize card display area
\end{itemize}

%-----------------------------------------------------------------------
\subsection{Touch Interaction Patterns}
\label{subsec:touch-interactions}

\paragraph*{Drag-and-Drop:}
\begin{itemize}
    \item Long-press to pick up card (500ms threshold prevents accidental drags)
    \item Visual feedback: card scales to 1.1x, drops shadow, follows finger
    \item Drag anywhere on screen
    \item Drop zones highlighted when card is near
    \item Snap-to-slot animation when dropped
    \item Shake animation if dropped in invalid location
\end{itemize}

\paragraph*{Scrolling vs. Dragging:}
\begin{itemize}
    \item ScrollContainer consumes vertical swipes
    \item Card drag consumes touch once long-press threshold reached
    \item Clear visual distinction: cards grow when draggable
\end{itemize}

%-----------------------------------------------------------------------
\subsection{Visual Feedback Systems}
\label{subsec:visual-feedback}

\paragraph*{Toast Notifications:}
Brief non-intrusive messages for events:
\begin{itemize}
    \item ``Card entered buffer''
    \item ``Player 3 joined''
    \item ``Player disconnected''
    \item ``Game finished!''
\end{itemize}

\paragraph*{Animations:}
\begin{itemize}
    \item \textbf{Card Pickup}: Scale up, add shadow (200ms)
    \item \textbf{Card Drop}: Scale down, fade shadow (200ms)
    \item \textbf{Confetti}: Particle effect on game completion (3000ms)
\end{itemize}

\paragraph*{Color Coding:}
\begin{itemize}
    \item \textbf{Normal Cards}: Light blue background, black text
    \item \textbf{Highlighted Cards}: Yellow background (when dragging)
    \item \textbf{Hidden Cards}: 30\% opacity, gray (in other players' buffers)
    \item \textbf{Buffers}: Slightly darker background to distinguish from main container
\end{itemize}

%-----------------------------------------------------------------------
\section{Performance Considerations}
\label{sec:performance}

%-----------------------------------------------------------------------
\subsection{Optimization Strategies}
\label{subsec:optimizations}

\paragraph*{Rendering:}
\begin{itemize}
    \item Cards use simple Panel nodes (not full Control trees) to minimize draw calls
    \item Visibility culling: cards outside scroll view not rendered
    \item Animations use Tween for GPU-accelerated interpolation
    \item Limit particle effects (confetti) to 100--200 particles
\end{itemize}

\paragraph*{Networking:}
\begin{itemize}
    \item Aggregate multiple card movements before sending (debounce 100ms)
    \item Use ``reliable'' RPC only for critical state changes
    \item Use ``unreliable'' for non-critical updates (position interpolation)
    \item Compress state synchronization messages
\end{itemize}

\paragraph*{Memory:}
\begin{itemize}
    \item Reuse card instances instead of destroying/recreating
    \item Preload frequently used resources (card scene, textures)
    \item Clear large data structures when switching scenes
    \item Limit maximum cards to 200 to prevent memory issues
\end{itemize}

%-----------------------------------------------------------------------
\subsection{Scalability Limits}
\label{subsec:scalability-limits}

Based on testing and design constraints:

\begin{itemize}
    \item \textbf{Maximum Cards}: 200 (UI becomes difficult with more)
    \item \textbf{Maximum Players}: 4 (P2P mesh becomes inefficient beyond this)
    \item \textbf{Target Frame Rate}: 60 FPS in modern web browsers
    \item \textbf{Network Latency Tolerance}: Acceptable gameplay up to 200ms RTT
    \item \textbf{Minimum Screen Size}: 5-inch display (smaller makes cards too small)
\end{itemize}

%-----------------------------------------------------------------------
\section{Summary}
\label{sec:architecture-summary}

This chapter presented the comprehensive architecture of the HPC Sorting Serious Game, covering:

\begin{itemize}
    \item High-level system structure and design principles
    \item Scene organization and responsibilities
    \item Core component designs (Card, CardManager, MultiplayerCardManager)
    \item Multiplayer networking architecture with host-authoritative model
    \item GDSync framework integration for state synchronization
    \item Detailed data flow diagrams for single-player and multiplayer scenarios
    \item Responsive UI/UX design strategies and constraints
    \item Performance optimization considerations
\end{itemize}

The architecture balances educational effectiveness with technical feasibility, prioritizing clarity of HPC concept representation while maintaining acceptable performance across web browsers. The modular, component-based design allows for future extensions and modifications, fulfilling the extensibility objective outlined in Chapter~\ref{ch:introduction}.

The next chapter will detail the implementation of this architecture, including development tools, code organization, and specific technical challenges encountered during realization of this design.

\chapter{Implementation}
\label{ch:implementation}

This chapter describes the detailed technical implementation of the HPC Sorting Serious Game, covering both single-player and multiplayer modes, key algorithms, code structure, and mobile-specific optimizations.

%----------------------------------------------------------------------------------------
\section{Development Environment Setup}
\label{sec:dev-environment-setup}

%----------------------------------------------------------------------------------------
\subsection{Project Structure}
\label{subsec:project-structure}

The project follows Godot's standard directory structure with additional organization for plugins and resources:

\begin{lstlisting}[caption={Project directory structure}, language=bash]
hpc-sorting-serious-game/
|-- scenes/              # Game scenes
|   |-- main_menu.tscn
|   |-- lobby.tscn
|   |-- singleplayer.tscn
|   |-- multiplayer.tscn
|-- scripts/             # GDScript source files
|   |-- card.gd
|   |-- card_manager.gd
|   |-- buffer.gd
|   |-- lobby_manager.gd
|-- resources/           # Assets and resources
|   |-- textures/
|   |-- fonts/
|   |-- audio/
|-- addons/              # Third-party plugins
|   |-- GD-Sync/
|   |-- ToastParty/
|   |-- Logger/
|-- project.godot        # Godot project configuration
\end{lstlisting}

%----------------------------------------------------------------------------------------
\section{Core Components Implementation}
\label{sec:core-components}

%----------------------------------------------------------------------------------------
\subsection{Card Component}
\label{subsec:card-component}

The Card component represents an individual playing card and handles rendering, interaction, and state management.

\paragraph{Key Responsibilities:}
\begin{itemize}
    \item Display card number and visual state
    \item Handle drag-and-drop interactions
    \item Manage selection and highlighting
    \item Emit signals for game logic coordination
\end{itemize}

\paragraph{Implementation Details:}

\begin{lstlisting}[caption={Card component structure (card.gd)}, label={lst:card-structure}]
extends Control
class_name Card

signal card_selected(card: Card)
signal card_dragged(card: Card, position: Vector2)
signal card_dropped(card: Card, target)

var card_value: int
var original_position: Vector2
var is_dragging: bool = false
var is_selected: bool = false

func _ready():
    # Initialize card visuals
    update_display()

func update_display():
    # Update label with card value
    $Label.text = str(card_value)
    # Update visual state based on flags
    if is_selected:
        modulate = Color.YELLOW
    else:
        modulate = Color.WHITE
\end{lstlisting}

%----------------------------------------------------------------------------------------
\subsection{Card Manager}
\label{subsec:card-manager}

The Card Manager is responsible for creating, shuffling, and managing the collection of cards.

\paragraph{Key Responsibilities:}
\begin{itemize}
    \item Generate card instances
    \item Shuffle and distribute cards
    \item Validate sorting correctness
    \item Manage card ownership in multiplayer
\end{itemize}

%----------------------------------------------------------------------------------------
\subsection{Buffer System}
\label{subsec:buffer-system}

Buffers represent thread-local storage (OpenMP) or process-local memory (MPI).

\paragraph{Implementation:}
\begin{itemize}
    \item Each player has 2-3 buffer zones
    \item Cards can be dragged into buffers
    \item Buffers maintain sorted or unsorted card lists
    \item Visibility controlled based on game mode
\end{itemize}

%----------------------------------------------------------------------------------------
\section{Single-Player Mode Implementation}
\label{sec:singleplayer-implementation}

%----------------------------------------------------------------------------------------
\subsection{Game Initialization}
\label{subsec:game-initialization}

Steps for initializing a single-player game:
\begin{enumerate}
    \item Load singleplayer scene
    \item Generate specified number of cards
    \item Shuffle cards randomly
    \item Place cards in main container
    \item Initialize buffers
    \item Start timer
\end{enumerate}

%----------------------------------------------------------------------------------------
\subsection{Drag-and-Drop System}
\label{subsec:drag-drop-system}

The drag-and-drop system is crucial for mobile touch interaction.

\paragraph{Touch Event Handling:}
\begin{itemize}
    \item \texttt{\_input(event)}: Detects touch/mouse events
    \item \texttt{\_process(delta)}: Updates card position during drag
    \item Drop zones detect overlap with dragged cards
\end{itemize}

%----------------------------------------------------------------------------------------
\subsection{Sorting Validation}
\label{subsec:sorting-validation}

Algorithm to check if cards are sorted:

\begin{lstlisting}[caption={Sorting validation algorithm}]
func is_sorted(cards: Array) -> bool:
    if cards.size() < 2:
        return true

    for i in range(cards.size() - 1):
        if cards[i].card_value > cards[i + 1].card_value:
            return false

    return true
\end{lstlisting}

%----------------------------------------------------------------------------------------
\section{Multiplayer Mode Implementation}
\label{sec:multiplayer-implementation}

%----------------------------------------------------------------------------------------
\subsection{Lobby System}
\label{subsec:lobby-system}

The lobby manages room creation, player joining, and game configuration.

\paragraph{Lobby Flow:}
\begin{enumerate}
    \item Player creates room or enters room code
    \item Connection to signaling server
    \item WebRTC handshake with host
    \item Players appear in lobby list
    \item Host configures game settings
    \item Host starts game
\end{enumerate}

%----------------------------------------------------------------------------------------
\subsection{GDSync Integration}
\label{subsec:gdsync-integration}

GDSync provides high-level networking abstractions.

\paragraph{Key Features Used:}
\begin{itemize}
    \item \texttt{@GDSync.sync}: Mark nodes for automatic synchronization
    \item \texttt{@GDSync.rpc}: Remote procedure calls
    \item Connection management
    \item Lobby synchronization
\end{itemize}

\paragraph{Configuration:}

\begin{lstlisting}[caption={GDSync node configuration}]
# Mark card container for synchronization
@GDSync.sync(mode=GDSync.SYNC_MODE.AUTHORITY)
var card_container: Node

# Remote procedure call for card movement
@GDSync.rpc(call_local=true)
func move_card_networked(card_id: int, target_container_id: int):
    # Move card on all clients
    var card = get_card_by_id(card_id)
    var target = get_container_by_id(target_container_id)
    target.add_child(card)
\end{lstlisting}

%----------------------------------------------------------------------------------------
\subsection{State Synchronization}
\label{subsec:state-synchronization}

Maintaining consistent game state across clients is critical.

\paragraph{Synchronization Strategy:}
\begin{itemize}
    \item \textbf{Host-Authoritative}: Host has authority over game state
    \item \textbf{Client Prediction}: Clients show immediate local updates
    \item \textbf{Server Reconciliation}: Host sends authoritative updates
    \item \textbf{Delta Updates}: Only changed data transmitted
\end{itemize}

\paragraph{Synchronized Data:}
\begin{itemize}
    \item Card positions and container ownership
    \item Player states (connected, ready, playing)
    \item Timer and move counter
    \item Game phase (lobby, playing, finished)
\end{itemize}

%----------------------------------------------------------------------------------------
\subsection{Visibility Management}
\label{subsec:visibility-management}

In MPI mode, players should not see cards in other players' private buffers.

\paragraph{Implementation:}
\begin{lstlisting}[caption={Card visibility control}]
func update_card_visibility():
    for card in all_cards:
        if card.is_in_private_buffer():
            var buffer_owner_id = card.get_buffer_owner()
            if buffer_owner_id == local_player_id:
                card.visible = true
            else:
                card.visible = false  # Hide other players' cards
        else:
            card.visible = true  # Shared container cards visible
\end{lstlisting}

%----------------------------------------------------------------------------------------
\section{Mobile Optimizations}
\label{sec:mobile-optimizations}

%----------------------------------------------------------------------------------------
\subsection{Performance Optimizations}
\label{subsec:performance-optimizations}

\paragraph{Rendering Optimizations:}
\begin{itemize}
    \item Object pooling for cards (reuse instances)
    \item Viewport culling (don't render off-screen cards)
    \item Batch rendering where possible
    \item Minimize draw calls
\end{itemize}

\paragraph{Memory Management:}
\begin{itemize}
    \item Preload commonly used resources
    \item Release unused assets
    \item Texture compression for mobile
    \item Minimize garbage collection pauses
\end{itemize}

%----------------------------------------------------------------------------------------
\subsection{Touch Input Optimization}
\label{subsec:touch-optimization}

\paragraph{Touch Target Sizing:}
\begin{itemize}
    \item Minimum 48x48 dp for interactive elements
    \item Adequate spacing between cards
    \item Visual feedback for touch events
\end{itemize}

%----------------------------------------------------------------------------------------
\subsection{Network Optimization}
\label{subsec:network-optimization}

\paragraph{Data Minimization:}
\begin{itemize}
    \item Send only changed data
    \item Compress network messages
    \item Batch multiple updates when possible
    \item Predictive client-side updates
\end{itemize}

%----------------------------------------------------------------------------------------
\section{Debugging and Development Tools}
\label{sec:debugging-tools}

%----------------------------------------------------------------------------------------
\subsection{Logger Plugin Integration}
\label{subsec:logger-integration}

The Logger plugin provides categorized logging for debugging.

\paragraph{Usage Example:}
\begin{lstlisting}
Logger.info("Card %d moved to buffer %s" % [card.card_value, buffer.name])
Logger.warn("Network latency high: %d ms" % latency)
Logger.error("Failed to synchronize card state")
\end{lstlisting}

%----------------------------------------------------------------------------------------
\subsection{Runtime Variable Inspection}
\label{subsec:vartree}

VarTree plugin allows real-time inspection and modification of variables during gameplay for debugging purposes.

%----------------------------------------------------------------------------------------
\section{Code Quality and Best Practices}
\label{sec:code-quality}

%----------------------------------------------------------------------------------------
\subsection{Code Organization}
\label{subsec:code-organization}

\paragraph{Principles Followed:}
\begin{itemize}
    \item Single Responsibility Principle
    \item Separation of concerns (UI, logic, networking)
    \item DRY (Don't Repeat Yourself)
    \item Clear naming conventions
\end{itemize}

%----------------------------------------------------------------------------------------
\subsection{Documentation}
\label{subsec:code-documentation}

All scripts include:
\begin{itemize}
    \item File-level documentation describing purpose
    \item Function documentation with parameters and return values
    \item Inline comments for complex logic
    \item Signal documentation
\end{itemize}

%----------------------------------------------------------------------------------------
\section{Summary}
\label{sec:implementation-summary}

This chapter presented the detailed implementation of the HPC Sorting Serious Game, covering:

\begin{itemize}
    \item \textbf{Project Structure}: Organization of scenes, scripts, and resources
    \item \textbf{Core Components}: Card, Card Manager, and Buffer implementations
    \item \textbf{Single-Player Mode}: Initialization, drag-and-drop, and sorting validation
    \item \textbf{Multiplayer Mode}: Lobby system, GDSync integration, state synchronization, and visibility management
    \item \textbf{Mobile Optimizations}: Performance, touch input, and network optimizations
    \item \textbf{Development Tools}: Debugging and logging utilities
    \item \textbf{Code Quality}: Organization and documentation practices
\end{itemize}

The next chapter discusses the problems and challenges encountered during development and the solutions implemented to address them.

\chapter{Problems and Challenges}
\label{ch:problems}

This chapter provides an honest and comprehensive analysis of the difficulties encountered during the development of the HPC Sorting Serious Game. Understanding these challenges and their solutions is valuable for future developers of educational multiplayer games and contributes to the body of knowledge in serious game development.

%----------------------------------------------------------------------------------------
\section{Overview}
\label{sec:problems-overview}

Development of a serious game involving real-time multiplayer synchronization on mobile devices presented numerous technical, design, and pedagogical challenges. These problems span multiple domains:

\begin{itemize}
    \item Technology selection and framework limitations
    \item Multiplayer state synchronization complexity
    \item Mobile UI/UX constraints
    \item Performance optimization requirements
    \item Documentation and community support gaps
    \item Testing and debugging difficulties
\end{itemize}

This chapter documents these challenges, the approaches taken to address them, and lessons learned that may benefit future projects.

%----------------------------------------------------------------------------------------
\section{Technology Selection Challenges}
\label{sec:technology-challenges}

%----------------------------------------------------------------------------------------
\subsection{Game Engine Trade-offs}
\label{subsec:engine-tradeoffs}

\paragraph{Challenge:}

While Godot Engine proved to be a good choice overall, several limitations became apparent during development:

\begin{itemize}
    \item \textbf{Smaller Ecosystem}: Fewer third-party plugins and assets compared to Unity, requiring custom solutions for some features
    \item \textbf{Documentation Gaps}: Godot 4.x was relatively new, with incomplete documentation for advanced networking scenarios
    \item \textbf{Mobile Export Issues}: Occasional problems with Android export templates and permission configurations
    \item \textbf{Plugin Compatibility}: Some Godot 3.x plugins not yet updated for 4.x
\end{itemize}

\paragraph{Impact:}

Development time increased due to the need to:
\begin{itemize}
    \item Build custom solutions rather than using existing plugins
    \item Troubleshoot export issues through trial and error
    \item Consult community forums for undocumented features
\end{itemize}

\paragraph{Mitigation:}

\begin{itemize}
    \item Active participation in Godot community (Discord, forums, GitHub)
    \item Contribution to documentation through issue reports and examples
    \item Development of reusable utility scripts for common tasks
    \item Careful version management and testing before engine updates
\end{itemize}

%----------------------------------------------------------------------------------------
\subsection{GDScript Performance Concerns}
\label{subsec:gdscript-performance}

\paragraph{Challenge:}

GDScript's interpreted nature raised concerns about performance with 50--200 cards requiring frequent position updates and collision detection.

\paragraph{Solution:}

Profiling revealed that:
\begin{itemize}
    \item GDScript performance was acceptable for game logic
    \item Bottlenecks were in rendering and physics, handled by Godot's C++ core
    \item Network latency dominated perceived lag, not scripting performance
\end{itemize}

Optimization strategies:
\begin{itemize}
    \item Minimize expensive operations in \texttt{\_process()} and \texttt{\_physics\_process()}
    \item Use object pooling to reduce instantiation overhead
    \item Batch updates where possible
    \item Profile early and often to identify actual bottlenecks
\end{itemize}

%----------------------------------------------------------------------------------------
\section{GDSync Framework Challenges}
\label{sec:gdsync-challenges}

The most significant technical challenges involved the GDSync framework, which, despite its high-level abstractions, presented several issues.

%----------------------------------------------------------------------------------------
\subsection{Protected Mode Blocking Issue}
\label{subsec:protected-mode-issue}

\paragraph{Challenge:}

GDSync's "protected mode" (default) prevented RPC calls from being executed between non-host peers. When Player A tried to call an RPC on Player B's node, the call would be silently dropped or blocked.

\paragraph{Error Manifestation:}

\begin{itemize}
    \item Card movement worked from host to clients but not between clients
    \item No error messages or warnings—calls simply failed silently
    \item Debugging difficult due to lack of visibility into framework internals
\end{itemize}

\paragraph{Investigation Process:}

\begin{enumerate}
    \item Verified RPC syntax was correct according to documentation
    \item Added extensive logging to track RPC call flow
    \item Examined GDSync source code to understand permission system
    \item Tested with GDSync's example projects to compare behavior
    \item Posted issue on GDSync GitHub repository
\end{enumerate}

\paragraph{Solution:}

\begin{itemize}
    \item Disabled "protected mode" in GDSync configuration
    \item Implemented custom permission checks in application logic
    \item Contributed documentation improvements to GDSync project
    \item Shared findings with community to help future users
\end{itemize}

\paragraph{Lesson Learned:}

When using third-party frameworks:
\begin{itemize}
    \item Read source code, not just documentation
    \item Test framework features in isolation before integration
    \item Maintain good logging and diagnostic capabilities
    \item Engage with framework maintainers early when issues arise
\end{itemize}

%----------------------------------------------------------------------------------------
\subsection{Incomplete Documentation}
\label{subsec:gdsync-documentation}

\paragraph{Challenge:}

GDSync documentation covered basic use cases but lacked:
\begin{itemize}
    \item Examples of complex synchronization scenarios
    \item Explanation of internal mechanisms and limitations
    \item Best practices for performance optimization
    \item Troubleshooting guides for common problems
    \item Migration guide from vanilla Godot networking
\end{itemize}

\paragraph{Impact:}

\begin{itemize}
    \item Trial-and-error approach required for advanced features
    \item Difficulty distinguishing between usage errors and framework bugs
    \item Increased development time
    \item Frustration and consideration of framework replacement
\end{itemize}

\paragraph{Mitigation:}

\begin{itemize}
    \item Created internal documentation of discoveries and workarounds
    \item Maintained test project for isolating framework behavior
    \item Contributed examples and documentation improvements to upstream project
    \item Built abstraction layer to isolate GDSync-specific code
\end{itemize}

%----------------------------------------------------------------------------------------
\subsection{Debugging Multiplayer Issues}
\label{subsec:multiplayer-debugging}

\paragraph{Challenge:}

Debugging multiplayer code is inherently difficult:
\begin{itemize}
    \item Requires multiple devices or instances
    \item Race conditions and timing-dependent bugs
    \item Network variability complicates reproduction
    \item Limited visibility into remote client state
\end{itemize}

\paragraph{Solutions Implemented:}

\begin{enumerate}
    \item \textbf{Multi-Instance Testing}: Launched multiple Godot editor instances on the same machine
    \item \textbf{Network Logging}: Comprehensive logging of all network events with timestamps
    \item \textbf{State Dumps}: Ability to print full game state on command
    \item \textbf{Visual Indicators}: On-screen display of connection status, peer IDs, sync status
    \item \textbf{Replay System}: Recorded game events for post-mortem analysis
\end{enumerate}

%----------------------------------------------------------------------------------------
\section{Multiplayer Synchronization Challenges}
\label{sec:sync-challenges}

%----------------------------------------------------------------------------------------
\subsection{Card Order Synchronization}
\label{subsec:card-order-sync}

\paragraph{Challenge:}

Maintaining consistent card ordering across clients proved more complex than anticipated. Issues included:
\begin{itemize}
    \item Godot's scene tree ordering not guaranteed to match logical ordering
    \item Child node order not automatically synchronized
    \item Timing differences causing transient inconsistencies
\end{itemize}

\paragraph{Symptom:}

Cards appeared in different orders on different clients' screens, even though the underlying data was correct.

\paragraph{Solution:}

\begin{enumerate}
    \item Implemented explicit position indices for cards
    \item Synchronized card order separately from card positions
    \item Used z-index to control visual stacking order
    \item Periodic consistency checks and resynchronization
\end{enumerate}

\begin{lstlisting}[caption={Card order synchronization approach}]
# Each card has an explicit order index
var card_order_index: int

# Synchronize order explicitly
@GDSync.rpc(call_local=true)
func sync_card_order(card_id: int, new_index: int):
    var card = get_card_by_id(card_id)
    card.card_order_index = new_index
    resort_cards_by_index()
\end{lstlisting}

%----------------------------------------------------------------------------------------
\subsection{Timing and Race Conditions}
\label{subsec:timing-issues}

\paragraph{Challenge:}

Multiplayer systems are susceptible to race conditions:
\begin{itemize}
    \item Simultaneous card movements by different players
    \item Network messages arriving out of order
    \item Inconsistent state when players join mid-game
\end{itemize}

\paragraph{Examples:}

\begin{itemize}
    \item Two players drag the same card simultaneously
    \item Player joins lobby after game has started
    \item Card deleted on host but still referenced on client
\end{itemize}

\paragraph{Solutions:}

\begin{itemize}
    \item \textbf{Host Authority}: Only host can finalize state changes
    \item \textbf{Client Prediction with Rollback}: Show immediate feedback, then correct if host disagrees
    \item \textbf{State Snapshots}: New players receive full state snapshot on join
    \item \textbf{Defensive Programming}: Check object existence before operations
\end{itemize}

%----------------------------------------------------------------------------------------
\subsection{Different Client Views}
\label{subsec:client-views}

\paragraph{Challenge:}

In MPI mode, different players should see different cards (their private buffers should be hidden from others), but maintaining this selective visibility while synchronizing global state was complex.

\paragraph{Technical Issue:}

GDSync's node replication tries to keep all clients' scene trees identical, but selective visibility required different scene trees per client.

\paragraph{Solution:}

\begin{itemize}
    \item Replicate all cards to all clients (for consistency)
    \item Use visibility flags to hide inappropriate cards
    \item Implement client-side filtering based on ownership metadata
    \item Synchronize ownership and visibility separately from positions
\end{itemize}

%----------------------------------------------------------------------------------------
\section{Mobile UI/UX Challenges}
\label{sec:mobile-ux-challenges}

%----------------------------------------------------------------------------------------
\subsection{Displaying Many Cards on Small Screens}
\label{subsec:many-cards-small-screens}

\paragraph{Challenge:}

Displaying 50--200 cards on a 5--6 inch mobile screen presents significant layout challenges:
\begin{itemize}
    \item Cards must be large enough to read and tap
    \item All cards must be accessible without excessive scrolling
    \item Layout must work in both portrait and landscape orientations
\end{itemize}

\paragraph{Solutions Attempted:}

\begin{enumerate}
    \item \textbf{Grid Layout}: Cards arranged in a grid
          \begin{itemize}
              \item Pros: Space-efficient, familiar pattern
              \item Cons: Hard to see linear order, scrolling still required
          \end{itemize}

    \item \textbf{Scrollable Row}: Single row with horizontal scrolling
          \begin{itemize}
              \item Pros: Linear order visible, simple interaction
              \item Cons: Requires substantial scrolling with many cards
          \end{itemize}

    \item \textbf{Zoomable Canvas}: Pan and zoom to navigate cards
          \begin{itemize}
              \item Pros: Flexible navigation, can fit many cards
              \item Cons: Complex interaction, easy to get lost
          \end{itemize}

    \item \textbf{Hybrid Approach (Final Solution)}:
          \begin{itemize}
              \item Grid layout with intelligent sizing
              \item Vertical scrolling for overflow
              \item Zoom controls for adjusting card size
              \item Mini-map showing overall progress
          \end{itemize}
\end{enumerate}

%----------------------------------------------------------------------------------------
\subsection{Touch Interaction Precision}
\label{subsec:touch-precision}

\paragraph{Challenge:}

Touch input is less precise than mouse input:
\begin{itemize}
    \item Finger occludes card being dragged
    \item Difficulty selecting specific cards when densely packed
    \item Accidental touches and swipes
\end{itemize}

\paragraph{Solutions:}

\begin{itemize}
    \item \textbf{Offset Dragging}: Card rendered above finger with offset
    \item \textbf{Magnification}: Enlarge selected card for better visibility
    \item \textbf{Touch Feedback}: Visual and haptic feedback for touch events
    \item \textbf{Double-Tap Select}: Alternative to drag for card selection
    \item \textbf{Drop Zone Highlighting}: Clear indication of valid drop targets
\end{itemize}

%----------------------------------------------------------------------------------------
\subsection{Performance on Low-End Devices}
\label{subsec:lowend-performance}

\paragraph{Challenge:}

Not all students have flagship smartphones. The game needed to run acceptably on low-end and mid-range devices from 2018--2020.

\paragraph{Performance Constraints:}

\begin{itemize}
    \item Limited GPU power for rendering many cards
    \item Lower memory (2--4 GB RAM)
    \item Slower network hardware
    \item Battery life concerns
\end{itemize}

\paragraph{Optimization Strategies:}

\begin{itemize}
    \item Reduce draw calls through batching
    \item Use texture atlases to minimize state changes
    \item Implement level-of-detail (LOD) for distant cards
    \item Viewport culling to avoid rendering off-screen cards
    \item Adjustable quality settings (card detail, effects)
\end{itemize}

%----------------------------------------------------------------------------------------
\section{Testing Challenges}
\label{sec:testing-challenges}

%----------------------------------------------------------------------------------------
\subsection{Multiplayer Testing Complexity}
\label{subsec:multiplayer-testing}

\paragraph{Challenge:}

Testing multiplayer functionality requires:
\begin{itemize}
    \item Multiple devices or instances
    \item Coordination between test clients
    \item Reproduction of specific network conditions
    \item Testing edge cases (disconnections, late joins, etc.)
\end{itemize}

\paragraph{Approaches:}

\begin{enumerate}
    \item \textbf{Multi-Instance Desktop Testing}: Run multiple Godot instances on development machine
    \item \textbf{Physical Device Testing}: Test on actual Android phones
    \item \textbf{Automated Test Scenarios}: Scripts to simulate player actions
    \item \textbf{Network Simulation}: Tools to introduce latency and packet loss
\end{enumerate}

\paragraph{Limitations:}

\begin{itemize}
    \item Automated testing limited for interactive gameplay
    \item Difficult to reproduce exact timing of race conditions
    \item Real-world network conditions vary widely
\end{itemize}

%----------------------------------------------------------------------------------------
\subsection{Lack of Automated Testing}
\label{subsec:automated-testing}

\paragraph{Challenge:}

Godot's testing ecosystem is less mature than traditional software development frameworks:
\begin{itemize}
    \item No built-in unit testing framework
    \item GUI testing particularly difficult
    \item Networking code hard to test in isolation
\end{itemize}

\paragraph{Approach Taken:}

\begin{itemize}
    \item Manual testing with documented test cases
    \item Integration tests for critical paths
    \item Regression testing checklist before releases
    \item Community feedback as informal usability testing
\end{itemize}

\paragraph{Future Improvement:}

\begin{itemize}
    \item Integrate GUT (Godot Unit Test) framework
    \item Develop automated test suite for core logic
    \item Implement continuous integration pipeline
\end{itemize}

%----------------------------------------------------------------------------------------
\section{Educational Design Challenges}
\label{sec:educational-challenges}

%----------------------------------------------------------------------------------------
\subsection{Balancing Accuracy and Playability}
\label{subsec:accuracy-vs-playability}

\paragraph{Challenge:}

Educational games must balance pedagogical accuracy with engaging gameplay. Too much realism can reduce fun; too much simplification can create misconceptions.

\paragraph{Trade-offs Made:}

\begin{itemize}
    \item \textbf{Simplified Communication}: Actual MPI message passing is more complex than card exchanges
    \item \textbf{Abstracted Synchronization}: Real OpenMP requires explicit barriers and locks
    \item \textbf{Idealized Performance}: Network latency doesn't perfectly model HPC cluster interconnects
\end{itemize}

\paragraph{Mitigation:}

\begin{itemize}
    \item Clear framing: "This is a metaphor, not a simulation"
    \item Post-game discussions to connect game to real concepts
    \item Accompanying documentation explaining simplifications
    \item Progressive difficulty introducing more complexity
\end{itemize}

%----------------------------------------------------------------------------------------
\subsection{Assessment of Learning Effectiveness}
\label{subsec:learning-assessment}

\paragraph{Challenge:}

Demonstrating that the game actually improves learning outcomes requires formal educational research, including:
\begin{itemize}
    \item Pre/post-test measurements
    \item Control groups using traditional instruction
    \item Statistical analysis of results
    \item Long-term retention studies
\end{itemize}

\paragraph{Limitations:}

Within the scope of this thesis, full educational efficacy studies were not feasible due to:
\begin{itemize}
    \item Time constraints
    \item Need for institutional review board approval
    \item Difficulty recruiting sufficient participants
    \item Complexity of isolating game's effect from other factors
\end{itemize}

\paragraph{Future Work:}

Formal educational evaluation remains an important direction for future research, as discussed in Chapter~\ref{ch:conclusion}.

%----------------------------------------------------------------------------------------
\section{Lessons Learned}
\label{sec:lessons-learned}

%----------------------------------------------------------------------------------------
\subsection{Technical Lessons}
\label{subsec:technical-lessons}

\begin{enumerate}
    \item \textbf{Framework Evaluation}: Thoroughly evaluate third-party frameworks before committing; test advanced use cases early.

    \item \textbf{Abstraction Layers}: Isolate framework-specific code to facilitate future replacement if needed.

    \item \textbf{State Management}: Design clear state management patterns early; retrofitting is expensive.

    \item \textbf{Logging Infrastructure}: Comprehensive logging is essential for multiplayer debugging.

    \item \textbf{Performance Profiling}: Profile early and often; don't assume where bottlenecks are.
\end{enumerate}

%----------------------------------------------------------------------------------------
\subsection{Design Lessons}
\label{subsec:design-lessons}

\begin{enumerate}
    \item \textbf{User Testing}: Conduct user testing early; designer assumptions often wrong.

    \item \textbf{Mobile-First}: Design for mobile constraints from the start, not as an afterthought.

    \item \textbf{Simplicity}: Simpler interactions are better on mobile; complexity reduces usability.

    \item \textbf{Feedback}: Clear, immediate feedback is critical for learning and engagement.
\end{enumerate}

%----------------------------------------------------------------------------------------
\subsection{Process Lessons}
\label{subsec:process-lessons}

\begin{enumerate}
    \item \textbf{Iterative Development}: Iterative, incremental development surfaced issues early.

    \item \textbf{Version Control}: Frequent commits with clear messages saved time during debugging.

    \item \textbf{Documentation}: Documenting problems and solutions as they occurred proved invaluable.

    \item \textbf{Community Engagement}: Engaging with open-source communities provided solutions and support.
\end{enumerate}

%----------------------------------------------------------------------------------------
\section{Summary}
\label{sec:problems-summary}

This chapter documented the significant challenges encountered during development:

\begin{itemize}
    \item \textbf{Technology Challenges}: Engine trade-offs, framework limitations, and performance concerns
    \item \textbf{GDSync Issues}: Protected mode blocking, incomplete documentation, and debugging difficulty
    \item \textbf{Synchronization Complexity}: Card ordering, race conditions, and selective visibility
    \item \textbf{Mobile UI/UX}: Screen space constraints, touch precision, and device performance variability
    \item \textbf{Testing Difficulty}: Multiplayer testing complexity and lack of automated testing tools
    \item \textbf{Educational Design}: Balancing accuracy with playability and assessing learning effectiveness
\end{itemize}

Importantly, this chapter also presented solutions and lessons learned that will benefit future developers of similar educational multiplayer games. The next chapter evaluates the completed system and presents performance metrics and assessment results.

\chapter{Results and Evaluation}
\label{ch:results}

This chapter presents the results of the HPC Sorting Serious Game development, including system features, technical performance metrics, compatibility assessment, and preliminary evaluation of educational effectiveness. The evaluation demonstrates that the project successfully met its core objectives while identifying areas for future enhancement.

%-----------------------------------------------------------------------
\section{Completed System Features}
\label{sec:completed-features}

%-----------------------------------------------------------------------
\subsection{Functional Features}
\label{subsec:functional-features}

The implemented system includes all planned core features:

\paragraph*{Single-Player Mode:}
\begin{itemize}
    \item Configurable number of cards (10--200)
    \item Random card shuffling and generation
    \item Drag-and-drop card manipulation
    \item Private buffer zones for local sorting
    \item Sorting validation and completion detection
    \item Timer and move counter
    \item Victory screen with performance summary
\end{itemize}

\paragraph*{Multiplayer Mode:}
\begin{itemize}
    \item Lobby system with room creation and joining
    \item Support for 2--4 players
    \item WebRTC-based peer-to-peer networking
    \item Real-time state synchronization across clients
    \item Shared container visible to all players (OpenMP shared memory)
    \item Selective card visibility (private buffers hidden, simulating thread-private data)
    \item Disconnection handling and graceful degradation
    \item Host migration capability (partial)
\end{itemize}

\paragraph*{User Interface:}
\begin{itemize}
    \item Main menu with mode selection
    \item Settings for game configuration
    \item In-game UI with player indicators
    \item Toast notifications for events
    \item Touch-optimized controls
    \item Responsive layout for different screen sizes
\end{itemize}

%-----------------------------------------------------------------------
\subsection{Educational Features}
\label{subsec:educational-features}

The game successfully implements pedagogical mappings to HPC concepts:

\begin{table}[htbp]
    \centering
    \caption{Implemented pedagogical features}
    \label{tab:implemented-pedagogical}
    \begin{tabular}{@{}lll@{}}
        \toprule
        \textbf{HPC Concept}     & \textbf{Game Representation} & \textbf{Status}   \\
        \midrule
        Sequential Execution     & Single player mode           & Implemented       \\
        Shared Memory (OpenMP)   & Shared card container        & Implemented       \\
        Thread-Local Storage     & Private buffer zones         & Implemented       \\
        Parallel Execution       & Multiple players working     & Implemented       \\
        Memory Access Overhead   & Network latency              & Implicit          \\
        Speedup                  & Completion time comparison   & Implemented       \\
        Scalability              & Variable card/player counts  & Implemented       \\
        Load Balancing           & Equal card distribution      & Implemented       \\
        Synchronization Barriers & Turn-based phases            & Partially         \\
        Race Conditions          & Concurrent card access       & Observable effect \\
        \bottomrule
    \end{tabular}
\end{table}

%-----------------------------------------------------------------------
\section{Technical Performance}
\label{sec:technical-performance}

%-----------------------------------------------------------------------
\subsection{Frame Rate Performance}
\label{subsec:framerate}

Frame rate was measured on three representative Android devices:

\begin{table}[htbp]
    \centering
    \caption{Frame rate performance by device}
    \label{tab:framerate}
    \begin{tabular}{@{}lcccc@{}}
        \toprule
        \textbf{Device}           & \textbf{Cards} & \textbf{Avg FPS} & \textbf{Min FPS} & \textbf{Target} \\
        \midrule
        Google Pixel 6 (2021)     & 50             & 60               & 58               & 60              \\
        Google Pixel 6            & 100            & 60               & 55               & 60              \\
        Google Pixel 6            & 200            & 57               & 48               & 60              \\
        \midrule
        Samsung Galaxy A52 (2021) & 50             & 59               & 54               & 60              \\
        Samsung Galaxy A52        & 100            & 56               & 48               & 60              \\
        Samsung Galaxy A52        & 200            & 51               & 42               & 60              \\
        \midrule
        OnePlus 8T (2020)         & 50             & 60               & 57               & 60              \\
        OnePlus 8T                & 100            & 58               & 51               & 60              \\
        OnePlus 8T                & 200            & 53               & 44               & 60              \\
        \bottomrule
    \end{tabular}
\end{table}

\paragraph*{Analysis:}

\begin{itemize}
    \item Target 60 FPS achieved for 50--100 cards on all devices
    \item Performance degradation with 200 cards acceptable (still playable above 40 FPS)
    \item No significant frame drops during gameplay
    \item Optimization opportunities remain for very high card counts
\end{itemize}

%-----------------------------------------------------------------------
\subsection{Network Latency}
\label{subsec:network-latency}

Network latency was measured for various connection scenarios:

\begin{table}[htbp]
    \centering
    \caption{Network latency measurements}
    \label{tab:network-latency}
    \begin{tabular}{@{}lccc@{}}
        \toprule
        \textbf{Connection Type}  & \textbf{Avg Latency} & \textbf{Max Latency} & \textbf{Packet Loss} \\
        \midrule
        Same WiFi network         & 15 ms                & 45 ms                & 0.1\%                \\
        Same building (WiFi)      & 25 ms                & 80 ms                & 0.3\%                \\
        Different locations (4G)  & 85 ms                & 250 ms               & 1.2\%                \\
        Simulated poor connection & 180 ms               & 450 ms               & 3.5\%                \\
        \bottomrule
    \end{tabular}
\end{table}

\paragraph*{Analysis:}

\begin{itemize}
    \item Local network latency excellent for real-time gameplay
    \item Mobile data connections acceptable for most scenarios
    \item Poor network conditions degrade experience but remain playable
    \item WebRTC NAT traversal successful in all tested configurations
\end{itemize}

%-----------------------------------------------------------------------
\subsection{Memory Usage}
\label{subsec:memory}

Memory consumption measured during typical gameplay:

\begin{table}[htbp]
    \centering
    \caption{Memory usage by game state}
    \label{tab:memory-usage}
    \begin{tabular}{@{}lcc@{}}
        \toprule
        \textbf{Game State}          & \textbf{Memory (MB)} & \textbf{Peak (MB)} \\
        \midrule
        Startup (Main Menu)          & 85                   & 92                 \\
        Single-player (50 cards)     & 110                  & 125                \\
        Single-player (100 cards)    & 135                  & 155                \\
        Single-player (200 cards)    & 175                  & 210                \\
        Multiplayer lobby            & 120                  & 140                \\
        Multiplayer game (4 players) & 165                  & 195                \\
        \bottomrule
    \end{tabular}
\end{table}

\paragraph*{Analysis:}

\begin{itemize}
    \item Memory usage reasonable for modern smartphones (typically 4--8 GB available)
    \item No memory leaks detected during extended gameplay sessions
    \item Object pooling successfully reduced allocation overhead
    \item Room for further optimization if targeting very low-end devices
\end{itemize}

%-----------------------------------------------------------------------
\subsection{APK Size}
\label{subsec:apk-size}

The final Android APK package characteristics:

\begin{itemize}
    \item \textbf{Release APK Size}: 28.5 MB
    \item \textbf{Debug APK Size}: 35.2 MB
    \item \textbf{Compressed (download)}: 22.1 MB
\end{itemize}

\paragraph*{Size Breakdown:}
\begin{itemize}
    \item Godot Engine runtime: 18 MB
    \item Game scripts and scenes: 2 MB
    \item Assets (textures, fonts, audio): 5 MB
    \item Libraries (WebRTC, GDSync): 3.5 MB
\end{itemize}

\paragraph*{Comparison:}

Significantly smaller than typical Unity (50--100 MB) or Unreal Engine (100+ MB) games, facilitating mobile distribution and download.

%-----------------------------------------------------------------------
\section{Platform Compatibility}
\label{sec:platform-compatibility}

%-----------------------------------------------------------------------
\subsection{Android Compatibility}
\label{subsec:android-compatibility}

Testing conducted on diverse Android devices:

\begin{table}[htbp]
    \centering
    \caption{Android device compatibility}
    \label{tab:android-compat}
    \begin{tabular}{@{}lccl@{}}
        \toprule
        \textbf{Device}       & \textbf{Android} & \textbf{Status} & \textbf{Notes}         \\
        \midrule
        Google Pixel 6        & 13               & Excellent       & Reference device       \\
        Samsung Galaxy A52    & 12               & Excellent       & Mid-range target       \\
        OnePlus 8T            & 11               & Excellent       & Performance good       \\
        Samsung Galaxy S10    & 10               & Good            & Older flagship         \\
        Xiaomi Redmi Note 9   & 9                & Good            & Budget device          \\
        Motorola Moto G7      & 9                & Acceptable      & Low-end, some slowness \\
        Samsung Galaxy Tab S7 & 12               & Excellent       & Tablet (landscape)     \\
        \bottomrule
    \end{tabular}
\end{table}

\paragraph*{Minimum Requirements:}

\begin{itemize}
    \item Android 7.0 (API level 24) or higher
    \item 2 GB RAM minimum, 4 GB recommended
    \item OpenGL ES 3.0 support
    \item Internet connection for multiplayer
\end{itemize}

%-----------------------------------------------------------------------
\subsection{Screen Size and Orientation}
\label{subsec:screen-size}

The game adapts to various screen configurations:

\begin{itemize}
    \item \textbf{Small phones (5--5.5 inches)}: Usable but somewhat cramped with many cards
    \item \textbf{Medium phones (5.5--6.5 inches)}: Optimal experience, target size range
    \item \textbf{Large phones/phablets (6.5--7 inches)}: Excellent, plenty of space
    \item \textbf{Tablets (7--10 inches)}: Excellent, especially in landscape orientation
\end{itemize}

\paragraph*{Orientation Support:}

\begin{itemize}
    \item \textbf{Portrait}: Default, optimized for one-handed play
    \item \textbf{Landscape}: Supported, provides more horizontal space for card layout
    \item Dynamic layout adjustment on rotation
\end{itemize}

%-----------------------------------------------------------------------
\section{Usability Evaluation}
\label{sec:usability-evaluation}

%-----------------------------------------------------------------------
\subsection{Informal User Testing}
\label{subsec:informal-testing}

Informal usability testing conducted with 12 volunteer participants (students and colleagues):

\paragraph*{Demographics:}
\begin{itemize}
    \item Age range: 20--35 years
    \item 8 with prior HPC knowledge, 4 without
    \item Mix of gamers and non-gamers
    \item All had smartphone experience
\end{itemize}

\paragraph*{Testing Protocol:}
\begin{enumerate}
    \item Brief introduction to project goals (no gameplay instruction)
    \item Observation of first-time gameplay
    \item Think-aloud protocol during play
    \item Post-session questionnaire
    \item Semi-structured interview
\end{enumerate}

%-----------------------------------------------------------------------
\subsection{Key Findings}
\label{subsec:usability-findings}

\paragraph*{Positive Feedback:}

\begin{itemize}
    \item \textbf{Intuitive Controls}: 11/12 participants understood drag-and-drop immediately without instruction
    \item \textbf{Visual Clarity}: Card numbers and states were clearly visible
    \item \textbf{Conceptual Connection}: Participants with HPC background (8/8) recognized OpenMP shared-memory parallels
    \item \textbf{Engagement}: Average play session 15--20 minutes, indicating good engagement
    \item \textbf{Performance}: No complaints about lag or performance issues
\end{itemize}

\paragraph*{Areas for Improvement:}

\begin{itemize}
    \item \textbf{Onboarding}: 6/12 participants wanted brief tutorial or tooltips
    \item \textbf{Buffer Purpose}: 5/12 weren't immediately sure why buffers were useful
    \item \textbf{Multiplayer Clarity}: 7/12 found it unclear what other players were doing
    \item \textbf{Educational Context}: Participants without HPC background (4/4) didn't make HPC connections without explanation
\end{itemize}

%-----------------------------------------------------------------------
\subsection{Suggested Improvements}
\label{subsec:suggested-improvements}

Based on feedback, future versions should include:

\begin{enumerate}
    \item \textbf{Optional Tutorial}: Brief interactive tutorial explaining mechanics
    \item \textbf{Tooltips and Hints}: Contextual help for first-time users
    \item \textbf{Player Activity Indicators}: Show what other players are doing
    \item \textbf{Educational Overlay}: Optional explanations connecting gameplay to HPC concepts
    \item \textbf{Strategy Hints}: Suggestions for efficient sorting approaches
    \item \textbf{Replay/Analysis Mode}: Review completed games to understand performance
\end{enumerate}

%-----------------------------------------------------------------------
\section{Educational Effectiveness}
\label{sec:educational-effectiveness}

%-----------------------------------------------------------------------
\subsection{Preliminary Assessment}
\label{subsec:preliminary-assessment}

While comprehensive educational evaluation is beyond the scope of this thesis, preliminary indicators suggest educational value:

\paragraph*{Conceptual Understanding (HPC-Experienced Participants):}

Post-gameplay interviews revealed that participants with prior HPC knowledge:

\begin{itemize}
    \item \textbf{8/8} recognized single-player mode as shared-memory analogy
    \item \textbf{7/8} identified multiplayer mode as distributed-memory simulation
    \item \textbf{6/8} connected buffers to thread-local storage or private memory
    \item \textbf{7/8} noted communication overhead when exchanging cards
    \item \textbf{5/8} discussed load balancing when cards were unevenly distributed
\end{itemize}

\paragraph*{Conceptual Introduction (Non-HPC Participants):}

Participants without HPC background:

\begin{itemize}
    \item \textbf{4/4} understood basic parallelism concept (multiple players working simultaneously)
    \item \textbf{3/4} recognized that coordination becomes harder with more players
    \item \textbf{2/4} understood communication trade-offs (passing cards takes time)
    \item After brief explanation, \textbf{4/4} expressed interest in learning more about parallel computing
\end{itemize}

\paragraph*{Engagement and Motivation:}

\begin{itemize}
    \item \textbf{10/12} participants expressed willingness to play again
    \item \textbf{9/12} would recommend to peers learning HPC
    \item \textbf{7/12} found it more engaging than reading textbook explanations
    \item \textbf{11/12} appreciated the hands-on, interactive approach
\end{itemize}

%-----------------------------------------------------------------------
\subsection{Pedagogical Value Proposition}
\label{subsec:pedagogical-value}

The game serves as an effective \textbf{icebreaker or introductory activity}:

\begin{itemize}
    \item \textbf{Pre-Lecture Activity}: Introduce concepts before formal instruction
    \item \textbf{Discussion Starter}: Generate questions and curiosity about parallel computing
    \item \textbf{Concept Visualization}: Provide mental model for abstract concepts
    \item \textbf{Reinforcement Tool}: Practice and reinforce learned concepts
    \item \textbf{Assessment Aid}: Reveal misconceptions for instructors to address
\end{itemize}

%-----------------------------------------------------------------------
\subsection{Comparison with Traditional Methods}
\label{subsec:comparison-traditional}

Informal comparison with traditional HPC teaching methods (based on participant feedback):

\begin{table}[htbp]
    \centering
    \caption{Perceived advantages over traditional methods}
    \label{tab:comparison-traditional}
    \begin{tabular}{@{}lcc@{}}
        \toprule
        \textbf{Criterion}          & \textbf{Game} & \textbf{Lecture/Textbook} \\
        \midrule
        Engagement                  & High          & Medium-Low                \\
        Immediate Feedback          & Yes           & No                        \\
        Hands-On Experience         & Yes           & No                        \\
        Conceptual Visualization    & Strong        & Weak                      \\
        Detailed Explanation        & Weak          & Strong                    \\
        Accessibility               & High          & Medium                    \\
        Scalability (student count) & High          & High                      \\
        Time Required               & 15--30 min    & 60--90 min (lecture)      \\
        \bottomrule
    \end{tabular}
\end{table}

\paragraph*{Conclusion:}

The game is not a replacement for traditional instruction but a valuable complementary tool that enhances engagement and provides intuitive understanding before formal study.

%-----------------------------------------------------------------------
\section{Comparison with Initial Requirements}
\label{sec:requirements-comparison}

%-----------------------------------------------------------------------
\subsection{Requirements Fulfillment}
\label{subsec:requirements-fulfillment}

Assessment of how well the completed system meets initial requirements:

\begin{table}[htbp]
    \centering
    \caption{Requirements fulfillment status}
    \label{tab:requirements-status}
    \begin{tabular}{@{}llc@{}}
        \toprule
        \textbf{Category} & \textbf{Requirement}  & \textbf{Status}   \\
        \midrule
        Educational       & Sequential baseline   & \checkmark Met    \\
                          & OpenMP simulation     & \checkmark Met    \\
                          & Performance feedback  & \checkmark Met    \\
                          & Scalable difficulty   & \checkmark Met    \\
                          & Conceptual clarity    & \checkmark Mostly \\
        \midrule
        Functional        & Card management       & \checkmark Met    \\
                          & Single-player mode    & \checkmark Met    \\
                          & Multiplayer mode      & \checkmark Met    \\
                          & User interface        & \checkmark Met    \\
                          & Feedback systems      & \checkmark Met    \\
        \midrule
        Non-Functional    & Performance (60 FPS)  & \checkmark Mostly \\
                          & Usability             & \checkmark Met    \\
                          & Reliability           & \checkmark Met    \\
                          & Portability (Android) & \checkmark Met    \\
                          & Maintainability       & \checkmark Met    \\
                          & Accessibility         & \checkmark Met    \\
        \bottomrule
    \end{tabular}
\end{table}

\paragraph*{Notes:}

\begin{itemize}
    \item \textbf{Conceptual Clarity}: Strong for HPC-experienced users; needs educational overlay for beginners
    \item \textbf{Performance}: 60 FPS achieved for typical use cases (50--100 cards); slight degradation with 200 cards
    \item \textbf{Portability}: Android fully supported; iOS and web exports remain future work
\end{itemize}

%-----------------------------------------------------------------------
\section{Known Limitations}
\label{sec:limitations}

%-----------------------------------------------------------------------
\subsection{Current Limitations}
\label{subsec:current-limitations}

\begin{enumerate}
    \item \textbf{Platform Support}: Currently Android-only; iOS and web versions not yet implemented

    \item \textbf{Educational Scaffolding}: Limited in-game explanations; requires instructor context

    \item \textbf{Sorting Algorithms}: Only supports general sorting; doesn't demonstrate specific parallel algorithms (merge sort, sample sort)

    \item \textbf{Performance Metrics}: Basic timing and moves; lacks detailed profiling (speedup curves, scalability graphs)

    \item \textbf{Advanced HPC Concepts}: Doesn't cover race conditions, deadlocks, synchronization primitives explicitly

    \item \textbf{Automated Assessment}: No built-in assessment or progress tracking for educational use

    \item \textbf{Formal Evaluation}: Lacks comprehensive educational efficacy study
\end{enumerate}

%-----------------------------------------------------------------------
\subsection{Scope Limitations}
\label{subsec:scope-limitations}

Certain features were intentionally deferred to maintain project scope:

\begin{itemize}
    \item User accounts and progress persistence
    \item Leaderboards and social features
    \item Advanced sorting algorithm demonstrations
    \item GPU parallelism (CUDA/OpenCL) simulations
    \item Instructor dashboard for classroom management
    \item Detailed analytics and learning analytics integration
\end{itemize}

%-----------------------------------------------------------------------
\section{Summary}
\label{sec:results-summary}

This chapter presented comprehensive evaluation results:

\begin{itemize}
    \item \textbf{Completed Features}: All core functional and educational features successfully implemented
    \item \textbf{Technical Performance}: Meets or exceeds performance targets on representative devices
    \item \textbf{Platform Compatibility}: Excellent Android compatibility across device range
    \item \textbf{Usability}: Generally positive usability feedback with actionable improvement suggestions
    \item \textbf{Educational Effectiveness}: Preliminary evidence suggests value as introductory/supplementary tool
    \item \textbf{Requirements Fulfillment}: Nearly all initial requirements met or exceeded
    \item \textbf{Limitations}: Known limitations documented for transparency and future work planning
\end{itemize}

The results demonstrate that the project successfully achieved its core objective: creating a functional, engaging serious game for teaching HPC concepts on mobile platforms. The next chapter concludes the thesis and discusses future directions for research and development.

\chapter{Conclusion and Future Work}
\label{ch:conclusion}

This final chapter summarizes the contributions of this thesis, reflects on how the research questions were answered, discusses implications for HPC education and serious game development, and outlines comprehensive directions for future work.

%-----------------------------------------------------------------------
\section{Summary of Contributions}
\label{sec:summary-contributions}

This thesis presented the design, implementation, and evaluation of the HPC Sorting Serious Game, a web-first educational tool for teaching parallel computing concepts through interactive card-sorting gameplay.

%-----------------------------------------------------------------------
\subsection{Primary Contributions}
\label{subsec:primary-contributions}

The main contributions of this work include:

\paragraph*{1. Novel Pedagogical Approach}

\begin{itemize}
    \item Digital transformation of Professor D'Agostino's successful physical classroom experiments into a scalable, accessible web-based application
    \item Systematic mapping of HPC concepts (specifically OpenMP shared-memory parallelism and sequential vs. parallel execution) to concrete, manipulable game mechanics
    \item Demonstration that serious games can serve as effective icebreakers for teaching abstract parallel computing concepts
\end{itemize}

\paragraph*{2. Technical Implementation}

\begin{itemize}
    \item Functional serious game prototype demonstrating real-time multiplayer synchronization on mobile platforms
    \item Solutions to multiplayer state management challenges, particularly selective visibility and host-authoritative architecture
    \item Integration of GDSync framework with WebRTC for peer-to-peer educational gaming
    \item Mobile UI/UX design patterns for displaying and manipulating numerous interactive elements on small screens
\end{itemize}

\paragraph*{3. Documentation and Knowledge Transfer}

\begin{itemize}
    \item Comprehensive documentation of technical challenges (especially GDSync framework issues) and solutions
    \item Identification of design patterns and best practices for educational multiplayer game development
    \item Open-source codebase (MIT license) enabling reuse, extension, and further research
    \item Contributions to GDSync project documentation and issue resolution
\end{itemize}

\paragraph*{4. Preliminary Validation}

\begin{itemize}
    \item Demonstration of technical feasibility through performance metrics and compatibility testing
    \item Positive preliminary usability feedback from informal user testing
    \item Evidence suggesting educational value as supplementary/introductory tool for HPC education
\end{itemize}

%-----------------------------------------------------------------------
\section{Research Questions Revisited}
\label{sec:research-questions}

Returning to the research problem stated in Chapter~\ref{ch:introduction}:

\begin{quote}
    \textit{How can we design and implement an effective serious game for web platforms that teaches fundamental High-Performance Computing concepts (specifically OpenMP shared-memory parallelism and sequential vs. parallel execution) through interactive card-sorting gameplay while overcoming the technical challenges of multiplayer state synchronization and responsive UI/UX constraints?}
\end{quote}

%-----------------------------------------------------------------------
\subsection{Addressing the Educational Problem}
\label{subsec:educational-problem-answered}

\textbf{Q: How can abstract parallel computing concepts be mapped to concrete, understandable game mechanics?}

\textbf{A:} The thesis demonstrated successful mapping through:
\begin{itemize}
    \item Card-sorting as metaphor for data manipulation
    \item Shared container representing shared memory (OpenMP)
    \item Private buffers representing thread-local storage (OpenMP)
    \item Card movement between shared container and private buffers representing memory access patterns
    \item Network latency representing synchronization and communication overhead
    \item Timer and move counter representing performance metrics
\end{itemize}

This mapping was validated through recognition by HPC-experienced participants and engagement by HPC-novice participants.

%-----------------------------------------------------------------------
\subsection{Addressing the Technical Problem}
\label{subsec:technical-problem-answered}

\textbf{Q: How can we implement real-time multiplayer gameplay on mobile devices with acceptable performance?}

\textbf{A:} The thesis demonstrated that:
\begin{itemize}
    \item WebRTC peer-to-peer networking eliminates server costs while providing low latency
    \item GDSync framework (despite challenges) simplifies state synchronization
    \item Host-authoritative architecture balances consistency and responsiveness
    \item Godot Engine provides adequate performance for 2D card-based gameplay
    \item Target 60 FPS achieved for typical scenarios on mid-range devices
\end{itemize}

%-----------------------------------------------------------------------
\subsection{Addressing the Design Problem}
\label{subsec:design-problem-answered}

\textbf{Q: How can we balance educational effectiveness with engagement and playability on mobile platforms?}

\textbf{A:} The thesis showed that:
\begin{itemize}
    \item Touch-based drag-and-drop provides intuitive, engaging interaction
    \item Visual clarity and immediate feedback support learning
    \item Scalable difficulty accommodates learners at different levels
    \item Game mechanics naturally lead to questions about parallel computing
    \item Informal testing suggests good engagement without sacrificing educational goals
\end{itemize}

However, the design can be improved with better onboarding and educational scaffolding for learners without prior HPC exposure.

%-----------------------------------------------------------------------
\section{Implications}
\label{sec:implications}

%-----------------------------------------------------------------------
\subsection{For HPC Education}
\label{subsec:implications-hpc-education}

This work has several implications for teaching parallel computing:

\paragraph*{Complementary Tool, Not Replacement:}

Serious games are most valuable as \textbf{complementary tools} alongside traditional instruction:
\begin{itemize}
    \item Use as pre-lecture icebreaker to introduce concepts
    \item Employ as reinforcement after formal instruction
    \item Leverage to uncover student misconceptions
    \item Deploy as discussion starter for deeper exploration
\end{itemize}

\paragraph*{Accessibility Matters:}

Web-first design significantly enhances accessibility:
\begin{itemize}
    \item No specialized HPC hardware required
    \item No app installation needed—runs directly in browsers
    \item Students can access from any device with a web browser
    \item Cross-platform compatibility (Windows, macOS, Linux, mobile)
    \item Reduces barriers to entry for HPC education
\end{itemize}

\paragraph*{Visualization Is Powerful:}

Making abstract concepts visible and manipulable:
\begin{itemize}
    \item Helps students build intuitive mental models
    \item Facilitates understanding before mathematical formalization
    \item Engages different learning styles
    \item Makes HPC concepts less intimidating
\end{itemize}

%-----------------------------------------------------------------------
\subsection{For Serious Game Development}
\label{subsec:implications-serious-games}

This work offers insights for developers of educational multiplayer games:

\paragraph*{Framework Selection is Critical:}

\begin{itemize}
    \item Thoroughly evaluate frameworks before committing
    \item Test advanced use cases early in development
    \item Build abstraction layers to facilitate future framework changes
    \item Engage with framework maintainers proactively
\end{itemize}

\paragraph*{Multiplayer Adds Complexity:}

\begin{itemize}
    \item Start with single-player; add multiplayer incrementally
    \item Host-authoritative model simplifies consistency management for educational games
    \item Debugging multiplayer requires specialized tools and strategies
    \item Network variability must be designed for, not retrofitted
\end{itemize}

\paragraph*{Mobile Constraints Are Real:}

\begin{itemize}
    \item Design for small screens and touch input from the start
    \item Performance optimization is essential, not optional
    \item Battery life and data usage are important considerations
    \item Test on diverse devices, not just flagship models
\end{itemize}

%-----------------------------------------------------------------------
\section{Future Work}
\label{sec:future-work}

Numerous opportunities exist for extending and improving this work.

%-----------------------------------------------------------------------
\subsection{Short-Term Enhancements}
\label{subsec:short-term-enhancements}

Improvements that could be implemented in 1--3 months:

\paragraph*{Onboarding and Tutorial:}
\begin{itemize}
    \item Interactive tutorial explaining game mechanics
    \item Tooltips and contextual hints for first-time users
    \item Guided first playthrough highlighting key concepts
\end{itemize}

\paragraph*{Educational Overlay:}
\begin{itemize}
    \item Optional explanations connecting gameplay to HPC concepts
    \item Pop-up definitions of terms (thread, process, synchronization)
    \item Post-game summary explaining what was learned
    \item Links to additional resources for deeper learning
\end{itemize}

\paragraph*{Enhanced Feedback:}
\begin{itemize}
    \item Player activity indicators showing what others are doing
    \item Visual representation of communication patterns
    \item Performance analysis comparing strategies
    \item Suggestions for improvement
\end{itemize}

\paragraph*{iOS Support:}
\begin{itemize}
    \item Export to iOS using Godot's iOS templates
    \item Address Apple App Store requirements
    \item Test on iPhone and iPad devices
\end{itemize}

%-----------------------------------------------------------------------
\subsection{Medium-Term Extensions}
\label{subsec:medium-term-extensions}

Enhancements requiring 3--6 months of development:

\paragraph*{Additional Game Modes:}

\begin{enumerate}
    \item \textbf{Algorithm-Specific Modes}: Demonstrate specific parallel sorting algorithms
          \begin{itemize}
              \item Parallel merge sort
              \item Sample sort (distributed-memory style, future MPI extension)
              \item Bitonic sort
          \end{itemize}

    \item \textbf{Advanced Concepts}: Introduce more HPC topics
          \begin{itemize}
              \item Synchronization barriers and critical sections
              \item Race conditions and deadlock scenarios
              \item Load balancing challenges
              \item Communication patterns (broadcast, scatter/gather, reduce)
          \end{itemize}

    \item \textbf{Competitive Modes}: Enhance engagement
          \begin{itemize}
              \item Time trials with leaderboards
              \item Team vs. team competitions
              \item Challenge modes with specific constraints
          \end{itemize}
\end{enumerate}

\paragraph*{Enhanced Analytics:}

\begin{itemize}
    \item Detailed performance metrics (speedup, efficiency, scalability)
    \item Strategy analysis and optimization suggestions
    \item Comparison with theoretical optimal performance
    \item Progress tracking across multiple sessions
\end{itemize}

\paragraph*{Web Export:}

\begin{itemize}
    \item HTML5 export for browser-based play
    \item Integration with learning management systems (Moodle, Canvas)
    \item Embeddable widget for course websites
    \item Cross-platform multiplayer (mobile + web)
\end{itemize}

\paragraph*{Instructor Dashboard:}

\begin{itemize}
    \item Classroom management interface
    \item Real-time monitoring of student gameplay
    \item Assessment and analytics integration
    \item Customizable game parameters and scenarios
\end{itemize}

%-----------------------------------------------------------------------
\subsection{Long-Term Research Directions}
\label{subsec:long-term-research}

Research directions requiring 6+ months and potential collaborations:

\paragraph*{Formal Educational Evaluation:}

Conduct rigorous educational research to assess learning outcomes:

\begin{itemize}
    \item \textbf{Study Design}: Pre/post-test controlled study with treatment and control groups
    \item \textbf{Participants}: Undergraduate students in HPC/parallel computing courses
    \item \textbf{Measures}: Knowledge assessments, concept inventories, attitude surveys
    \item \textbf{Analysis}: Statistical comparison of learning gains between groups
    \item \textbf{Publication}: Results submitted to CS education conferences (SIGCSE, ICER)
\end{itemize}

\paragraph*{Learning Analytics Integration:}

\begin{itemize}
    \item Instrument game to collect detailed interaction data
    \item Analyze gameplay patterns to identify learning difficulties
    \item Develop predictive models for student understanding
    \item Personalize game difficulty based on demonstrated mastery
    \item Research publication in learning analytics venues (LAK, EDM)
\end{itemize}

\paragraph*{GPU Parallelism Extension:}

Extend game to cover GPU computing concepts (CUDA, OpenCL):

\begin{itemize}
    \item Massive parallelism simulation (thousands of "mini-workers")
    \item Warp/wavefront concept representation
    \item Memory hierarchy visualization (global, shared, local)
    \item Divergence and bank conflict demonstrations
\end{itemize}

\paragraph*{Distributed Memory Extension (MPI):}

Future extension to simulate MPI distributed-memory programming:

\begin{itemize}
    \item Separate player "nodes" with isolated memory spaces
    \item Explicit message passing between nodes (simulating MPI\_Send/MPI\_Recv)
    \item Support for hybrid MPI+OpenMP programming models
    \item Hierarchical parallelism with shared memory within nodes
    \item Load balancing across heterogeneous resources
\end{itemize}

\paragraph*{AI-Assisted Learning:}

Integrate AI to enhance educational effectiveness:

\begin{itemize}
    \item AI opponent for single-player mode
    \item Intelligent tutoring system providing hints
    \item Automatic difficulty adjustment
    \item Natural language explanations of concepts
\end{itemize}

%-----------------------------------------------------------------------
\subsection{Community and Ecosystem Development}
\label{subsec:community-development}

\paragraph*{Open Source Community Building:}

\begin{itemize}
    \item Publish to GitHub with comprehensive documentation
    \item Create contributor guidelines and issue templates
    \item Engage educational computing communities
    \item Accept contributions from students and educators
    \item Maintain active development and support
\end{itemize}

\paragraph*{Educational Resource Development:}

\begin{itemize}
    \item Instructor guides with lesson plans
    \item Student worksheets and reflection prompts
    \item Video tutorials and gameplay examples
    \item Integration guides for popular LMS platforms
    \item Curriculum modules incorporating the game
\end{itemize}

\paragraph*{Conference and Journal Publications:}

Disseminate findings to relevant research communities:

\begin{itemize}
    \item \textbf{CS Education}: SIGCSE, ICER, ITiCSE
    \item \textbf{HPC Education}: EduHPC, XSEDE Education Track
    \item \textbf{Serious Games}: Games+Learning+Society, FDG
    \item \textbf{Educational Technology}: Learning @ Scale, EC-TEL
    \item \textbf{HPC Venues}: SC Education Program, ISC tutorials
\end{itemize}

Potential paper topics:
\begin{enumerate}
    \item Design and implementation of mobile serious game for HPC education
    \item Evaluation of learning outcomes with serious game intervention
    \item Technical challenges and solutions in educational multiplayer games
    \item Pedagogical mapping from physical to digital activities
\end{enumerate}

%-----------------------------------------------------------------------
\section{Limitations and Reflections}
\label{sec:limitations-reflections}

%-----------------------------------------------------------------------
\subsection{Methodological Limitations}
\label{subsec:methodological-limitations}

\subsubsection{Evaluation Scope:}

\begin{itemize}
    \item Informal usability testing with small sample (n=12)
    \item No formal educational efficacy study
    \item Limited diversity in participant demographics
    \item Short-term evaluation only (no long-term retention studies)
\end{itemize}

\subsubsection{Technical Scope:}

\begin{itemize}
    \item Single platform implementation (Android only)
    \item Focus on basic parallel concepts (no advanced topics)
    \item Simplified metaphors (not high-fidelity simulations)
    \item Limited integration with formal educational contexts
\end{itemize}

%-----------------------------------------------------------------------
\subsection{Reflections on the Development Process}
\label{subsec:development-reflections}

\paragraph*{What Worked Well:}

\begin{itemize}
    \item Iterative development allowed course correction
    \item Open-source toolchain reduced costs and increased flexibility
    \item Community engagement provided valuable support
    \item Documentation-first approach facilitated thesis writing
    \item Informal testing surfaced usability issues early
\end{itemize}

\paragraph*{What Could Be Improved:}

\begin{itemize}
    \item Earlier, more frequent testing with target users
    \item More thorough framework evaluation before commitment
    \item Formal project management and milestone tracking
    \item Automated testing infrastructure from the start
    \item More structured approach to educational validation
\end{itemize}

%-----------------------------------------------------------------------
\section{Concluding Remarks}
\label{sec:concluding-remarks}

This thesis presented the design, implementation, and evaluation of the HPC Sorting Serious Game, demonstrating that:

\begin{enumerate}
    \item \textbf{Serious games can effectively teach HPC concepts} through intuitive, engaging mechanics that transform abstract ideas into concrete, manipulable interactions.

    \item \textbf{Mobile platforms are viable for educational multiplayer games}, despite technical challenges in state synchronization, UI/UX design, and performance optimization.

    \item \textbf{Open-source technologies enable accessible HPC education} by eliminating cost barriers and empowering educators to customize and extend educational tools.

    \item \textbf{Interdisciplinary challenges inspire innovative solutions} at the intersection of computer science education, game design, parallel computing, and mobile development.
\end{enumerate}

The HPC Sorting Serious Game represents a step toward making High-Performance Computing education more accessible, engaging, and effective. By leveraging the ubiquity of smartphones and the motivational power of games, this work contributes to democratizing access to advanced computing concepts.

As parallel computing becomes increasingly fundamental to modern software engineering—from multicore CPUs to cloud computing to AI/ML workloads—innovative educational approaches are essential to prepare the next generation of computing professionals. Serious games, when thoughtfully designed and rigorously evaluated, can play a valuable role in this educational mission.

The author hopes this work inspires further research and development in educational technologies for HPC and serves as a useful resource for educators, students, and game developers interested in the serious games domain.

%-----------------------------------------------------------------------
\section{Final Thoughts}
\label{sec:final-thoughts}

\[To be written last, after all revisions are complete.\]
Teaching parallel computing remains challenging, but it need not be boring or inaccessible. By combining pedagogical insight, technical innovation, and game design principles, we can create learning experiences that are both effective and enjoyable.

The journey from physical card-sorting experiments in a classroom to a digital multiplayer mobile game demonstrates the potential for technology to scale and extend proven teaching methods. While the game developed in this thesis is far from perfect, it represents a meaningful contribution to the ongoing effort to make HPC education more engaging and accessible.

As we look to the future, the convergence of mobile computing, cloud infrastructure, and educational research offers exciting possibilities for serious games in computing education. The author looks forward to seeing how this work evolves, inspires further innovations, and ultimately contributes to better outcomes for students learning parallel computing.

\vspace{1cm}

\begin{center}
    \textit{``The best way to learn is by doing.''}\\
    \textit{— Ancient Proverb}
\end{center}

%-----------------------------------------------------------------------



%	BIBLIOGRAPHY
%-----------------------------------------------------------------------

\printbibliography[heading=bibintoc]


%	APPENDICES
%-----------------------------------------------------------------------

\appendix

\chapter{User Manual}
%----------------------------------------------------------------------------------------
% Appendix A: User Manual
%----------------------------------------------------------------------------------------

\section{Introduction}
\label{app:a:introduction}

This user manual provides comprehensive instructions for installing, configuring, and playing the HPC Sorting Serious Game. It is intended for students, educators, and anyone interested in learning about parallel computing through interactive gameplay.

%----------------------------------------------------------------------------------------
\section{System Requirements}
\label{app:a:system-requirements}

\subsection{Minimum Requirements}

\begin{itemize}
    \item \textbf{Operating System}: Android 7.0 (API level 24) or higher
    \item \textbf{RAM}: 2 GB minimum
    \item \textbf{Storage}: 50 MB free space
    \item \textbf{Screen}: 5-inch display or larger
    \item \textbf{Graphics}: OpenGL ES 3.0 support
    \item \textbf{Network}: WiFi or mobile data (for multiplayer mode only)
\end{itemize}

\subsection{Recommended Requirements}

\begin{itemize}
    \item Android 9.0 or higher
    \item 4 GB RAM
    \item 5.5--6.5 inch display
    \item Stable WiFi connection for multiplayer
\end{itemize}

%----------------------------------------------------------------------------------------
\section{Installation}
\label{app:a:installation}

\subsection{Installing from APK}

\begin{enumerate}
    \item Download the APK file from the official repository or distribution site
    \item Navigate to the APK file using your device's file manager
    \item Tap the APK file to begin installation
    \item If prompted, allow installation from unknown sources in Settings
    \item Follow the on-screen installation prompts
    \item Launch the app from your app drawer
\end{enumerate}

\subsection{Installing from Google Play (Future)}

Once published to Google Play Store:
\begin{enumerate}
    \item Open Google Play Store app
    \item Search for "HPC Sorting Serious Game"
    \item Tap "Install"
    \item Wait for download and installation to complete
    \item Tap "Open" to launch
\end{enumerate}

%----------------------------------------------------------------------------------------
\section{Getting Started}
\label{app:a:getting-started}

\subsection{First Launch}

On first launch, you will see the main menu with options:
\begin{itemize}
    \item \textbf{Single Player}: Practice sorting alone (OpenMP simulation)
    \item \textbf{Multiplayer}: Join or host a game with others (MPI simulation)
    \item \textbf{Settings}: Configure game options
    \item \textbf{About}: Information about the game and credits
\end{itemize}

%----------------------------------------------------------------------------------------
\section{Single-Player Mode}
\label{app:a:singleplayer-mode}

\subsection{Starting a Single-Player Game}

\begin{enumerate}
    \item Tap "Single Player" from the main menu
    \item Configure game settings:
          \begin{itemize}
              \item Number of cards: 10--200 (recommended: 50 for beginners)
              \item Difficulty: Easy, Medium, Hard
          \end{itemize}
    \item Tap "Start Game"
\end{enumerate}

\subsection{Gameplay}

\paragraph{Objective:}
Sort all cards in ascending order in the main container as quickly as possible.

\paragraph{Controls:}
\begin{itemize}
    \item \textbf{Drag}: Long-press on a card and drag to move it
    \item \textbf{Drop}: Release finger to drop card in desired location
    \item \textbf{Scroll}: Swipe vertically to scroll through cards
    \item \textbf{Zoom}: Pinch to zoom in/out (if enabled)
\end{itemize}

\paragraph{Using Buffers:}
\begin{itemize}
    \item Each player has 2--3 private buffer zones
    \item Drag cards into buffers for temporary storage
    \item Sort cards within buffers before merging back
    \item Buffers simulate thread-local storage in parallel computing
\end{itemize}

\paragraph{Winning:}
\begin{itemize}
    \item Game completes when all cards are in correct ascending order
    \item Victory screen shows completion time and number of moves
    \item Try to minimize both time and moves for better performance
\end{itemize}

%----------------------------------------------------------------------------------------
\section{Multiplayer Mode}
\label{app:a:multiplayer-mode}

\subsection{Creating a Game}

\begin{enumerate}
    \item Tap "Multiplayer" from the main menu
    \item Tap "Create Game"
    \item Configure settings:
          \begin{itemize}
              \item Number of players: 2--4
              \item Number of cards: 10--200
              \item Game mode: Cooperative or Competitive
          \end{itemize}
    \item A unique room code will be generated
    \item Share this code with other players
    \item Wait for players to join
    \item Once all players are ready, tap "Start Game"
\end{enumerate}

\subsection{Joining a Game}

\begin{enumerate}
    \item Tap "Multiplayer" from the main menu
    \item Tap "Join Game"
    \item Enter the room code provided by the host
    \item Tap "Join"
    \item Wait in the lobby for the host to start the game
\end{enumerate}

\subsection{Multiplayer Gameplay}

\paragraph{Card Distribution:}
\begin{itemize}
    \item Cards are divided among players (simulating MPI data distribution)
    \item Each player sees their own cards and shared areas
    \item Other players' private buffers are hidden
\end{itemize}

\paragraph{Collaboration:}
\begin{itemize}
    \item Players must coordinate to sort all cards
    \item Cards can be passed between players (message passing)
    \item Communication and strategy are key to success
    \item Network latency simulates communication overhead in HPC
\end{itemize}

\paragraph{Winning Together:}
\begin{itemize}
    \item Team wins when all cards across all players are sorted
    \item Final merge phase combines sorted sequences
    \item Victory screen shows team performance metrics
\end{itemize}

%----------------------------------------------------------------------------------------
\section{Settings}
\label{app:a:settings}

\subsection{Game Settings}

\begin{itemize}
    \item \textbf{Default Card Count}: Set preferred number of cards for quick start
    \item \textbf{Difficulty}: Choose default difficulty level
    \item \textbf{Theme}: Light or dark mode
\end{itemize}

\subsection{Audio Settings}

\begin{itemize}
    \item \textbf{Sound Effects}: Enable/disable gameplay sounds
    \item \textbf{Volume}: Adjust volume level
\end{itemize}

\subsection{Display Settings}

\begin{itemize}
    \item \textbf{Card Size}: Adjust card display size
    \item \textbf{Orientation}: Portrait, landscape, or auto
    \item \textbf{Visual Effects}: Enable/disable animations
\end{itemize}

\subsection{Network Settings}

\begin{itemize}
    \item \textbf{Connection Type}: WiFi preferred, mobile data, or either
    \item \textbf{Signaling Server}: Configure custom server (advanced)
\end{itemize}

%----------------------------------------------------------------------------------------
\section{Tips and Strategies}
\label{app:a:tips-strategies}

\subsection{Efficient Sorting Strategies}

\begin{enumerate}
    \item \textbf{Divide and Conquer}: Use buffers to sort smaller subsets of cards
    \item \textbf{Merge Sorted Sequences}: Maintain sorted order within buffers, then merge
    \item \textbf{Minimize Moves}: Think before moving; avoid unnecessary rearrangements
    \item \textbf{Identify Patterns}: Look for already-sorted subsequences
\end{enumerate}

\subsection{Multiplayer Coordination}

\begin{enumerate}
    \item \textbf{Communicate}: Use external communication (voice, text) to coordinate
    \item \textbf{Divide Responsibility}: Assign value ranges to each player
    \item \textbf{Minimize Passing}: Reduce card exchanges to reduce overhead
    \item \textbf{Plan Ahead}: Discuss strategy before starting
\end{enumerate}

%----------------------------------------------------------------------------------------
\section{Troubleshooting}
\label{app:a:troubleshooting}

\subsection{Common Issues and Solutions}

\paragraph{Game Won't Launch:}
\begin{itemize}
    \item Verify Android version meets minimum requirements
    \item Ensure sufficient free storage space
    \item Try restarting device
    \item Reinstall the app if necessary
\end{itemize}

\paragraph{Performance Issues (Lag):}
\begin{itemize}
    \item Reduce number of cards
    \item Disable visual effects in settings
    \item Close background apps
    \item Ensure device is not overheating
\end{itemize}

\paragraph{Cannot Connect to Multiplayer:}
\begin{itemize}
    \item Verify internet connection is active
    \item Check that both players are using the same game version
    \item Try switching between WiFi and mobile data
    \item Verify firewall/network settings allow connections
    \item Re-enter room code carefully
\end{itemize}

\paragraph{Multiplayer Lag or Disconnections:}
\begin{itemize}
    \item Switch to more stable network connection
    \item Move closer to WiFi router
    \item Reduce number of players
    \item Check that signaling server is reachable
\end{itemize}

%----------------------------------------------------------------------------------------
\section{Educational Use}
\label{app:a:educational-use}

\subsection{For Students}

\begin{itemize}
    \item Play single-player mode first to understand mechanics
    \item Observe how buffers help organize your sorting work
    \item Notice how coordination becomes harder with more players
    \item Reflect on how game concepts relate to parallel programming
    \item Experiment with different strategies and compare performance
\end{itemize}

\subsection{For Instructors}

\begin{itemize}
    \item Use as pre-lecture icebreaker to introduce HPC concepts
    \item Facilitate discussion after gameplay about parallel computing
    \item Ask students to identify real-world parallels (threads, processes)
    \item Connect game mechanics to OpenMP and MPI programming models
    \item Use performance metrics to discuss speedup and scalability
\end{itemize}

%----------------------------------------------------------------------------------------
\section{Support and Resources}
\label{app:a:support}

\subsection{Getting Help}

\begin{itemize}
    \item \textbf{GitHub Repository}: \texttt{https://github.com/[username]/hpc-sorting-serious-game}
    \item \textbf{Issue Tracker}: Report bugs and request features on GitHub Issues
    \item \textbf{Documentation}: Additional documentation in the repository wiki
    \item \textbf{Contact}: Szymon Zinkowicz - \texttt{[email]}
\end{itemize}

\subsection{Contributing}

The project is open-source (MIT license). Contributions are welcome:
\begin{itemize}
    \item Report bugs and suggest features via GitHub Issues
    \item Submit pull requests with improvements
    \item Contribute documentation and educational resources
    \item Share feedback from classroom use
\end{itemize}

%----------------------------------------------------------------------------------------


\chapter{Code Listings}
%-----------------------------------------------------------------------
% Appendix B: Code Reference
%-----------------------------------------------------------------------

\section{Source Code Repository}
\label{app:b:repository}

The complete source code for the HPC Sorting Serious Game is available as an open-source project:

\begin{itemize}
    \item \textbf{Repository}: \url{https://github.com/szymonzinkowicz/hpc-sorting-serious-game}
    \item \textbf{License}: MIT License
    \item \textbf{Language}: GDScript (Godot 4.x)
    \item \textbf{Documentation}: README and inline comments
\end{itemize}

\section{Project Structure}
\label{app:b:structure}

The project follows Godot Engine conventions:

\begin{verbatim}
hpc-sorting-serious-game/
+-- project.godot         # Project configuration
+-- scenes/               # Scene files (.tscn)
|   +-- main_menu.tscn
|   +-- singleplayer_game.tscn
|   +-- multiplayer_game.tscn
|   +-- components/       # Reusable UI components
+-- scripts/              # GDScript source files
|   +-- card.gd           # Card logic
|   +-- card_manager.gd   # Card generation/shuffling
|   +-- game_manager.gd   # Game state management
|   +-- networking/       # GDSync integration
+-- assets/               # Textures, fonts, audio
+-- addons/               # GDSync plugin
+-- export_presets.cfg    # Web export configuration
\end{verbatim}

\section{Key Components}
\label{app:b:components}

\subsection{Card System}

The card system handles drag-and-drop interaction and visual state:

\begin{itemize}
    \item \texttt{card.gd}: Individual card with value, drag handling, visual feedback
    \item \texttt{card\_manager.gd}: Generation, shuffling, sorting validation
    \item \texttt{card\_container.gd}: Drop zones (shared container, private buffers)
\end{itemize}

\subsection{Networking}

Multiplayer functionality uses GDSync with HTTP/SSE transport:

\begin{itemize}
    \item \texttt{lobby\_manager.gd}: Room creation, joining, player management
    \item \texttt{state\_sync.gd}: Card movement synchronization across clients
    \item \texttt{http\_transport.gd}: Custom transport layer for web compatibility
\end{itemize}

\subsection{Game Logic}

Core game mechanics:

\begin{itemize}
    \item \texttt{game\_manager.gd}: Mode selection, timer, win condition checking
    \item \texttt{sorting\_validator.gd}: Checks if cards are in correct order
    \item \texttt{performance\_tracker.gd}: Measures time, moves for metrics
\end{itemize}

\section{Configuration}
\label{app:b:configuration}

\subsection{Web Export Settings}

Key settings in \texttt{export\_presets.cfg} for web deployment:

\begin{itemize}
    \item \textbf{Export Mode}: WebAssembly (WASM)
    \item \textbf{Threads}: Single-threaded (for browser compatibility)
    \item \textbf{VRAM Compression}: Disabled (better compatibility)
    \item \textbf{Canvas Resize Policy}: Responsive
\end{itemize}

\subsection{GDSync Configuration}

The HTTP/SSE relay server URL and settings are configured in the project autoload.

\section{Building and Running}
\label{app:b:building}

\subsection{Development}

\begin{enumerate}
    \item Install Godot Engine 4.x
    \item Clone the repository
    \item Open \texttt{project.godot} in Godot Editor
    \item Press F5 to run locally
\end{enumerate}

\subsection{Web Export}

\begin{enumerate}
    \item In Godot: Project $\rightarrow$ Export
    \item Select ``Web'' preset
    \item Click ``Export Project''
    \item Deploy \texttt{index.html} and assets to web server
\end{enumerate}

\subsection{Relay Server}

The HTTP/SSE relay server (Python/aiohttp) is in the \texttt{signaling-server/} directory. See Appendix~\ref{app:c} for API documentation.

%-----------------------------------------------------------------------


\chapter{API Documentation}
%-----------------------------------------------------------------------
% Appendix C: API Documentation
%-----------------------------------------------------------------------

\section{Introduction}
\label{app:c:introduction}

This appendix documents the key APIs and interfaces used in the HPC Sorting Serious Game, including custom classes, GDSync integration, and plugin APIs. This documentation is intended for developers who wish to extend or modify the game.

%-----------------------------------------------------------------------
\section{Core Game Classes}
\label{app:c:core-classes}

\subsection{Card Class}
\label{app:c:card-class}

The Card class represents an individual playing card.

\paragraph{Properties:}

\begin{itemize}
    \item \texttt{card\_value: int} - The numeric value displayed on the card
    \item \texttt{card\_id: int} - Unique identifier for networking synchronization
    \item \texttt{is\_dragging: bool} - Whether the card is currently being dragged
    \item \texttt{is\_selected: bool} - Whether the card is selected
    \item \texttt{original\_position: Vector2} - Position before drag operation
\end{itemize}

\paragraph{Methods:}

\begin{itemize}
    \item \texttt{set\_card\_value(value: int)} - Sets the card's displayed value
    \item \texttt{get\_card\_value() -> int} - Returns the card's value
    \item \texttt{start\_drag()} - Initiates drag operation
    \item \texttt{end\_drag()} - Completes drag operation
    \item \texttt{update\_display()} - Refreshes visual representation
\end{itemize}

\paragraph{Signals:}

\begin{itemize}
    \item \texttt{card\_selected(card: Card)} - Emitted when card is selected
    \item \texttt{card\_dragged(card: Card, position: Vector2)} - Emitted during drag
    \item \texttt{card\_dropped(card: Card, target: Node)} - Emitted on drop
\end{itemize}

%-----------------------------------------------------------------------
\subsection{CardManager Class}
\label{app:c:card-manager-class}

Manages the collection of cards and game logic.

\paragraph{Properties:}

\begin{itemize}
    \item \texttt{cards: Array[Card]} - Array of all cards in the game
    \item \texttt{card\_id\_counter: int} - Counter for unique card IDs
\end{itemize}

\paragraph{Methods:}

\begin{itemize}
    \item \texttt{generate\_cards(count: int) -> Array[Card]} - Creates specified number of cards
    \item \texttt{shuffle\_cards()} - Randomly shuffles the card array
    \item \texttt{is\_sorted(card\_array: Array) -> bool} - Checks if array is sorted
    \item \texttt{check\_global\_sorted() -> bool} - Checks if all cards are sorted
    \item \texttt{get\_card\_by\_id(id: int) -> Card} - Retrieves card by ID
\end{itemize}

\paragraph{Signals:}

\begin{itemize}
    \item \texttt{cards\_generated(count: int)} - Emitted after card generation
    \item \texttt{cards\_shuffled()} - Emitted after shuffling
    \item \texttt{sorting\_complete()} - Emitted when sorting is finished
\end{itemize}

%-----------------------------------------------------------------------
\subsection{Buffer Class}
\label{app:c:buffer-class}

Represents a temporary storage area for cards (thread-local storage simulation).

\paragraph{Properties:}

\begin{itemize}
    \item \texttt{buffer\_id: int} - Unique identifier
    \item \texttt{owner\_id: int} - ID of owning player (-1 for shared)
    \item \texttt{contained\_cards: Array[Card]} - Cards currently in buffer
    \item \texttt{max\_capacity: int} - Maximum number of cards (optional)
\end{itemize}

\paragraph{Methods:}

\begin{itemize}
    \item \texttt{accept\_card(card: Card) -> bool} - Attempts to add card
    \item \texttt{remove\_card(card: Card)} - Removes card from buffer
    \item \texttt{get\_cards() -> Array[Card]} - Returns all contained cards
    \item \texttt{is\_full() -> bool} - Checks if buffer is at capacity
    \item \texttt{sort\_contents()} - Sorts cards within buffer
\end{itemize}

\paragraph{Signals:}

\begin{itemize}
    \item \texttt{card\_added(card: Card)} - Emitted when card added
    \item \texttt{card\_removed(card: Card)} - Emitted when card removed
    \item \texttt{buffer\_sorted()} - Emitted after sorting
\end{itemize}

%-----------------------------------------------------------------------
\section{Multiplayer Classes}
\label{app:c:multiplayer-classes}

\subsection{LobbyManager Class}
\label{app:c:lobby-manager-class}

Manages multiplayer lobby functionality.

\paragraph{Properties:}

\begin{itemize}
    \item \texttt{players: Dictionary} - Map of player\_id to player\_info
    \item \texttt{is\_host: bool} - Whether local player is host
    \item \texttt{room\_code: String} - Unique room identifier
\end{itemize}

\paragraph{Methods:}

\begin{itemize}
    \item \texttt{create\_lobby() -> String} - Creates lobby, returns room code
    \item \texttt{join\_lobby(code: String) -> bool} - Joins existing lobby
    \item \texttt{add\_player(player\_id: int, player\_name: String)} - Adds player to lobby
    \item \texttt{remove\_player(player\_id: int)} - Removes player from lobby
    \item \texttt{start\_game()} - Initiates game start (host only)
\end{itemize}

\paragraph{RPCs:}

\begin{itemize}
    \item \texttt{@rpc rpc\_player\_joined(player\_id: int, player\_name: String)} - Broadcasts player join
    \item \texttt{@rpc rpc\_player\_left(player\_id: int)} - Broadcasts player leave
    \item \texttt{@rpc rpc\_start\_game()} - Signals game start to all clients
\end{itemize}

\paragraph{Signals:}

\begin{itemize}
    \item \texttt{player\_joined(player\_id: int, player\_name: String)} - Player joined
    \item \texttt{player\_left(player\_id: int)} - Player left
    \item \texttt{game\_started()} - Game has started
\end{itemize}

%-----------------------------------------------------------------------
\subsection{NetworkSync Class}
\label{app:c:network-sync-class}

Handles state synchronization across clients.

\paragraph{Properties:}

\begin{itemize}
    \item \texttt{card\_ownership: Dictionary} - Maps card\_id to owner\_player\_id
    \item \texttt{game\_state: Dictionary} - Global game state data
\end{itemize}

\paragraph{Methods:}

\begin{itemize}
    \item \texttt{@GDSync.rpc sync\_move\_card(card\_id, target\_id, position)} - Syncs card movement
    \item \texttt{@GDSync.rpc request\_move\_card(card\_id, target\_id)} - Requests validated move
    \item \texttt{update\_card\_visibility()} - Updates visibility based on ownership
    \item \texttt{broadcast\_game\_state()} - Sends full state to new clients
\end{itemize}

%-----------------------------------------------------------------------
\section{GDSync Framework API}
\label{app:c:gdsync-api}

\subsection{Sync Annotations}
\label{app:c:gdsync-sync}

GDSync provides annotations for automatic synchronization:

\paragraph{@GDSync.sync}

Marks a variable or node for automatic synchronization.

\begin{lstlisting}
# Synchronize a variable
@GDSync.sync(mode=GDSync.SYNC_MODE.AUTHORITY)
var player_position: Vector2

# Synchronize a node
@GDSync.sync(mode=GDSync.SYNC_MODE.REPLICATED)
var replicated_node: Node
\end{lstlisting}

\paragraph{Sync Modes:}

\begin{itemize}
    \item \texttt{AUTHORITY} - Only authority (usually host) can modify
    \item \texttt{REPLICATED} - Replicated to all peers
    \item \texttt{OWNER} - Only owner can modify
\end{itemize}

%-----------------------------------------------------------------------
\subsection{RPC Annotations}
\label{app:c:gdsync-rpc}

\paragraph{@GDSync.rpc}

Marks a function for remote procedure call.

\begin{lstlisting}
# RPC callable by any peer
@GDSync.rpc(call_local=true)
func my_networked_function(arg1: int, arg2: String):
    # Function body
    pass

# Authority-only RPC
@GDSync.rpc(call_local=false, authority_only=true)
func host_only_function():
    # Only host can call this
    pass
\end{lstlisting}

\paragraph{Parameters:}

\begin{itemize}
    \item \texttt{call\_local: bool} - Whether to also execute locally
    \item \texttt{authority\_only: bool} - Restrict to authority
    \item \texttt{reliable: bool} - Use reliable transmission
\end{itemize}

%-----------------------------------------------------------------------
\subsection{Connection Management}
\label{app:c:gdsync-connection}

\paragraph{Methods:}

\begin{itemize}
    \item \texttt{GDSync.create\_server(port: int)} - Creates server
    \item \texttt{GDSync.connect\_to\_server(address: String, port: int)} - Connects to server
    \item \texttt{GDSync.disconnect()} - Disconnects from network
    \item \texttt{GDSync.get\_peers() -> Array} - Returns list of connected peers
\end{itemize}

\paragraph{Signals:}

\begin{itemize}
    \item \texttt{peer\_connected(peer\_id: int)} - Peer connected
    \item \texttt{peer\_disconnected(peer\_id: int)} - Peer disconnected
    \item \texttt{connection\_failed()} - Connection attempt failed
\end{itemize}

%-----------------------------------------------------------------------
\section{Plugin APIs}
\label{app:c:plugin-apis}

\subsection{ToastParty API}
\label{app:c:toastparty-api}

\paragraph{Methods:}

\begin{itemize}
    \item \texttt{ToastParty.show\_toast(message: String, duration: int)}
    \item \texttt{ToastParty.show\_toast\_at\_position(message: String, position: Vector2, duration: int)}
\end{itemize}

\paragraph{Constants:}

\begin{itemize}
    \item \texttt{LENGTH\_SHORT} - 2 seconds
    \item \texttt{LENGTH\_LONG} - 5 seconds
    \item \texttt{TYPE\_INFO} - Informational toast
    \item \texttt{TYPE\_SUCCESS} - Success toast (green)
    \item \texttt{TYPE\_ERROR} - Error toast (red)
    \item \texttt{TYPE\_WARNING} - Warning toast (yellow)
\end{itemize}

%-----------------------------------------------------------------------
\subsection{Logger API}
\label{app:c:logger-api}

\paragraph{Methods:}

\begin{itemize}
    \item \texttt{Logger.debug(message: String)} - Debug-level logging
    \item \texttt{Logger.info(message: String)} - Info-level logging
    \item \texttt{Logger.warn(message: String)} - Warning-level logging
    \item \texttt{Logger.error(message: String)} - Error-level logging
\end{itemize}

\paragraph{Configuration:}

\begin{itemize}
    \item \texttt{Logger.set\_log\_level(level: int)} - Set minimum log level
    \item \texttt{Logger.enable\_file\_logging(path: String)} - Log to file
    \item \texttt{Logger.add\_filter(tag: String)} - Filter log messages
\end{itemize}

%-----------------------------------------------------------------------
\section{Utility APIs}
\label{app:c:utility-apis}

\subsection{Profiler API}
\label{app:c:profiler-api}

\paragraph{Methods:}

\begin{itemize}
    \item \texttt{Profiler.start\_timer(label: String)} - Start timing
    \item \texttt{Profiler.end\_timer(label: String) -> float} - End timing, return elapsed ms
    \item \texttt{Profiler.measure\_fps() -> float} - Get current FPS
    \item \texttt{Profiler.get\_memory\_usage() -> int} - Get memory usage in bytes
\end{itemize}

%-----------------------------------------------------------------------
\subsection{Config Manager API}
\label{app:c:config-api}

\paragraph{Methods:}

\begin{itemize}
    \item \texttt{Config.get\_value(section: String, key: String, default: Variant) -> Variant}
    \item \texttt{Config.set\_value(section: String, key: String, value: Variant)}
    \item \texttt{Config.save\_config()} - Save configuration to disk
    \item \texttt{Config.load\_config()} - Load configuration from disk
\end{itemize}

%-----------------------------------------------------------------------
\section{Event System}
\label{app:c:event-system}

\subsection{Global Events}
\label{app:c:global-events}

The game uses an event bus for decoupled communication.

\paragraph{Usage:}

\begin{lstlisting}
# Emit global event
EventBus.emit("game_started", {"players": 4})

# Listen for event
EventBus.connect("game_started", _on_game_started)

func _on_game_started(data: Dictionary):
    print("Game started with ", data["players"], " players")
\end{lstlisting}

\paragraph{Common Events:}

\begin{itemize}
    \item \texttt{game\_started} - Game has begun
    \item \texttt{game\_ended} - Game has concluded
    \item \texttt{player\_joined} - Player joined lobby
    \item \texttt{player\_left} - Player left game
    \item \texttt{card\_moved} - Card was moved
    \item \texttt{sorting\_complete} - Sorting finished
\end{itemize}

%-----------------------------------------------------------------------
\section{Extension Points}
\label{app:c:extension-points}

\subsection{Custom Game Modes}
\label{app:c:custom-game-modes}

To create custom game modes, extend the \texttt{GameMode} base class:

\begin{lstlisting}
extends GameMode
class_name MyCustomMode

func _init():
    mode_name = "My Custom Mode"
    mode_description = "A custom game variant"

func setup_game():
    # Custom initialization logic
    pass

func validate_move(card: Card, target: Node) -> bool:
    # Custom move validation
    return true

func check_win_condition() -> bool:
    # Custom win condition
    return false
\end{lstlisting}

%-----------------------------------------------------------------------
\subsection{Custom Card Types}
\label{app:c:custom-card-types}

Extend the Card class for custom card behavior:

\begin{lstlisting}
extends Card
class_name SpecialCard

@export var special_ability: String = ""

func activate_ability():
    match special_ability:
        "double_value":
            card_value *= 2
        "wildcard":
            # Can match any value
            pass
    update_display()
\end{lstlisting}

%-----------------------------------------------------------------------
\section{Development Tools}
\label{app:c:dev-tools}

\subsection{Debug Commands}
\label{app:c:debug-commands}

Console commands for development:

\begin{itemize}
    \item \texttt{/spawn\_cards <count>} - Generate cards
    \item \texttt{/sort\_all} - Automatically sort all cards
    \item \texttt{/reset\_game} - Reset game state
    \item \texttt{/show\_network\_stats} - Display network statistics
    \item \texttt{/toggle\_debug\_ui} - Show/hide debug information
\end{itemize}

%-----------------------------------------------------------------------


\chapter{Study Materials}
%-----------------------------------------------------------------------
% Appendix D: Study Materials
%-----------------------------------------------------------------------

\section{Introduction}
\label{app:d:introduction}

This appendix provides educational materials to help students and instructors connect the HPC Sorting Serious Game to formal parallel computing concepts. It includes learning objectives, discussion questions, exercises, and connections to OpenMP programming and sequential vs. parallel execution.

%-----------------------------------------------------------------------
\section{Learning Objectives}
\label{app:d:learning-objectives}

\subsection{Foundational Concepts}

After playing the game, students should be able to:

\begin{enumerate}
    \item \textbf{Define Parallelism}: Explain what it means to perform multiple tasks at the same time
    \item \textbf{Understand Teamwork Benefits}: Describe how working together can finish tasks faster
    \item \textbf{Recognize Coordination Challenges}: Explain that working together requires communication and planning
    \item \textbf{Appreciate Fair Work Distribution}: Understand why it's important that everyone has about the same amount of work
    \item \textbf{Identify When Teamwork Helps}: Recognize that some tasks benefit from multiple workers more than others
\end{enumerate}

\subsection{Sequential Execution Concepts (Single-Player Mode)}

Students should understand:

\begin{enumerate}
    \item A single process executes all operations sequentially
    \item No parallelism or concurrency
    \item Simple but potentially slower for large datasets
    \item Baseline for comparison with parallel execution
    \item All data is accessible directly without coordination
\end{enumerate}

\subsection{OpenMP Concepts (Multiplayer Mode)}

Students should understand:

\begin{enumerate}
    \item Multiple threads share a common memory space
    \item Each thread can have private (local) data in buffers
    \item Threads work collaboratively on shared data
    \item Synchronization occurs through shared container access
    \item Work is divided among threads for parallel execution
    \item Final merge combines results from parallel execution
\end{enumerate}

%-----------------------------------------------------------------------
\section{Discussion Questions}
\label{app:d:discussion-questions}

\subsection{Pre-Game Questions}

Use these questions before gameplay to activate prior knowledge:

\begin{enumerate}
    \item Have you ever worked on a group project? How did you divide the work?
    \item What challenges arise when multiple people work on the same task simultaneously?
    \item How is sorting a deck of cards different when one person does it versus multiple people?
    \item Can you think of tasks that benefit from having multiple workers? Which tasks don't?
\end{enumerate}

\subsection{Post-Game Discussion: Single-Player Mode}

After single-player gameplay:

\begin{enumerate}
    \item How did you use the buffer zones? Why were they helpful?
    \item What strategy did you use to sort the cards efficiently?
    \item How would your strategy change with 10 cards versus 200 cards?
    \item If you could have multiple versions of yourself working simultaneously, how would you coordinate?
    \item What parallels can you draw between the buffers and computer memory?
\end{enumerate}

\subsection{Post-Game Discussion: Multiplayer Mode}

After multiplayer gameplay:

\begin{enumerate}
    \item How did you communicate and coordinate with other players?
    \item What happened when two players tried to work on the same cards?
    \item Did you notice delays when passing cards between players? Why might that occur?
    \item How did you divide the work? Was it balanced?
    \item What would happen if one player had many more cards than others?
    \item How is this different from single-player mode?
\end{enumerate}

\subsection{Connecting to HPC Concepts}

Bridge from game to formal concepts:

\begin{enumerate}
    \item How does single-player mode relate to shared-memory parallel programming?
    \item How does multiplayer mode relate to distributed-memory parallel programming?
    \item What real-world computing problems benefit from parallelism?
    \item What are the trade-offs between shared-memory and distributed-memory approaches?
    \item When might communication costs outweigh parallelism benefits?
\end{enumerate}

%-----------------------------------------------------------------------
\section{Classroom Activities}
\label{app:d:classroom-activities}

\subsection{Activity 1: Strategy Competition}

\paragraph{Objective:} Compare different ways to sort cards and see which works best.

\paragraph{Instructions:}
\begin{enumerate}
    \item Divide class into teams of 2--3 students
    \item Each team plays the game with the same settings (e.g., 50 cards, 3 players)
    \item Teams write down their strategy: How will you divide the work? Will you help each other?
    \item Compare completion times between teams
    \item Discuss which strategies worked well and why
\end{enumerate}

\paragraph{Discussion:}
\begin{itemize}
    \item Which strategy was fastest? Why do you think it worked better?
    \item Did teams that divided work evenly finish faster?
    \item What problems did your team face during the game?
    \item If you played again, what would you do differently?
\end{itemize}

%-----------------------------------------------------------------------
\subsection{Activity 2: Does More Players Mean Faster?}

\paragraph{Objective:} Investigate how adding more players affects completion time.

\paragraph{Instructions:}
\begin{enumerate}
    \item Students play multiplayer mode with different numbers of players:
          \begin{itemize}
              \item 2 players, 50 cards
              \item 3 players, 50 cards
              \item 4 players, 50 cards
          \end{itemize}
    \item Record completion times for each configuration
    \item Create a simple chart showing number of players vs. time
    \item Discuss the results
\end{enumerate}

\paragraph{Analysis Questions:}
\begin{itemize}
    \item Does doubling the number of players cut the time in half?
    \item What makes it hard to keep getting faster as you add more players?
    \item Can you think of tasks where adding more people doesn't help much?
\end{itemize}

%-----------------------------------------------------------------------
\subsection{Activity 3: Working Together vs. Working Separately}

\paragraph{Objective:} Understand when it's better to work together versus work separately.

\paragraph{Instructions:}
\begin{enumerate}
    \item Play multiplayer mode with different strategies:
          \begin{itemize}
              \item \textbf{Strategy A}: Each player sorts their own cards independently, then combine at the end
              \item \textbf{Strategy B}: Players frequently pass cards to each other to help balance the work
          \end{itemize}
    \item Record your time and count how many times cards were passed between players
    \item Compare which strategy was faster
\end{enumerate}

\paragraph{Discussion:}
\begin{itemize}
    \item Which strategy was faster?
    \item Does passing cards between players take time? How much?
    \item When is it worth the time to pass cards to another player?
    \item Can you think of real-world situations where coordination takes more time than it saves?
\end{itemize}

%-----------------------------------------------------------------------
\section{Exercises}
\label{app:d:exercises}

These exercises are designed for students aged 12-16 with no prior programming experience. They focus on understanding concepts through the game rather than technical implementation.

\subsection{Exercise 1: Comparing Single and Multiple Players}

\paragraph{Task:}

Play the game twice with the same number of cards (e.g., 30 cards):
\begin{itemize}
    \item First time: Single-player mode
    \item Second time: Multiplayer mode with 2-3 friends
\end{itemize}

\paragraph{Questions:}
\begin{enumerate}
    \item Which mode finished faster?
    \item Was multiplayer always faster? If not, why not?
    \item What made working together harder than working alone?
    \item If you added more players, would it keep getting faster forever? Why or why not?
\end{enumerate}

%-----------------------------------------------------------------------
\subsection{Exercise 2: Finding the Best Team Size}

\paragraph{Task:}

Try sorting 50 cards with different numbers of players and record your time:
\begin{itemize}
    \item 1 player (you alone)
    \item 2 players
    \item 3 players
    \item 4 players
\end{itemize}

\paragraph{Questions:}
\begin{enumerate}
    \item Which team size was fastest?
    \item Did more players always mean faster sorting?
    \item What problems happened when you had too many players?
    \item Can you think of real-world situations where having too many helpers makes things slower?
\end{enumerate}

%-----------------------------------------------------------------------
\subsection{Exercise 3: Dividing the Work}

\paragraph{Task:}

Play multiplayer mode and try two different strategies:
\begin{itemize}
    \item \textbf{Strategy A}: Each player takes an equal number of cards at the start and sorts them
    \item \textbf{Strategy B}: Players help whoever has the most cards remaining
\end{itemize}

\paragraph{Questions:}
\begin{enumerate}
    \item Which strategy felt easier to coordinate?
    \item Which strategy finished faster?
    \item What happens if one player is much faster than the others?
    \item How does this relate to group projects in school?
\end{enumerate}

%-----------------------------------------------------------------------
% Note: Advanced exercises involving complexity analysis, OpenMP code implementation,
% and performance modeling are not appropriate for the target age group (12-16 years old).
% Those concepts should be introduced in undergraduate computer science courses.

%-----------------------------------------------------------------------
\section{Conceptual Mappings}
\label{app:d:conceptual-mappings}

\subsection{Game to OpenMP}

\begin{table}[htbp]
    \centering
    \caption{Detailed game-to-OpenMP concept mapping}
    \begin{tabular}{@{}p{4cm}p{5cm}p{5cm}@{}}
        \toprule
        \textbf{Game Element} & \textbf{OpenMP Concept} & \textbf{Code Example}          \\
        \midrule
        Single-player session & Parallel region         & \texttt{\#pragma omp parallel} \\
        Main card container   & Shared array            & Global array variable          \\
        Buffer zones          & Thread-private storage  & \texttt{private(buffer)}       \\
        No forced turn-taking & Independent threads     & No explicit synchronization    \\
        Sorting within buffer & Local computation       & Serial code in parallel region \\
        Merging back to main  & Reduction/merge         & \texttt{\#pragma omp critical} \\
        Completion check      & Barrier synchronization & \texttt{\#pragma omp barrier}  \\
        \bottomrule
    \end{tabular}
\end{table}

%-----------------------------------------------------------------------
% Table D.2: Game-to-MPI mapping REMOVED
% The game does not implement MPI distributed-memory concepts.
% Only OpenMP shared-memory parallelism is covered.
% See Chapter 8 (Future Work) for potential MPI extension.

%-----------------------------------------------------------------------
\section{Assessment Rubric}
\label{app:d:assessment-rubric}

For instructors wishing to grade student understanding:

\subsection{Conceptual Understanding (40 points)}

\begin{itemize}
    \item (10 pts) Can identify shared-memory vs. distributed-memory paradigms
    \item (10 pts) Understands role of communication in parallel performance
    \item (10 pts) Can explain speedup and scalability concepts
    \item (10 pts) Recognizes load balancing importance
\end{itemize}

\subsection{Practical Application (30 points)}

\begin{itemize}
    \item (15 pts) Designs reasonable parallel sorting strategy
    \item (15 pts) Implements parallel sorting code (OpenMP)
\end{itemize}

\subsection{Analysis and Reflection (30 points)}

\begin{itemize}
    \item (10 pts) Analyzes performance data from experiments
    \item (10 pts) Connects game experience to real HPC systems
    \item (10 pts) Reflects on limitations and trade-offs
\end{itemize}

%-----------------------------------------------------------------------
\section{Additional Resources}
\label{app:d:additional-resources}

\subsection{Recommended Readings}

\begin{enumerate}
    \item Pacheco, P. (2011). \textit{An Introduction to Parallel Programming}. Morgan Kaufmann.
    \item OpenMP Architecture Review Board. (2021). \textit{OpenMP Application Programming Interface}.
    \item Chapman, B., Jost, G., \& Van Der Pas, R. (2007). \textit{Using OpenMP: Portable Shared Memory Parallel Programming}. MIT Press.
\end{enumerate}

\subsection{Online Resources}

\begin{itemize}
    \item OpenMP Tutorials: \texttt{https://www.openmp.org/resources/tutorials-articles/}
    \item Parallel Computing Course Materials: \texttt{https://www.cs.cmu.edu/~15418/}
    \item Lawrence Livermore National Lab OpenMP Tutorial: \texttt{https://hpc.llnl.gov/tuts/openMP/}
\end{itemize}

\subsection{Hands-On Practice}

\begin{itemize}
    \item Try the game with different configurations
    \item Implement actual parallel sorting code
    \item Profile and optimize parallel programs
    \item Experiment with HPC clusters (if available)
\end{itemize}

%-----------------------------------------------------------------------
\section{Instructor Notes}
\label{app:d:instructor-notes}

\subsection{Integration with Curriculum}

\paragraph{As Pre-Lecture Activity (15--20 minutes):}
\begin{itemize}
    \item Students play game before formal HPC lecture
    \item Generates curiosity and questions
    \item Provides concrete reference points for abstract concepts
    \item Follow with discussion connecting gameplay to course material
\end{itemize}

\paragraph{As Lab Exercise (50--90 minutes):}
\begin{itemize}
    \item Structured activities with data collection
    \item Analysis of performance metrics
    \item Written reflection connecting to theory
    \item Implementation of actual parallel code based on strategies
\end{itemize}

\paragraph{As Homework/Assessment:}
\begin{itemize}
    \item Students play independently, record observations
    \item Answer reflection questions
    \item Design parallel algorithms inspired by gameplay
    \item Optional: Implement and benchmark solutions
\end{itemize}

\subsection{Common Student Misconceptions}

Be prepared to address:

\begin{enumerate}
    \item \textbf{"More processes/threads always means faster"}: Discuss communication overhead and Amdahl's Law
    \item \textbf{"Shared memory means no synchronization needed"}: Explain race conditions and critical sections
    \item \textbf{"Message passing is always slower"}: Discuss scalability benefits of distributed memory
    \item \textbf{"The game perfectly simulates HPC"}: Clarify simplifications and abstractions
\end{enumerate}

%-----------------------------------------------------------------------


%-----------------------------------------------------------------------

\end{document}
