%-----------------------------------------------------------------------
% Appendix B: Code Reference
%-----------------------------------------------------------------------

\section{Source Code Repository}
\label{app:b:repository}

The complete source code for the HPC Sorting Serious Game is available as an open-source project:

\begin{itemize}
    \item \textbf{Repository}: \url{https://github.com/szymonzinkowicz/hpc-sorting-serious-game}
    \item \textbf{License}: MIT License
    \item \textbf{Language}: GDScript (Godot 4.x)
    \item \textbf{Documentation}: README and inline comments
\end{itemize}

\section{Project Structure}
\label{app:b:structure}

The project follows Godot Engine conventions:

\begin{verbatim}
hpc-sorting-serious-game/
+-- project.godot         # Project configuration
+-- scenes/               # Scene files (.tscn)
|   +-- main_menu.tscn
|   +-- singleplayer_game.tscn
|   +-- multiplayer_game.tscn
|   +-- components/       # Reusable UI components
+-- scripts/              # GDScript source files
|   +-- card.gd           # Card logic
|   +-- card_manager.gd   # Card generation/shuffling
|   +-- game_manager.gd   # Game state management
|   +-- networking/       # GDSync integration
+-- assets/               # Textures, fonts, audio
+-- addons/               # GDSync plugin
+-- export_presets.cfg    # Web export configuration
\end{verbatim}

\section{Key Components}
\label{app:b:components}

\subsection{Card System}

The card system handles drag-and-drop interaction and visual state:

\begin{itemize}
    \item \texttt{card.gd}: Individual card with value, drag handling, visual feedback
    \item \texttt{card\_manager.gd}: Generation, shuffling, sorting validation
    \item \texttt{card\_container.gd}: Drop zones (shared container, private buffers)
\end{itemize}

\subsection{Networking}

Multiplayer functionality uses GDSync with HTTP/SSE transport:

\begin{itemize}
    \item \texttt{lobby\_manager.gd}: Room creation, joining, player management
    \item \texttt{state\_sync.gd}: Card movement synchronization across clients
    \item \texttt{http\_transport.gd}: Custom transport layer for web compatibility
\end{itemize}

\subsection{Game Logic}

Core game mechanics:

\begin{itemize}
    \item \texttt{game\_manager.gd}: Mode selection, timer, win condition checking
    \item \texttt{sorting\_validator.gd}: Checks if cards are in correct order
    \item \texttt{performance\_tracker.gd}: Measures time, moves for metrics
\end{itemize}

\section{Configuration}
\label{app:b:configuration}

\subsection{Web Export Settings}

Key settings in \texttt{export\_presets.cfg} for web deployment:

\begin{itemize}
    \item \textbf{Export Mode}: WebAssembly (WASM)
    \item \textbf{Threads}: Single-threaded (for browser compatibility)
    \item \textbf{VRAM Compression}: Disabled (better compatibility)
    \item \textbf{Canvas Resize Policy}: Responsive
\end{itemize}

\subsection{GDSync Configuration}

The HTTP/SSE relay server URL and settings are configured in the project autoload.

\section{Building and Running}
\label{app:b:building}

\subsection{Development}

\begin{enumerate}
    \item Install Godot Engine 4.x
    \item Clone the repository
    \item Open \texttt{project.godot} in Godot Editor
    \item Press F5 to run locally
\end{enumerate}

\subsection{Web Export}

\begin{enumerate}
    \item In Godot: Project $\rightarrow$ Export
    \item Select ``Web'' preset
    \item Click ``Export Project''
    \item Deploy \texttt{index.html} and assets to web server
\end{enumerate}

\subsection{Relay Server}

The HTTP/SSE relay server (Python/aiohttp) is in the \texttt{signaling-server/} directory. See Appendix~\ref{app:c} for API documentation.

%-----------------------------------------------------------------------
