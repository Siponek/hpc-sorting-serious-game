%-----------------------------------------------------------------------
% Appendix D: Study Materials
%-----------------------------------------------------------------------

\section{Introduction}
\label{app:d:introduction}

This appendix provides educational materials to help students and instructors connect the HPC Sorting Serious Game to formal parallel computing concepts. It includes learning objectives, discussion questions, exercises, and connections to OpenMP programming and sequential vs. parallel execution.

%-----------------------------------------------------------------------
\section{Learning Objectives}
\label{app:d:learning-objectives}

\subsection{Foundational Concepts}

After playing the game, students should be able to:

\begin{enumerate}
    \item \textbf{Define Parallelism}: Explain what it means to perform multiple tasks at the same time
    \item \textbf{Understand Teamwork Benefits}: Describe how working together can finish tasks faster
    \item \textbf{Recognize Coordination Challenges}: Explain that working together requires communication and planning
    \item \textbf{Appreciate Fair Work Distribution}: Understand why it's important that everyone has about the same amount of work
    \item \textbf{Identify When Teamwork Helps}: Recognize that some tasks benefit from multiple workers more than others
\end{enumerate}

\subsection{Sequential Execution Concepts (Single-Player Mode)}

Students should understand:

\begin{enumerate}
    \item A single process executes all operations sequentially
    \item No parallelism or concurrency
    \item Simple but potentially slower for large datasets
    \item Baseline for comparison with parallel execution
    \item All data is accessible directly without coordination
\end{enumerate}

\subsection{OpenMP Concepts (Multiplayer Mode)}

Students should understand:

\begin{enumerate}
    \item Multiple threads share a common memory space
    \item Each thread can have private (local) data in buffers
    \item Threads work collaboratively on shared data
    \item Synchronization occurs through shared container access
    \item Work is divided among threads for parallel execution
    \item Final merge combines results from parallel execution
\end{enumerate}

%-----------------------------------------------------------------------
\section{Discussion Questions}
\label{app:d:discussion-questions}

\subsection{Pre-Game Questions}

Use these questions before gameplay to activate prior knowledge:

\begin{enumerate}
    \item Have you ever worked on a group project? How did you divide the work?
    \item What challenges arise when multiple people work on the same task simultaneously?
    \item How is sorting a deck of cards different when one person does it versus multiple people?
    \item Can you think of tasks that benefit from having multiple workers? Which tasks don't?
\end{enumerate}

\subsection{Post-Game Discussion: Single-Player Mode}

After single-player gameplay:

\begin{enumerate}
    \item How did you use the buffer zones? Why were they helpful?
    \item What strategy did you use to sort the cards efficiently?
    \item How would your strategy change with 10 cards versus 200 cards?
    \item If you could have multiple versions of yourself working simultaneously, how would you coordinate?
    \item What parallels can you draw between the buffers and computer memory?
\end{enumerate}

\subsection{Post-Game Discussion: Multiplayer Mode}

After multiplayer gameplay:

\begin{enumerate}
    \item How did you communicate and coordinate with other players?
    \item What happened when two players tried to work on the same cards?
    \item Did you notice delays when passing cards between players? Why might that occur?
    \item How did you divide the work? Was it balanced?
    \item What would happen if one player had many more cards than others?
    \item How is this different from single-player mode?
\end{enumerate}

\subsection{Connecting to HPC Concepts}

Bridge from game to formal concepts:

\begin{enumerate}
    \item How does single-player mode relate to shared-memory parallel programming?
    \item How does multiplayer mode relate to distributed-memory parallel programming?
    \item What real-world computing problems benefit from parallelism?
    \item What are the trade-offs between shared-memory and distributed-memory approaches?
    \item When might communication costs outweigh parallelism benefits?
\end{enumerate}

%-----------------------------------------------------------------------
\section{Classroom Activities}
\label{app:d:classroom-activities}

\subsection{Activity 1: Strategy Competition}

\paragraph{Objective:} Compare different ways to sort cards and see which works best.

\paragraph{Instructions:}
\begin{enumerate}
    \item Divide class into teams of 2--3 students
    \item Each team plays the game with the same settings (e.g., 50 cards, 3 players)
    \item Teams write down their strategy: How will you divide the work? Will you help each other?
    \item Compare completion times between teams
    \item Discuss which strategies worked well and why
\end{enumerate}

\paragraph{Discussion:}
\begin{itemize}
    \item Which strategy was fastest? Why do you think it worked better?
    \item Did teams that divided work evenly finish faster?
    \item What problems did your team face during the game?
    \item If you played again, what would you do differently?
\end{itemize}

%-----------------------------------------------------------------------
\subsection{Activity 2: Does More Players Mean Faster?}

\paragraph{Objective:} Investigate how adding more players affects completion time.

\paragraph{Instructions:}
\begin{enumerate}
    \item Students play multiplayer mode with different numbers of players:
          \begin{itemize}
              \item 2 players, 50 cards
              \item 3 players, 50 cards
              \item 4 players, 50 cards
          \end{itemize}
    \item Record completion times for each configuration
    \item Create a simple chart showing number of players vs. time
    \item Discuss the results
\end{enumerate}

\paragraph{Analysis Questions:}
\begin{itemize}
    \item Does doubling the number of players cut the time in half?
    \item What makes it hard to keep getting faster as you add more players?
    \item Can you think of tasks where adding more people doesn't help much?
\end{itemize}

%-----------------------------------------------------------------------
\subsection{Activity 3: Working Together vs. Working Separately}

\paragraph{Objective:} Understand when it's better to work together versus work separately.

\paragraph{Instructions:}
\begin{enumerate}
    \item Play multiplayer mode with different strategies:
          \begin{itemize}
              \item \textbf{Strategy A}: Each player sorts their own cards independently, then combine at the end
              \item \textbf{Strategy B}: Players frequently pass cards to each other to help balance the work
          \end{itemize}
    \item Record your time and count how many times cards were passed between players
    \item Compare which strategy was faster
\end{enumerate}

\paragraph{Discussion:}
\begin{itemize}
    \item Which strategy was faster?
    \item Does passing cards between players take time? How much?
    \item When is it worth the time to pass cards to another player?
    \item Can you think of real-world situations where coordination takes more time than it saves?
\end{itemize}

%-----------------------------------------------------------------------
\section{Exercises}
\label{app:d:exercises}

These exercises are designed for students aged 12-16 with no prior programming experience. They focus on understanding concepts through the game rather than technical implementation.

\subsection{Exercise 1: Comparing Single and Multiple Players}

\paragraph{Task:}

Play the game twice with the same number of cards (e.g., 30 cards):
\begin{itemize}
    \item First time: Single-player mode
    \item Second time: Multiplayer mode with 2-3 friends
\end{itemize}

\paragraph{Questions:}
\begin{enumerate}
    \item Which mode finished faster?
    \item Was multiplayer always faster? If not, why not?
    \item What made working together harder than working alone?
    \item If you added more players, would it keep getting faster forever? Why or why not?
\end{enumerate}

%-----------------------------------------------------------------------
\subsection{Exercise 2: Finding the Best Team Size}

\paragraph{Task:}

Try sorting 50 cards with different numbers of players and record your time:
\begin{itemize}
    \item 1 player (you alone)
    \item 2 players
    \item 3 players
    \item 4 players
\end{itemize}

\paragraph{Questions:}
\begin{enumerate}
    \item Which team size was fastest?
    \item Did more players always mean faster sorting?
    \item What problems happened when you had too many players?
    \item Can you think of real-world situations where having too many helpers makes things slower?
\end{enumerate}

%-----------------------------------------------------------------------
\subsection{Exercise 3: Dividing the Work}

\paragraph{Task:}

Play multiplayer mode and try two different strategies:
\begin{itemize}
    \item \textbf{Strategy A}: Each player takes an equal number of cards at the start and sorts them
    \item \textbf{Strategy B}: Players help whoever has the most cards remaining
\end{itemize}

\paragraph{Questions:}
\begin{enumerate}
    \item Which strategy felt easier to coordinate?
    \item Which strategy finished faster?
    \item What happens if one player is much faster than the others?
    \item How does this relate to group projects in school?
\end{enumerate}

%-----------------------------------------------------------------------
% Note: Advanced exercises involving complexity analysis, OpenMP code implementation,
% and performance modeling are not appropriate for the target age group (12-16 years old).
% Those concepts should be introduced in undergraduate computer science courses.

%-----------------------------------------------------------------------
\section{Conceptual Mappings}
\label{app:d:conceptual-mappings}

\subsection{Game to OpenMP}

\begin{table}[htbp]
    \centering
    \caption{Detailed game-to-OpenMP concept mapping}
    \begin{tabular}{@{}p{4cm}p{5cm}p{5cm}@{}}
        \toprule
        \textbf{Game Element} & \textbf{OpenMP Concept} & \textbf{Code Example}          \\
        \midrule
        Single-player session & Parallel region         & \texttt{\#pragma omp parallel} \\
        Main card container   & Shared array            & Global array variable          \\
        Buffer zones          & Thread-private storage  & \texttt{private(buffer)}       \\
        No forced turn-taking & Independent threads     & No explicit synchronization    \\
        Sorting within buffer & Local computation       & Serial code in parallel region \\
        Merging back to main  & Reduction/merge         & \texttt{\#pragma omp critical} \\
        Completion check      & Barrier synchronization & \texttt{\#pragma omp barrier}  \\
        \bottomrule
    \end{tabular}
\end{table}

%-----------------------------------------------------------------------
% Table D.2: Game-to-MPI mapping REMOVED
% The game does not implement MPI distributed-memory concepts.
% Only OpenMP shared-memory parallelism is covered.
% See Chapter 8 (Future Work) for potential MPI extension.

%-----------------------------------------------------------------------
\section{Assessment Rubric}
\label{app:d:assessment-rubric}

For instructors wishing to grade student understanding:

\subsection{Conceptual Understanding (40 points)}

\begin{itemize}
    \item (10 pts) Can identify shared-memory vs. distributed-memory paradigms
    \item (10 pts) Understands role of communication in parallel performance
    \item (10 pts) Can explain speedup and scalability concepts
    \item (10 pts) Recognizes load balancing importance
\end{itemize}

\subsection{Practical Application (30 points)}

\begin{itemize}
    \item (15 pts) Designs reasonable parallel sorting strategy
    \item (15 pts) Implements parallel sorting code (OpenMP)
\end{itemize}

\subsection{Analysis and Reflection (30 points)}

\begin{itemize}
    \item (10 pts) Analyzes performance data from experiments
    \item (10 pts) Connects game experience to real HPC systems
    \item (10 pts) Reflects on limitations and trade-offs
\end{itemize}

%-----------------------------------------------------------------------
\section{Additional Resources}
\label{app:d:additional-resources}

\subsection{Recommended Readings}

\begin{enumerate}
    \item Pacheco, P. (2011). \textit{An Introduction to Parallel Programming}. Morgan Kaufmann.
    \item OpenMP Architecture Review Board. (2021). \textit{OpenMP Application Programming Interface}.
    \item Chapman, B., Jost, G., \& Van Der Pas, R. (2007). \textit{Using OpenMP: Portable Shared Memory Parallel Programming}. MIT Press.
\end{enumerate}

\subsection{Online Resources}

\begin{itemize}
    \item OpenMP Tutorials: \texttt{https://www.openmp.org/resources/tutorials-articles/}
    \item Parallel Computing Course Materials: \texttt{https://www.cs.cmu.edu/~15418/}
    \item Lawrence Livermore National Lab OpenMP Tutorial: \texttt{https://hpc.llnl.gov/tuts/openMP/}
\end{itemize}

\subsection{Hands-On Practice}

\begin{itemize}
    \item Try the game with different configurations
    \item Implement actual parallel sorting code
    \item Profile and optimize parallel programs
    \item Experiment with HPC clusters (if available)
\end{itemize}

%-----------------------------------------------------------------------
\section{Instructor Notes}
\label{app:d:instructor-notes}

\subsection{Integration with Curriculum}

\paragraph{As Pre-Lecture Activity (15--20 minutes):}
\begin{itemize}
    \item Students play game before formal HPC lecture
    \item Generates curiosity and questions
    \item Provides concrete reference points for abstract concepts
    \item Follow with discussion connecting gameplay to course material
\end{itemize}

\paragraph{As Lab Exercise (50--90 minutes):}
\begin{itemize}
    \item Structured activities with data collection
    \item Analysis of performance metrics
    \item Written reflection connecting to theory
    \item Implementation of actual parallel code based on strategies
\end{itemize}

\paragraph{As Homework/Assessment:}
\begin{itemize}
    \item Students play independently, record observations
    \item Answer reflection questions
    \item Design parallel algorithms inspired by gameplay
    \item Optional: Implement and benchmark solutions
\end{itemize}

\subsection{Common Student Misconceptions}

Be prepared to address:

\begin{enumerate}
    \item \textbf{"More processes/threads always means faster"}: Discuss communication overhead and Amdahl's Law
    \item \textbf{"Shared memory means no synchronization needed"}: Explain race conditions and critical sections
    \item \textbf{"Message passing is always slower"}: Discuss scalability benefits of distributed memory
    \item \textbf{"The game perfectly simulates HPC"}: Clarify simplifications and abstractions
\end{enumerate}

%-----------------------------------------------------------------------
